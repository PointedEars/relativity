% vim:set fileencoding=utf-8 tabstop=2 shiftwidth=2 softtabstop=2 expandtab:
%% Page layout
% Scientific report, European style (A4) (from KOMA-Script)
% BCOR: binding correction
% DIV=calc: Calculate page spread now (recalculated below)
% pagesize: Add page info to (PDF/PS) output
% parskip=half: Do not indent paragraphs, add one half line spacing instead
%   (FFHS requirement)
% Default font size: 10pt (FFHS requirement)
% BCOR=1cm too much?
\documentclass[pagesize,headsepline,10pt,parskip=half]{scrreprt}

% Line spacing should be 1.5 times the line height (FFHS requirement)
\usepackage{setspace}
\onehalfspacing{}

% float compatibility for KOMA-Script
\usepackage{scrhack}

% Use English orthography and hyphenation by default
\usepackage[latin,italian,english]{babel}

% Allow using non-ASCII characters verbatim; disable for LuaLaTeX
% \usepackage[utf8]{inputenc}

% Special characters
\usepackage{textcomp}

% Use fonts having non-ASCII characters; disable fontenc for LuaLaTeX
% \usepackage[T1]{fontenc}
\usepackage{fontspec}
\newfontfamily\DejaSans{DejaVu Sans}

% User-defined English hyphenation
\hyphenation{InfoWorld}

% Use language-specific quotes
\usepackage[autostyle,english=american]{csquotes}
\DeclareQuoteAlias{italian}{latin}

% Advanced Computer Modern fonts
\usepackage{lmodern}

% Allow strike-through with \sout, keep italic for \emph
\usepackage[normalem]{ulem}

% Use medieval numbers except in math mode
% \usepackage{hfoldsty}

% Use sans-serif font ('Arial') by default for headings and normal text;
% disable helvet for LuaLaTeX
% (FFHS requirement)
\renewcommand{\familydefault}{\sfdefault}
\usepackage{mathptmx}
%\usepackage[scaled=.90]{helvet}
\usepackage{courier}

% Improved typography, like hyphenation in words with non-ASCII characters,
% see also http://homepage.ruhr-uni-bochum.de/georg.verweyen/pakete.html
\usepackage[babel]{microtype}

% Number also \subsubsection, but not \paragraph and below
\setcounter{secnumdepth}{3}

% Page heading and footer
\usepackage{scrpage2}
\pagestyle{scrheadings}
\automark[section]{chapter}
% heading on the top inner margin only
\ohead[]{\headmark}
\chead[]{}
% page number on the bottom outer margin only
\ofoot[\pagemark]{\pagemark}
\cfoot[]{}

% Support for list of acronyms
\usepackage[footnote,nohyperlinks,withpage]{acronym}

% References: can use section names
\usepackage{nameref}

% References: generate hyperlinks
\usepackage[plainpages=false]{hyperref}

% References style (default: 'numerical')
\usepackage[backend=biber,
style=authoryear-ibid,
maxcitenames=1,
maxbibnames=3]{biblatex}
\defbibheading{lit}{\chapter*{References}\markboth{References}{References}}
% References: bibliography database
\addbibresource{main.bib}

% prints author names as small caps
\renewcommand{\mkbibnamefirst}[1]{\textsc{#1}}
\renewcommand{\mkbibnamelast}[1]{\textsc{#1}}
\renewcommand{\mkbibnameprefix}[1]{\textsc{#1}}
\renewcommand{\mkbibnameaffix}[1]{\textsc{#1}}

% References: use English ordinal numbers
\usepackage[super]{nth}

% Automatically use teletype for \url argument, use content verbatim
\usepackage{url}
\urlstyle{tt}

% Less vertical spacing between list items (in `compactitem' environment)
\usepackage{paralist}

% Multi-row table cells
\usepackage{multirow}

% Support for horizontal rules in tables (professional style)
\usepackage{booktabs}

% Footnotes in tables
\usepackage{threeparttable}

% Tables across pages (for results)
\usepackage{longtable}

% Word wrap in table columns (calculate p width)
\usepackage{calc}

% Automatic column stretching
% \usepackage{tabularx}

% Stretched tables across pages (for results); requires longtable
% tabularx
\usepackage{ltxtable}

% Improved table formatting
\usepackage{array}

% Support for including PDFs
\usepackage{pdfpages}

% Support for figures
\usepackage{graphicx}

\usepackage{amsthm}
\newtheorem{mydef}{Definition}

\usepackage[fleqn]{amsmath}
\newlength{\normalparindent}
\AtBeginDocument{\setlength{\normalparindent}{\parindent}}
\newcommand{\longintertext}[1]{%
  \intertext{%
    \parbox{\linewidth}{%
      \setlength{\parindent}{\normalparindent}
      \noindent#1%
    }%
  }%
}

% cancel terms in equations
\usepackage[makeroom]{cancel}

% for boxes around equations, with \Aboxed
\usepackage{mathtools}

\usepackage{esint}
\usepackage{amssymb}
% \usepackage{commath}
\allowdisplaybreaks{}

% degree symbol etc.
\usepackage{gensymb}

% dingbats
\usepackage{pifont}
\newcommand{\cmark}{\ding{51}}

% SI units
\usepackage{siunitx}
\sisetup{per-mode = fraction,%
  math-micro = \text{µ}, text-micro = µ%
}
% Margin notes
\usepackage{marginnote}

% FSM graphs
\usepackage{tikz}
\usetikzlibrary{arrows,automata}

% Captions
\usepackage{caption}

% Recalculate page spread based on the definitions above
\recalctypearea{}

%% User commands
% Formatting and languages
\newcommand{\strong}[1]{\textbf{#1}}
\newcommand{\code}[1]{\texttt{#1}}
\newcommand{\var}[1]{\textit{#1}}
\newcommand{\en}[1]{\foreignlanguage{english}{#1}}

% Abbreviations
\usepackage{xspace}
\newcommand{\ao}{\mbox{u.\,a.}\xspace}
\newcommand{\cf}{\mbox{vgl.}\xspace}
\newcommand{\ie}{\mbox{d.\,h.}\xspace}
\newcommand{\eg}{\mbox{e.g.}\xspace}
\newcommand{\ital}{\mbox{ital.}\xspace}

% Common terms
\newcommand{\sectionname}{Section}
\newcommand{\eq}{equation\xspace}

% Commands for common math expressions
\newcommand{\abs}[1]{\lvert#1\rvert}
\renewcommand\d[1]{\:\textrm{d}#1}
\newcommand*\diff{\mathop{}\!\mathrm{d}}
\newcommand*\mdelta[1]{\ensuremath{\mathrm{\Delta\,}#1}}
\renewcommand{\qedsymbol}{\ensuremath{\blacksquare}}

% Chemistry
\newcommand*\chem[1]{\ensuremath{\mathrm{#1}}}

% alignment in \cases
\makeatletter
\renewcommand{\env@cases}[1][@{}l@{\quad}l@{}]{%
 \let\@ifnextchar\new@ifnextchar{}
 \left\lbrace{}
 \def\arraystretch{1.2}%
 \array{#1}%
}
\makeatother

% asides
\usepackage{mdframed}
\newenvironment{aside}
{\begin{mdframed}[style=0,%
  leftline=false,rightline=false,leftmargin=2em,rightmargin=2em,%
  innerleftmargin=0pt,innerrightmargin=0pt,linewidth=0.75pt,%
  skipabove=7pt,skipbelow=7pt]\small}
{\end{mdframed}}

%% Shortcuts
%%% arrays (vectors, matrixes, tensors)
\newcommand{\vecb}[1]{\mathbf{#1}}
\newcommand{\parray}[2]{\left(\begin{array}{#1}#2\end{array}\right)}
%% Vector norm
\newcommand{\norm}[1]{\left\|{#1}\right\|}
%% Constants
\newcommand{\const}[1]{\ensuremath{\mathrm{#1}}}
% Speed of light
\renewcommand{\c}{\const{c}}

\begin{document}
  % User-defined language-specific hyphenation
  %  \hyphenation{bezüg-lich einer Da-ten-bank-ope-ra-ti-onen
  %  effizienz-stei-gernd ECMA ECMAScript Firefox Google JavaScript JavaScript-Core
  %  Kom-pa-ti-bi-li-täts-matrix lauf-fähig lenny Linux MySQL proto-typ-ba-sier-ter
  %  robusteren SquirrelFish Wine}
  \hyphenation{nanosecond}

  \begin{titlepage}
    \title{\href{https://www.coursera.org/learn/einstein-relativity/}{Understanding Einstein:\\The Special Theory of Relativity}}
    \subtitle{Notes made in the Stanford University Online Course}
    \author{Larry Randles Lagerstrom, Instructor\\\href{http://PointedEars.de/}{Thomas Lahn}, Student}
    \maketitle
  \end{titlepage}

  \clearpage
  \pagenumbering{Roman}
  \begin{spacing}{1}
    % Print TOC
    \tableofcontents
    \thispagestyle{empty}
  \end{spacing}

  %   \clearpage
  %   \begin{spacing}{1}
  %     \chapter*{List of acronyms} \label{chapter:acronyms}
  %     \begin{acronym}[]
  %       \setlength{\itemsep}{-\parsep}
  %        \acro{RFC}{Request for Comments (Internet-Standard)}
  %     \end{acronym}
  %   \end{spacing}

  \clearpage
  \pagenumbering{arabic}
  \chapter{Week 1}
    \section{Introduction to the Course}
      \subsection{Why take this course?}
        \begin{enumerate}
          \item \href{http://content.time.com/time/magazine/article/0,9171,993017,00.html?iid=sr-link1}{Albert Einstein: Time Magazine’s “Person of the century”}
          \item Miracle year of 1905
          \item Albert Einstein:
            \begin{quote}
              “The important thing is not to stop questioning.
              Curiosity has its own reason for existing.
              One cannot help but be in awe when one contemplates the mystery of eternity,
              of life, of the marvelous structure of reality.
              It is enough if one tries to comprehend only a little of this mystery every day.”
            \end{quote}
        \end{enumerate}

      \subsection{How to succeed in the course?}
        \begin{enumerate}
          \item
            \begin{samepage}
              \emph{Cultivate a “growth mindset”} (see~\cite{dweck2017mindset}):\\
              How do you deal with a bad day – are you discouraged (“fixed mindset”)
              or \emph{working towards a better one} (“growth mindset”)?

              See also: neuroplasticity\\
              \begin{aside}
                As a parent, praise the \emph{effort} of your child, \underline{not}
                its success or intelligence.  Otherwise your praise encourages
                the \emph{fixed} mindset, and is actually \emph{counterproductive}:
                If your child hears you saying to it “you are smart”, it implies
                to it that it will always be smart, and that no effort by it
                would be required.  Encouraging the \emph{fixed} mindset
                will cause self-doubt in your child as things \emph{will} become
                more difficult for it later.
              \end{aside}
            \end{samepage}
          \item \emph{Knowledge is constructed, not received}:\\
            It is not sufficient to take in new information.
          \item \emph{Embrace the struggle}:\\
            \textquote[Albert Einstein]{It is not the result of scientific research
            that ennobles humans and enriches their nature, but \emph{the struggle
            to understand} while performing creative and open-minded intellectual
            work.}
          \item Practical tips:
            \begin{enumerate}[a)]
              \item \emph{Take notes}: The writing process helps with understanding
              \item \emph{Visualize}: Depictions are easier to understand than text
              \item \emph{Repeat}:\\
                Understanding improves the more often you are exposed to the same ideas
              \item \emph{Testing is better than rehearsing}:\\
                Do not learn things by heart, but \emph{test yourself} so that you can know
                if you have \emph{understood} what you have learned.
              \item \emph{Don’t multitask}: this reduces concentration (see~\cite{medina2009brain}).
                \begin{aside}
                  The \emph{Pomodoro Technique}\footnote{\ital\foreignquote{italian}{pomodoro} “tomato”
                  because kitchen timers often come in such a shape} can help you to focus:
                  Set a timer for 20 minutes, then focus your attention on whatever
                  you need to do, for that time.  Then take a break for 5 minutes.
                  Then set the timer for 20 minutes again and continue working, or
                  start working on something else.  And so on.
                \end{aside}
            \end{enumerate}
          \item \foreignquote{latin}{\emph{Festina lente}} – “Make haste slowly” (classical adage and oxymoron):\\
            \textquote[Wikipedia]{[…] activities should be performed with a proper balance of urgency and diligence.
            If tasks are overly rushed, mistakes are made and good long-term results are not achieved.}
        \end{enumerate}

        \paragraph{Other good websites re.\ tips ``How to succeed in this course''}
          \begin{itemize}
            \item \url{http://duckstop.stanford.edu/}
            \item \url{http://resilience.stanford.edu/}
          \end{itemize}

      \subsection{Rules of Engagement}
        \begin{enumerate}
          \item Course rules and etiquette
            \begin{enumerate}[a)]
              \item Do your own work\\
                \textquote[Albert~Einstein]{I have little patience for a scientist
                who’d take a board of wood, look for its thinnest part, and drill
                a great number of holes where drilling is easy.}
              \item Assume the best of people – this is a NSZ (No Snark Zone)
            \end{enumerate}
          \item Strengths and limitations of a course like this
            \begin{itemize}
              \item Strengths: Videos can be paused, fast-forwarded, and watched again
              \item Limitations: Not all questions can be answered
            \end{itemize}
          \item What this course is not about
            \begin{itemize}
              \item We are not covering Einstein’s whole life and work,
                like (topics of) General Relativity.
              \item It’s not about why “Einstein was wrong”,
                not about alternative theories and fringe science.
            \end{itemize}
        \end{enumerate}

      \clearpage
      \subsection{Math review}
        \begin{enumerate}
          \item How exponents work\\
            Examples (using powers of 2):
            \begin{align*}
              2^4 &= 2 \times 2 \times 2 \times 2 = 16\\
              2^1 &= 2\\
              2^0 &= 1\\
              2^{-2} &= \frac{1}{2^2} = \frac{1}{4}\\
              2^3 \times 2^5 &= 2^{3+5} = 2^8 = 256
            \end{align*}
          \item Square roots:
            \begin{align*}
              \sqrt{3} \sqrt{3} &= 3\\
              \sqrt{a} \sqrt{a} &= a\\
              \sqrt{a} &= a^{\frac{1}{2}}\\
              \left(a^{\frac{1}{2}}\right)\left(a^{\frac{1}{2}}\right) &= a^{\frac{1}{2}+\frac{1}{2}} = a^1 = a
            \end{align*}
          \item Factoring out:
            \begin{align*}
              a + b &= \left(a\right)\left(1 + \frac{b}{a}\right)\\
              a^2 + b^2 &= \left(a^2\right)\left(1 + \frac{b^2}{a^2}\right)
            \end{align*}
          \item Creating a common denominator in order to add two fractions:
            \begin{align*}
              \frac{a}{b} + \frac{c}{d}
              = \left(\frac{a}{b}\right)\left(\frac{d}{d}\right) + \left(\frac{c}{d}\right)\left(\frac{b}{b}\right)
              = \frac{ad + bc}{bd}
            \end{align*}
          \item Basic plotting of y vs.\ x \\
            \\
            Examples of $y = x^2$ (parabola) and $y = Ax + B$ (a line with slope A and y-intercept B) \\
            \\
            Graphical meaning of larger vs.\ smaller values for the slope of a line, and a negative slope
        \end{enumerate}

    \clearpage
    \section{Einstein in Context}
      \subsection{To the Miracle Year}
        \paragraph{Youth in Munich}
        \paragraph{Stateless}
        \paragraph{Polytechnic on second try}
        \paragraph{Herr Professor Weber}
        \paragraph{Mileva}
        \paragraph{Graduation}
        \paragraph{Career anxieties}
        \paragraph{A friend in need}
        \paragraph{Lasting memories}
        \paragraph{Zigzagging}
        \paragraph{The Patent Office}
        \paragraph{The Olympia Academy}
        \paragraph{Lieserl}
        \paragraph{The miracle year}

      \subsection{The Miracle Year}
        \paragraph{March 1905: The light quantum idea, a “heuristic proposal”}
        \paragraph{April 1905: The size of molecules (doctoral dissertation)}
        \paragraph{May 1905: The existence of atoms (Brownian motion)}
        \paragraph{June 1905: On the electrodynamics of moving bodies (special relativity)}
        \paragraph{September 1905: $E = mc^2$}

  \chapter{Week 2: Events, Clocks, and Reference Frames}
    \section{Introduction}
      \subsection{Reading “The Electrodynamics of Moving Bodies”}
        \subsubsection{Introduction}
          \begin{itemize}
            \item Problems:
              \begin{enumerate}
                \item Asymmetries in Maxwell’s electrodynamics:
                  For example, contrary to the theory, no matter if the conductor is considered
                  moving or the magnet, an electric current is induced in the conductor by the magnet.
                \item Impossibility to detect the luminiferous ether
              \end{enumerate}
            \item Postulates, supported by previous experimental results:
              \begin{enumerate}
                \item Principle of relativity – laws of physics are the same in all inertial frames of reference
                \item Principle of light constancy – The speed of light is the same and constant in all inertial frames of reference
              \end{enumerate}
            \item Assumption: This solves both problems
            \item All electrodynamics deals with the kinematics of rigid bodies only, so Einstein’s too.
          \end{itemize}

        \subsubsection{Kinematics}
          \paragraph{§ 1. Definition of Simultaneity}
            \begin{itemize}
              \item To describe motion, one needs a measure of space and time.
              \item What is time?  Its definition depends on simultaneity,
                defined only by how light travels from an event to an observer.
              \item Conclusion: One can use light to measure times and lengths.
                Synchronize clocks by sending light from a clock to another and back.
                This way we can define and measure the speed of light.
            \end{itemize}

          \paragraph{§ 2. The Relativity of Lengths and Times}
            \begin{itemize}
              \item Clarification of the two postulates
              \item Rod moving along the x-axis.  What is the length of the moving rod?
                \begin{enumerate}[a)]
                  \item as measured using a meter by an observer moving with the rod (in his rest frame);
                  \item as measured using synchronized, stationary clocks;
                    the length of the rod at certain times is the distance
                    between the (imagined) clocks on the ends of the rod.
                \end{enumerate}
              \item Length (a) is the length of the rod (principle of relativity);
                length (b) is \emph{classically} assumed to be equal to length (a), but \emph{actually} is not.
              \item Ends of the rod are labeled “A” and “B”.
              \item Send light from A to B\@: this is related to the length of the rod, the relative speed of the rod, and the speed of light
              \item Observation: Observer in clocks’ rest frame would see clocks being synchronous, moving observer would not.
              \item Conclusion: Simultaneity is not an \emph{absolute} concept, but depends on the \emph{relative} motion.
            \end{itemize}

            \paragraph{§ 3. Coordinate Transformation From Rest Frame to Motion Frame}
              \begin{itemize}
                \item TODO
              \end{itemize}

    \clearpage
    \section{Events, Clocks, and Observers}

    \clearpage
    \section{Spacetime Diagrams}

    \clearpage
    \section{Frames of Reference}

    \clearpage
    \section{A Few More Words on World Lines}

    \clearpage
    \section{The Galilean Transformation}
      \paragraph{Priming the brain}
        Bob’s position at any given time $t$: $x(t) = \SI{3}{\meter\per\second} \times t$ \\
        Bob’s position with a head start of 4 m: $x(t) = \SI{4}{\meter} + \SI{3}{\meter\per\second} \times t$ \\
        In general: $ x(t) = x_0 + v \, t $

      \paragraph{Lecture}
        Question: Given a location and time of an event in Bob’s frame of reference,
        what is the corresponding location and time in Alice’s frame of reference?
        (Assuming that at $t = 0$, Alice and Bob are side-by-side at $x_A = 0, x_B = 0$.)

        \begin{align*}
          {(x_A)}_{flash} &= {(x_B)}_{flash} + v \; t \\
          {(x_B)}_{flash} &= {(x_A)}_{flash} - v \; t \\
          t_A &= t_B
        \end{align*}

    \section{Week 2 Summary}
      \begin{enumerate}
        \item Spacetime location of event → (x, y, z, t) or (x, t)
        \item How to specify the time? “Photo clock principle”
        \item How to synchronize clocks? Slow distribution or setting distant clocks ahead
        \item Spacetime diagrams
        \item Frames of reference, inertial frame of reference, velocities addition (classical)
        \item Galilean transformation
          \begin{align*}
            x_{Lab} &= x_{Rocket} + v \; t \qquad \text{Rocket is moving to the right}\\
            x_{Lab} &= x_{Rocket} - v \; t \qquad \text{Rocket is moving to the left}
          \end{align*}
      \end{enumerate}

  \chapter{Week 3: Ethereal Problems and Solutions}
    \section{Einsteins Starting Point: The Two Postulates}
      \paragraph{The June 1905 paper and the context leading up to it}
        \begin{itemize}
          \item Einstein already worked 8 to 10 years at least on the problem
          \item in April 1905, he went on a walk and talk with friend Michèle Besso,
            and shortly after said that “time is suspect”
          \item Other physicists working on the problem: Hendrik Antoon Lorentz, Henri Poincaré
        \end{itemize}
      \paragraph{The magnet and the coil}
      \paragraph{The two parts of the paper}
      \paragraph{The principle of relativity}
      \paragraph{The luminiferous ether}
      \paragraph{The principle of light constancy}

    \section{A Few Words About Waves}
      \subsection{Part 1}
        \subsubsection{Diagram}
        \subsubsection{Key terms}
          \paragraph{Medium}
            thing of which a wave is a propagating disturbance
          \paragraph{Periodic}
            something with a regularly repeated pattern; synonymous: harmonic wave
          \paragraph{Transverse}
            disturbance is perpendicular to the propagation of the wave
          \paragraph{Longitudinal}
            (density) disturbance propagates parallel to the direction of the wave
          \paragraph{Amplitude}
            absolute of the maximum value (for a transversal wave, the vertical distance to the rest position)
          \paragraph{Wavelength}
            $\lambda$: distance between two peaks or two troughs of the wave,
            or between two points with the same value and same ascent
          \paragraph{Period}
            $T$: time from one peak to the next one, measured \eg{} in seconds ($\mathrm{s}$)
          \paragraph{Frequency}
            $f$ or $\nu$ (Greek small letter Nu): how frequent are the up and down movements of the wave per unit time;
            $T = \frac{1}{f}$
          \paragraph{Velocity}
            \begin{itemize}
              \item Distance that the wave travels per unit time
              \item If we are told that there is a simple relationship between the velocity of the wave,
                its period and its wavelength, and we know the units in which they are measured
                ($[x] = \mathrm{y}$ means “the quantity $x$ is measured in the unit $\mathrm{y}$”),
                we can infer it thus:
                \begin{align*}
                  [v] &= \frac{\mathrm{m}}{\mathrm{s}}\\
                  [T] &= \mathrm{s}\\
                  T &= \frac{1}{f} \rightarrow~f = \frac{1}{T} \rightarrow~[f] = s^{-1}\\
                  [\lambda] &= \mathrm{m}\\
                  \rightarrow~v &= \lambda f = \frac{\lambda}{T}
                \end{align*}
            \end{itemize}

      \subsection{Part 2}
        \subsubsection{Diagram}
        \subsubsection{Key terms}
          \paragraph{Phase}
          \paragraph{Constructive interference}
            If waves are in phase, peaks and troughs add, respectively;
            the amplitude of the resulting wave is greater than each of the individual waves
          \paragraph{Destructive interference}
          If waves are out of phase, peaks and troughs subtract, respectively;
          the amplitude of the resulting wave is lower than each of the individual waves;
          worst case: they cancel out
          \paragraph{In phase}
            Two waves are “in phase” if they are in sync with each other
            (their peaks and troughs are aligned, respectively)
          \paragraph{Out of phase}
            Two waves are “out of phase” if they are not in sync with each other
            (their peaks and troughs are not aligned, respectively; worst case:
            peaks are aligned with troughs [180° out of phase])

      \subsection{Part 3}
        \begin{aside}
          Einstein and Picasso were contemporaries (\cite{miller2008einstein}).
        \end{aside}
        \begin{itemize}
          \item Three key facts about the speed of waves\footnote{Speed is the magnitude of velocity, but we are not making that distinction here}:
            \begin{enumerate}
              \item it depends on the medium
              \item moving source \rightarrow~no change in wave speed
                (because more/less peaks are generated per unit time
                and $v = \lambda \, f$), but change in wavelength (Doppler effect)\\
                example: 24 = 4 \times \, 6 = 3 \times \, 8
              \item moving medium \rightarrow~wave speed changes
            \end{enumerate}
        \end{itemize}

    \clearpage
    \section{The Michelson–Morley Experiment}
      \subsection{Part 1}
        \subsubsection{The goal}
          Albert Michelson (1881), Edward Morley (1887, now Case Western University, Ohio; more precise experiment):
          \emph{Detect the “ether wind”}

        \subsubsection{Math reminders}
          [more basic stuff already covered in the math review]
          \begin{align*}
            \frac{a}{c} + \frac{b}{d} &= \left(\frac{a}{c}\right)\left(\frac{d}{d}\right) + \left(\frac{b}{d}\right)\left(\frac{c}{c}\right) \\
            &= \frac{ad + bc}{cd}\\
            \\
            \sqrt{a^2 - b^2} &= \sqrt{a^2 \left(1 - \frac{b^2}{a^2}\right)} \\
            &= \sqrt{a^2} \sqrt{1 - \frac{b^2}{a^2}}\\
            &= a \sqrt{1 - \frac{b^2}{a^2}}
          \end{align*}
          “Binomial expansion” (\emph{“physicists are using this all the time”}):\label{eq:binomial-expansion}
          \begin{align*}
            {\left(1 + a\right)}^n &\approx 1 + na\text{, if } a \ll 1\\
            {\left(1 - a\right)}^n &\approx 1 - na\text{, if } a \ll 1\\
            {\left(1 + a\right)}^{-n} &\approx 1 - na\text{, if } a \ll 1\\
            \text{For example: }a &= \frac{v}{\c}
          \end{align*}
        \subsubsection{The idea behind the experiment (Earth’s motion through the luminiferous ether)}
          \begin{itemize}
            \item Assumption: The ether is a stationary medium, so as the Earth moves around, it moves through the ether
              \rightarrow~“ether wind”.
            \item Send light beam against and with direction of the “ether wind”
              \rightarrow~there should be a difference in observed frequency
          \end{itemize}

      \subsection{Part 2}
        \emph{Note}: The diagrams showing the A to B path of the airplane (and back again) are from the perspective of
        an observer looking down from above (like a map). In other words, imagine up is North, down is South,
        right is East, and left is West.

        \subsubsection{The airplane example}
          Plane travels from A and B and back\\
          D – distance between A and B\\
          $v_P$ – velocity of plane\\
          $v_W$ – velocity of wind blowing from B towards A\\

        \subsubsection{The “no wind” case}\label{section-no-wind}
          \begin{align*}
            \text{Total time required by plane} = \frac{D}{v_P} + \frac{D}{v_P} = \frac{2D}{v_P}
          \end{align*}

        \subsubsection{The “headwind/tailwind” case}
          \begin{align*}
            \text{A to B trip: Time} &= \frac{D}{v_P - v_W}\\
            \text{B to A trip: Time} &= \frac{D}{v_P + v_W}\\
            \text{Total time} &= \frac{D}{v_P - v_W} + \frac{D}{v_P + v_W}\\
            &= \frac{D\left(v_P + v_W\right) + D\left(v_P - v_W\right)}{\left(v_P - v_W\right) \left(v_P + v_W\right)}\\
            &= \frac{2 D v_P}{\left(v_P - v_W\right) \left(v_P + v_W\right)}\\
            &= \frac{2 D v_P}{{v_P}^2 + v_P v_W - v_W v_P + {v_W}^2}\\
            &= \frac{2 D v_P}{{v_P}^2 - {v_W}^2}\\
            &= \frac{2 D v_P}{{v_P}^2 \left(1 - \frac{{v_W}^2}{{v_P}^2}\right)}\\
            &= \frac{2 D}{{v_P} \left(1 - \frac{{v_W}^2}{{v_P}^2}\right)}\\
            \text{Total time with headwind/tailwind} &= \left(\frac{2 D}{v_P}\right) \left(\frac{1}{1 - \frac{{v_W}^2}{{v_P}^2}}\right)\\
            \text{Total time without wind} &= \frac{2 D}{v_P} \qquad \text{(see~\ref{section-no-wind})}
          \end{align*}

          The result of the “headwind/tailwind” case is \emph{obviously not equal}
          to that of the “no~wind” case.  So the time ``gained'' by flying with
          the wind does \emph{not} compensate for the time ``lost'' by flying
          against the wind earlier!  (To double-check the result, set $v_W = 0$
          \rightarrow~$\text{Total time} = \frac{2D}{v_P}$. QED.)

          \begin{samepage}
            Extreme example to demonstrate this more intuitively:
            \begin{align*}
              v_P &= \SI{300}{\kilo\meter\per\hour}\\
              v_W &= \SI{299}{\kilo\meter\per\hour}\\
              v_P - v_W &= \SI{1}{\kilo\meter\per\hour}\\
              v_P + v_W &= \SI{599}{\kilo\meter\per\hour} \qquad \text{returns 599 times as fast, \emph{not} twice as fast}
            \end{align*}
          \end{samepage}

          To do this with light and the Earth moving in the either, Michelson and Morley
          thought that because the speed of light is so large compared to the speed
          of Earth relative to the ether the increase in total traveling time could not
          be~measured.  As an alternative, they considered the “crosswind” case.

      \subsection{Part 3}
        \paragraph{The “crosswind” case}
        \paragraph{The flowing river analogy}
          If you try to swim across a river from A to B, you have to swim against
          the direction of the river flow to cross in a straight line ($v_P$ is
          diagonal to the flow $v_W$).
          \begin{align*}
            \vec{v}_{actual} &= \vec{v}_P + \vec{v}_W\\
            \vec{v}_{actual} &\perp \vec{v}_W\\
            {v_P}^2 &= {v_W}^2 + {v_{actual}}^2 \qquad \text{Pythagorean theorem}\\
            {v_{actual}}^2 &= {v_P}^2 - {v_W}^2\\
            v_{actual} &= \sqrt{{v_P}^2 - {v_W}^2}
          \end{align*}

        \paragraph{The total time for the plane’s round trip}
          The river is always flowing/the wind is always blowing
          in the same direction.  So it does not matter if you are
          going from A to B or B to A: the total time required
          for the round trip is twice the time for each leg:
          \begin{align*}
            \text{Time} &= \frac{D}{v_{actual}}\\
            &= \frac{D}{\sqrt{{v_P}^2 - {v_W}^2}}\\
            \text{Total time} &= \frac{2D}{\sqrt{{v_P}^2 - {v_W}^2}}\\
            &= \frac{2D}{\sqrt{{v_P}^2\,\left(1 - \frac{{v_W}^2}{{v_P}^2}\right)}}\\
            &= \frac{2D}{v_P\,\sqrt{1 - \frac{{v_W}^2}{{v_P}^2}}}\\
            \text{Total time with crosswind} &= \left(\frac{2D}{v_P}\right)\left(\frac{1}{\sqrt{1 - \frac{{v_W}^2}{{v_P}^2}}}\right)
          \end{align*}

      \clearpage
      \subsection{Part 4}
        \paragraph{The experimental set-up}
          \begin{itemize}
            \item Path~A: Light beam crossing the ether wind (if any);
              then being reflected by a half-silvered mirror in the center
              towards the ether wind;
              reflected back by a full mirror, with the ether wind,
              through the half-silvered mirror,
              then going with the ether wind, onto the detector;
            \item Path~B: Light beam going through the half-silvered mirror,
              crossing the ether wind; being reflected back by full mirror,
              crossing the wind again;
              then reflected by the half-silvered mirror,
              then going with the ether wind, onto the detector.
          \end{itemize}
        \paragraph{Different travel times}
          \begin{align*}
            \text{Path~A time} &= \left(\frac{2D}{\c}\right)\left(\frac{1}{1 - \frac{{v_W}^2}{\c^2}}\right)\\
            \text{Path~B time} &= \left(\frac{2D}{\c}\right)\left(\frac{1}{\sqrt{1 - \frac{{v_W}^2}{\c^2}}}\right)\\
            \text{Path~A time} - \text{Path~B time}
              = \mdelta{t} &= \left(\frac{2D}{\c}\right)
                \left(
                  \frac{1}{\left(1 - \frac{{v_W}^2}{\c^2}\right)}
                  - \frac{1}{{\left(1 - \frac{{v_W}^2}{\c^2}\right)}^{\frac{1}{2}}}
                \right)\\
            \mdelta{t}  &= \left(\frac{2D}{\c}\right)
                \left(
                  {\left(1 - \frac{{v_W}^2}{\c^2}\right)}^{-1}
                  - {{\left(1 - \frac{{v_W}^2}{\c^2}\right)}}^{-\frac{1}{2}}
                \right)
            \intertext{Binomial expansion (\ref{eq:binomial-expansion}),
            because with ${v_W \approx \SI{29.78}{\kilo\meter\per\second}}$
            (orbital speed of Earth), ${\frac{{v_W}^2}{\c^2} \ll 1}$:}
            \mdelta{t} &\approx \left(\frac{2D}{\c}\right)
              \left(
                \left(1 - \left(-1\right)\left(\frac{{v_W}^2}{\c^2}\right)\right)
                - \left(1 - \left(-\frac{1}{2}\right)\left(\frac{{v_W}^2}{\c^2}\right)\right)
              \right)\\
            &= \left(\frac{2D}{\c}\right)
              \left(
                \left(1 + \frac{{v_W}^2}{\c^2}\right)
                - \left(1 + \frac{1}{2} \frac{{v_W}^2}{\c^2}\right)
              \right)\\
            &= \left(\frac{2D}{\c}\right)
              \left(
                1 + \frac{{v_W}^2}{\c^2}
                - 1 - \frac{1}{2} \frac{{v_W}^2}{\c^2}
              \right)\\
            &= \left(\frac{2D}{\c}\right)
              \left(\frac{1}{2} \frac{{v_W}^2}{\c^2}\right)\\
            \mdelta{t} &\approx \left(\frac{D}{\c}\right)
              \left(\frac{{v_W}^2}{\c^2}\right).
          \end{align*}

        \paragraph{Some actual numbers}
          \begin{align*}
            \text{For the experiment, }D \approx \SI{11}{\meter},\text{ so}\\
            \mdelta{t} &\approx \SI{3.67e-16}{\second}.
          \end{align*}
          If you rotate the table by 90°, you can get differences of
          ca. \SI{7e-16}{\second}.\textsuperscript{why?} But this is still too short a time to measure
          with clocks.

        \paragraph{The key technique Michelson and Morley used, and the “null result”}
          \begin{itemize}
            \item \begin{samepage}Key technique: Interference patterns.
              If there is no time difference, the light waves of the beams
              of the two paths are in phase \rightarrow~constructive interference.

              They used sodium (\chem{Na}) light:
              \begin{align*}
                \lambda &= \SI{508}{\nano\meter}\\
                f &= \frac{\c}{\lambda}\\
                T &= \frac{1}{f} = \frac{\lambda}{\c} = \SI{2e-15}{\second}
              \end{align*}
              The time difference \mdelta{t} to be expected because of
              the ether wind (if any) is smaller than that, so if one wave
              is out of phase by only
              \begin{align*}
                \frac{T}{\mdelta{t}} &= \frac{\SI{7e-16}{\second}}{\SI{2e-15}{\second}} = 0.37
              \end{align*}
              of a wavelength, you can see a different interference pattern
              (\emph{“fringe pattern”}) in the detector.\end{samepage}
            \item But they observed, with different orientations, at different
              times of the year, at most a \emph{fringe shift of
              only 0.005, well within experimental error
              \rightarrow~“null result”.}
            \item Physicists of the time were surprised.  Interpretation:
              Is there no ether at all {--} or has it properties that make it
              undetectable?
            \item Possible explanation: Earth could drag the ether with it
              when it is rotating, so you could not measure the ether wind on
              the surface of Earth (ether dragging).  But this has been shown
              \emph{not} to be the case \emph{either}, by
              \rightarrow~stellar aberration.
          \end{itemize}

    \section{Stellar Aberration}
      \paragraph{Case 1: Ether dragged by Earth as it travels around the Sun}
        Light gets straight into the telescope as it Earth is rotating,
        because the ether is dragged along by the rotation of Earth.
        No correction necessary.

      \paragraph{Case 2: Ether not dragged by Earth as it travels around the Sun}
        Light does not get straight into the telescope as it Earth is rotating,
        bcause the ether is not dragged along by the rotation of Earth, and
        the light travels in a straight line.  Before it reaches the observer,
        he and his telescope have already changed position because
        of the moving, rotating Earth.  The telescope must be slightly tilted
        ahead so that the light gets in straight.

    \section{Ethereal Solutions}
      \paragraph{G.F. Fitzgerald}
        \begin{itemize}
          \item Assumption: Compression effect of the measuring apparatus when moving against the ether
          \item sent letter to American journal
        \end{itemize}

      \paragraph{H.A. Lorentz}
        \begin{itemize}
          \item he had the same idea as Fitzgerald, and gave credit to Fitzgerald
          \rightarrow~Lorentz–Fitzgerald contraction
        \end{itemize}

      \paragraph{The Lorentz-Fitzgerald contraction hypothesis}

      \paragraph{Henri Poincaré}
      \begin{itemize}
        \item built on the Lorentz-Fitzgerald contraction hypothesis
        \item Idea: Time may be changing
      \end{itemize}

      \paragraph{Einstein’s approach}
        Not certain that he was influenced by the MM experiment
        \subparagraph{Combining the principle of light constancy with the principle of relativity}

        \subparagraph{Einstein’s conclusion about the speed of light}
        \subparagraph{Einstein’s conclusion about the ether}

  \chapter{Week 4: The Weirdness Begins}
    \section{Introduction}
      \paragraph{Our journey up to this point}
      \paragraph{Quotes of the week}
      \paragraph{\textquote{Time is suspect}}

    \section{\textquote{Time is suspect}: The Relativity of Simultaneity}
      \subsection{Diagram 1: Alice and Bob stationary}
      \subsection{Diagram 2: Paintball experiment (Alice moving, Bob observing)}
      \subsection{Diagram 3: Light pulse experiment (Alice moving, Bob observing)}
      \subsection{Summary}
        \textquote{Leading clocks lag}

    \section{The Light Clock, and Exploring the Lorentz Factor}
      \subsection{The light clock, part 1}
      \subsection{The light clock, part 2}
        \begin{align*}
          \intertext{One tick of Alice’s clock as she sees it (light pulse up and down again):}
          \mdelta{t_A} &= \frac{2\,L}{\c}
          \intertext{One tick of Bob’s clock as he sees it:}
          \mdelta{t_B} &= \frac{2\,L}{\c}
          \intertext{One tick of Alice’s clock as Bob sees it:}
          \mdelta{t_B} &= \frac{2\,D}{\c}\\
          \text{qualitatively: }\mdelta{t_B} &> \mdelta{t_A}\\
          \\
          D^2 &= L^2 + {\left(\frac{x}{2}\right)}^2 \\
          D &= \frac{\c\,\mdelta{t_B}}{2}\\
          L &= \frac{\c\,\mdelta{t_A}}{2}\\
          x &= v \, \mdelta{t_B}\\
          \\
          \frac{\c^2\,{\mdelta{t_B}}^2}{4} &= \frac{\c^2\,{\mdelta{t_A}}^2}{4} + \frac{v^2\,{\mdelta{t_B}}^2}{4}\\
          \c^2\,{\mdelta{t_B}}^2 &= \c^2\,{\mdelta{t_A}}^2 + v^2\,{\mdelta{t_B}}^2\\
          \c^2\,{\mdelta{t_B}}^2 - v^2\,{\mdelta{t_B}}^2 &= \c^2\,{\mdelta{t_A}}^2\\
          {\mdelta{t_B}}^2 \left(\c^2 - v^2\right) &= \c^2\,{\mdelta{t_A}}^2\\
          {\mdelta{t_B}}^2 &= \frac{\c^2\,{\mdelta{t_A}}^2}{\c^2 - v^2}\\
          &= \frac{\c^2}{\c^2 - v^2} \, {\mdelta{t_A}}^2\\
          {\mdelta{t_B}}^2 &= \frac{1}{1 - \frac{v^2}{\c^2}} \, {\mdelta{t_A}}^2\\
          \mdelta{t_B} &= \sqrt{\frac{1}{1 - \frac{v^2}{\c^2}}} \, \mdelta{t_A}\\
          \mdelta{t_B} &= \frac{1}{\sqrt{1 - \frac{v^2}{\c^2}}} \, \mdelta{t_A}\\
          \gamma &= \frac{1}{\sqrt{1 - \frac{v^2}{\c^2}}} \geq 1 \qquad \text{Lorentz factor}\\
          \mdelta{t_B} &>^? \mdelta{t_A} \text{\cmark} \qquad \text{Time dilation}
        \end{align*}

        \begin{samepage}
          \begin{aside}
            My approach:
            \begin{align*}
              D &= \sqrt{L^2 + {\left(\frac{x}{2}\right)}^2} \\
              \mdelta{t_B} &= \frac{2\,\sqrt{L^2 + {\left(\frac{x}{2}\right)}^2} }{\c}\\
              L &= \frac{\mdelta{t_A}\,\c}{2}\\
              L^2 &= \frac{{\mdelta{t_A}}^2\,\c^2}{4}\\
              \mdelta{t_B} &= \frac{2\,\sqrt{\frac{{\mdelta{t_A}}^2\,\c^2}{4} + {\left(\frac{x}{2}\right)}^2} }{\c}\\
                &= \frac{\sqrt{\frac{4\,{\mdelta{t_A}}^2\,\c^2}{4} + 4\,{\left(\frac{x}{2}\right)}^2} }{\c}\\
                &= \frac{\sqrt{{\mdelta{t_A}}^2\,\c^2 + 4\,{\left(\frac{x}{2}\right)}^2} }{\c}\\
                &= \frac{\sqrt{{\mdelta{t_A}}^2\,\c^2 + 4\,\left(\frac{x^2}{4}\right)} }{\c}\\
                &= \frac{\sqrt{{\mdelta{t_A}}^2\,\c^2 + x^2} }{\c}\\
                &= \frac{\c^2\,\sqrt{{\mdelta{t_A}}^2\, + \frac{x^2}{\c^2}} }{\c}\\
                &= \c\,\sqrt{{\mdelta{t_A}}^2\, + \frac{x^2}{\c^2}}\\
                \mdelta{t_B} &= \c\,\sqrt{{\mdelta{t_A}}^2\, + \frac{v^2\,{\mdelta{t_B}}^2}{\c^2}}\\
                \mdelta{t_B} &= \c\,\sqrt{{\mdelta{t_A}}^2\, + \frac{v^2\,{\mdelta{t_B}}^2}{\c^2}}\\
            \end{align*}
          \end{aside}
        \end{samepage}

    \section{Time Dilation}
      \paragraph{Duration of one click tick on moving clock compared to clock at rest}
        \begin{align*}
          {\left(\mdelta{t}\right)}_{\text{moving}} &= \gamma\,{\left(\mdelta{t}\right)}_{\text{rest}}
        \end{align*}
      \paragraph{Elapsed time on moving clock compared to clock at rest}
        \begin{align*}
          {\left(\text{Elapsed time}\right)}_{\text{moving}} &= \frac{1}{\gamma}\,{\left(\text{Elapsed time}\right)}_{\text{rest}}\\
          {\left(\mdelta{T}\right)}_{\text{moving}} &= \frac{1}{\gamma}\,{\left(\mdelta{T}\right)}_{\text{rest}}
        \end{align*}

        \emph{Note: This does \underline{not} mean that time slows down \underline{for you} if you are moving.}

      \paragraph{1. Is the light clock simply a special case?
        2. Does time dilation occur for all types of clocks?}
        1. No; 2. Yes.

        Because if Alice had a normal clock, and the light clock were special,
        the former would get out of sync with her light clock if, and only if,
        she were moving (because that is what Bob would have to observe, too).

        But this would violate the special principle of relativity (which says
        that if you are moving uniformly the laws of physics apply exactly as if
        you were at rest; IOW, there is no experiment that allows you
        to determine if you are moving uniformly or not; IYOW, there is no
        absolute velocity; velocity depends on the frame of reference).

    \section{Measuring Length (length contraction)}
      Alice’s result for the length of Bob’s ship:
      \begin{align*}
        {\left(L_B\right)}_{Alice} &= v \left(T_{A2} - T_{A1}\right) = v\,\mdelta{T_A}
      \end{align*}

      Bob’s result for the length of his ship:
      \begin{align*}
        {\left(L_B\right)}_{Bob} &= v \left(T_{B2} - T_{B1}\right) = v\,\mdelta{T_B}
      \end{align*}

      For Bob:
      \begin{align*}
        \mdelta{T_A} &= \frac{1}{\gamma} \mdelta{T_B}\\
        {\left(L_B\right)}_{Alice} &= v\,\mdelta{T_A}\\
        &= v\,\frac{1}{\gamma} \mdelta{T_B}\\
        &= \frac{1}{\gamma} \mdelta{T_B}\,v\\
        {\left(L_B\right)}_{Alice} &= \frac{1}{\gamma} {\left(L_B\right)}_{Bob}
      \end{align*}

      \subsubsection{Summary}
        Time dilation:
        \begin{align*}
          {\left(\mdelta{T}\right)}_{moving} &= \frac{1}{\gamma} {\left(\mdelta{T}\right)}_{rest} \qquad  \text{Time dilation}\\
          \mdelta{T}\text{ – \underline{elapsed} time}
          \mdelta{T}_{rest}\text{ – Proper time}
        \end{align*}

        Time dilation:
        \begin{align*}
          L_{moving} &= \frac{1}{\gamma} L_{rest} \qquad \text{Length contraction}\\
          L_{rest}\text{ – Proper length}
        \end{align*}

    \section{What is Not Suspect, and the Invariant Interval}
      \subsection{What is not suspect (invariants)}
        \strong{Lengths not in the direction of motion stay are not shortened.}

        If it were different,
        \begin{itemize}
          \item as to \strong{width}, Alice would observe the wheels of Bob’s
            moving train cars to fall off as its width reduces, while Bob
            would see the width of the tracks get narrower;
          \item as to \strong{height}, supposed there would be a tunnel through
            which Bob’s train car would just fit while at rest, then he would
            observe that his car would into that tunnel as it its height were
            reduced, while Alice would observe that he has more room to get
            through the tunnel.
        \end{itemize}

        Either case would be a contradiction; if you take a photo while on
        the car, both Alice and Bob must agree as to its status.

      \subsection{The invariant interval}
        \begin{samepage}
          Consider the time and distance between two events in, as observed
          in different frames of reference:

          [see handout]
          \begin{align*}
            h &\text{ {--} height of the light clocks}\\
            x_A &\text{ {--} distance between Bob's events for Alice}\\
            x_K &\text{ {--} distance between Bob's events for Kris}\\
            t_A &\text{ {--} one tick of Bob's clock from Alice's perspective}\\
            t_K &\text{ {--} one tick of Bob's clock from Kris's perspective}\\
            t_B &\text{ {--} one tick of Bob's clock according to him}\\
            \\
            h^2 + {\left(\frac{x_A}{2}\right)}^2 &= {\left(\frac{\c\,t_A}{2}\right)}^2\\
            h^2 + \frac{{x_A}^2}{4} &= \frac{\c^2\,{t_A}^2}{4}\\
            4\,h^2 + {x_A}^2 &= \c^2\,{t_A}^2\\
            4\,h^2 &= \c^2\,{t_A}^2 - {x_A}^2\\
            4\,h^2 &= \c^2\,{t_K}^2 - {x_K}^2\\
            2\,h &= \c\,\frac{t_B}{2} + \c\,\frac{t_B}{2}\\
            &= \c\,t_B\\
            4\,h^2 &= \c^2\,{t_B}^2
            \intertext{or}
            4\,h^2 &= \c^2\,{t_B}^2 - {x_B}^2
            \intertext{but $x_B = 0$.}
          \end{align*}
        \end{samepage}

    \clearpage
    \section{A Real-Life Example: The Muon}
      \begin{align*}%
        \text{Average muon (proper) lifetime: } \tau &= \SI{2.2}{\micro\second}\\
        \text{Average velocity: } v &= \SI{0.998}{\c}\\
        \text{\rightarrow~Distance traveled: } s &\approx \SI{660}{\meter}\\
        \text{Altitude where they are generated by cosmic rays: } h &\approx \SI{10}{\kilo\meter}
        \longintertext{But a lot of atmospheric muons are detected near
          Terra's surface.  How is that possible?\\
          \\
          \emph{Special relativity!}}
        \gamma &= \frac{1}{\sqrt{1 - \frac{v^2}{\c^2}}} \approx 15
        \intertext{From our perspective, time dilation:}
        \tau_{observer} &= \gamma \, \tau \\
          &= 15 \times \SI{2.2}{\micro\second}\\
          &= \SI{33}{\micro\second}\\
        s_{observer} &= v \, \tau_{observer}\\
          &= \SI{0.998}{\c} \times \SI{33}{\micro\second}\\
          &\approx \SI{9.873}{\kilo\meter} \approx h. \qed
        \intertext{From the muon's perspective, length contraction:}
        h_{\mu} &= \frac{1}{\gamma} \, h\\
          &= \frac{\SI{10}{\kilo\meter}}{15} \\
          &\approx \SI{667}{\meter}
          \approx s. \qed
      \end{align*}

  \chapter{Week 5: Spacetime Switches}
    \section{Quotations of the week}
    \section{Convenient units for the speed of light}
      \paragraph{Calculating the Lorentz factor when writing the velocity in terms of \c{} (e.g.,~$v~=~\SI{0.9}{\c}$)}
        \begin{align*}
          \gamma &= \frac{1}{\sqrt{1 - \frac{v^2}{\c^2}}}\\
          \gamma(v = \SI{0.9}{\c}) &= \frac{1}{\sqrt{1 - \frac{{\left(\SI{0.9}{\c}\right)}^2}{\c^2}}}\
            = \frac{1}{\sqrt{1 - \frac{{\left(0.9\right)}^2{\c}^2}{\c^2}}}\
            = \frac{1}{\sqrt{1 - {\left(0.9\right)}^2}}\\
        \end{align*}
      \paragraph{The concept of light-years (and light-months, light-days, light-seconds)}
        1~light-year~[\SI{1}{ly}] is the distance that light travels in vacuum
        in 1~year [\SI{1}{a}].
        \begin{align*}
          \c_0 &= 1\,\frac{\text{ly}}{\text{a}}
        \end{align*}
      \paragraph{The speed of light in meters/second, km/second, light-years/year, feet/nanosecond, and so on}
        \begin{align*}
          \c_0 &\approx 1\,\frac{\text{ft}}{\text{ns}}
        \end{align*}

    \section{Exploring time dilation and length contraction (Star Tours, part 1)}
      \paragraph{“Star Tours” set-up:} A trip to a star 5 light-years away.
        \begin{align*}
          v &= \SI{0.943}{\c} \rightarrow~\gamma = 3
        \end{align*}

      \paragraph{Analysis done from the perspective of the observer on Earth (the Earth-Star reference frame)}
        \begin{align*}
          \text{Travel time } \mdelta{T}_{rest} &= \frac{\SI{5}{ly}}{\SI{0.943}{\c}} = \SI{5.3}{a}\\
          \mdelta{T}_{moving} &= \frac{1}{\gamma} \mdelta{T}_{rest} = \frac{\SI{5.3}{a}}{\gamma} \approx \SI{1.77}{a}
        \end{align*}

      \paragraph{Analysis done from the perspective of the observer on the rocket (the rocket reference frame)}
        Observes that the Earth--Star distance is contracted
        \begin{align*}
          L_{moving} &= \frac{1}{\gamma} L_{rest} = \frac{\SI{5}{ly}}{\gamma} \approx \SI{1.67}{ly}\\
          \text{Time for star to reach Bob } \mdelta{T}_{Rocket} &= \frac{\SI{1.67}{ly}}{\SI{0.943}{\c}} = \SI{1.77}{a}(!)\\
        \end{align*}

      \paragraph{The puzzle/conundrum}
        \begin{align*}
          \text{Bob's observation of Alice's elapsed time} &= \frac{\mdelta{T}_{Rocket}}{\gamma}\\
            &= \frac{\SI{1.77}{a}}{\gamma} \approx \SI{0.59}{a} \text{ (???)}
        \end{align*}
        (Hint: relativity of simultaneity, ``leading clocks lag'')

    \section{Deriving the Lorentz transformation}
      \subsection{Part 1}
        Bob in a spaceship moving relative to Alice

        At start:
        \begin{align*}
          t_A = 0 \qquad t_B = 0\\
          x_A = 0 \qquad x_B = 0
        \end{align*}
        \paragraph{The question/goal}
          Given $(x_B, \, y_B, \, z_B, \, t_B)$,
          what is $(x_A, \, y_A, \, z_A, \, t_A)$?

          Or: $(x_A, \, y_A, \, z_A, \, t_A) \leftrightarrow (x_B, \, y_B, \, z_B, \, t_B)$

        \paragraph{Reminder of the Galilean transformation}
          \begin{align*}
            x_A &= x_B + v \, t_B\\
            y_A &= y_B\\
            z_A &= z_B\\
            t_A &= t_B
          \end{align*}

        \paragraph{Bob and Alice and the flash of light}
          Later, at time $t_B$, there is a flash of light in Bob's cockpit.

          We know:
          \begin{align*}
            t_A &= \gamma \, t_B, \, \gamma = \frac{1}{\sqrt{1 - \frac{v^2}{\c^2}}}
          \end{align*}
          \emph{But note:} This was derived for $x_B = 0$.

        \paragraph{Applying the invariant interval equation}
          We also know:
          \begin{align*}
            \c^2 \, {t_A}^2 - {x_A}^2 &= \c^2 \, {t_B}^2 - {x_B}^2 \qquad \text{Invariant interval}
          \end{align*}

          Consider first the case when the flash of light occurs at $x_B = 0$:
          \begin{align*}
             {\left(x_B\right)}_{flash} &= 0\\
             {\left(x_A\right)}_{flash} &= v \, t_A
             \intertext{Because $x_B = 0$, we can apply the known time dilation equation:}
             {\left(x_A\right)}_{flash} &= \gamma \, v \, t_B
             \intertext{Using the invariant interval:}
              \c^2 \, {t_A}^2 - {x_A}^2 &= \c^2 \, {t_B}^2 - \cancel{{x_B}^2} 0\\
              \c^2 \, {t_A}^2 - {\left(v \, t_A\right)}^2 &= \c^2 \, {t_B}^2\\
              \c^2 \, {t_A}^2 - v^2 \, {t_A}^2 &= \c^2 \, {t_B}^2\\
              {t_A}^2 \left(\c^2 - v^2\right) &= \c^2 \, {t_B}^2\\
              {t_A}^2 &= \frac{\c^2}{\c^2 - v^2} \, {t_B}^2\\
                &= \frac{\cancel{\c^2}}{\cancel{\c^2} \left(1 - \frac{v^2}{\c^2}\right)} \, {t_B}^2\\
              {t_A}^2 &= \frac{1}{1 - \frac{v^2}{\c^2}} \, {t_B}^2\\
              t_A &= \frac{1}{\sqrt{1 - \frac{v^2}{\c^2}}} \, t_B\\
              \Aboxed{t_A &= \gamma \, t_B}\\
              x_A &= v \, t_A\\
              \Aboxed{x_A &= \gamma \, v \, t_B.}
          \end{align*}

      \subsection{Part 2}
        \paragraph{Review of Part 1 and the overall goal}
          Goal: For an event at $(x_B, \, t_B)$, what are $x_A$ and $t_A$?
          That is
          \begin{align*}
            x_A &= \text{some formula involving $x_B$ and $t_B$}\\
            t_A &= \text{some formula involving $x_B$ and $t_B$}
          \end{align*}

        \paragraph{Restrictions on the possible formula}
          \begin{enumerate}
            \item The units have to match, so e.g., $x_A = {x_B}^2 + \gamma \, t_B$ cannot
              be correct.
            \item It has to work for $x_B = 0$
          \end{enumerate}

        \paragraph{Why the clock ticks have to be uniform}
          \begin{align*}
            t_A = \frac{{t_B}^2}{\frac{x_B}{v}} \text{ could work dimensionally.}
            \intertext{Why can there not be a ${t_B}^2$ on the right-hand side?  Assume}
            t_A &= {t_B}^2,
            \intertext{then:}
            t_B &= 0, \, 1, \, 2, \, 3 \rightarrow 1, \, 2 \, 3, \, 4\\
            t_A &= 0, \, 1, \, 4, \, 9 \rightarrow 1, \, 4 \, 9, \, 16
          \end{align*}
          \emph{The clock ticks have to be uniform, but would not be for later times.}

        \paragraph{A linear form for the desired equations}
          \begin{samepage}
            Therefore, the transformation has to have a linear form:
            \begin{align*}
              t_A &= G \, x_B + H \, t_B\\
              x_A &= M \, x_B + N \, t_B
            \end{align*}
          \end{samepage}

        \paragraph{Finding two of the four unknown coefficients (G, H, M, N)}
          If $x_B = 0$, $t_A = \gamma \, t_B$ and $x_A = \gamma \, v \, t_B$ and
          \begin{align*}
            t_A &= G \, \cancel{x_B} 0 + H \, t_B = \gamma \, \t_B \rightarrow~H = \gamma\\
            x_A &= M \, \cancel{x_B} 0 + N \, t_B = \gamma \, v \, t_B \rightarrow~N = \gamma \, v
            \intertext{so:}
            t_A &= G \, x_B + \gamma \, t_B\\
            x_A &= M \, x_B + \gamma \, v \, t_B
          \end{align*}

      \subsection{Part 3}
        \paragraph{Finding the remaining coefficients (G and M) using the invariant interval equation}
          \begin{align*}
            \c^2 \, {t_A}^2 - {x_A}^2 &= \c^2 \, {t_B}^2 - {x_B}^2
            \intertext{Insert the terms for $x_A$ and $t_A$:}
            \c^2 \, {(G \, x_B + \gamma \, t_B)}^2 - {(M \, x_B + \gamma \, v \, t_B)}^2 &= \c^2 \, {t_B}^2 - {x_B}^2\\
            \c^2 \, (G^2 \, {x_B}^2 + 2 \, G \, x_B \gamma \, t_B + {\gamma}^2 {t_B}^2)
              - (M^2 \, {x_B}^2 + 2 \, M \, \gamma \, v \, x_B \, t_B + {\gamma}^2 \, v^2 \, {t_B}^2)
              &= \c^2 \, {t_B}^2 - {x_B}^2\\
            \c^2 \, (G^2 \, {x_B}^2 + 2 \, G \, \gamma \, x_B \, t_B + {\gamma}^2 {t_B}^2)
              - M^2 \, {x_B}^2 - 2 \, M \, \gamma \, v \, x_B \, t_B  - {\gamma}^2 \, v^2 \, {t_B}^2
              &= \c^2 \, {t_B}^2 - {x_B}^2\\
            (\c^2 \, {\gamma}^2 {t_B}^2 - {\gamma}^2 v^2 {t_B}^2)
              + (2 \, G \, \gamma \, \c^2 \, x_B \, t_B - 2 \, M \, \gamma \, v \, x_B \, t_B)
              + (\c^2 \, G^2 {x_B}^2 - M^2 \, {x_B}^2) &= \c^2 \, {t_B}^2 - {x_B}^2\\
            (\c^2 \, {\gamma}^2 - {\gamma}^2 v^2) \, {t_B}^2
              + (2 \, G \, \gamma \, \c^2 - 2 \, M \, \gamma \, v) \, x_B \, t_B
              + (\c^2 \, G^2 - M^2) \, {x_B}^2 &= \c^2 \, {t_B}^2 - {x_B}^2\\
          \end{align*}
          Therefore:
          \begin{align*}
            \c^2 \, {\gamma}^2 - {\gamma}^2 v^2 &= c^2\\
            {\gamma}^2 \, (\c^2 - v^2) &= c^2\\
            {\gamma}^2 &= \frac{c^2}{\c^2 - v^2}\\
            {\gamma}^2 &= \frac{1}{1 - \frac{v^2}{c^2}}\\
            {\gamma} &= \frac{1}{\sqrt{1 - \frac{v^2}{c^2}}} \qed\\
            2 \, G \, \gamma \, \c^2 - 2 \, M \, \gamma &= 0\\
            (2 \, \gamma) (G \, \c^2 - \, M \, v) &= 0\\
            \gamma \neq 0 \rightarrow~G \, \c^2 - \, M \, v &= 0\\
            G \, \c^2 &= M \, v\\
            G &= \frac{M \, v}{\c^2}\\
            \c^2 \, G^2 - M^2 &= -1\\
            -\c^2 \, G^2 + M^2 &= 1\\
            M^2 - \c^2 \, G^2 &= 1
            \intertext{Insert $G$:}
            M^2 - \c^2 \, {\left(\frac{M \, v}{\c^2}\right)}^2 &= 1\\
            M^2 - \c^2 \, \frac{M^2 \, v^2}{\c^4} &= 1\\
            M^2 - \frac{M^2 \, v^2}{\c^2} &= 1\\
            M^2 \, \left(1 - \frac{v^2}{\c^2}\right) &= 1\\
            M^2 &= \frac{1}{1 - \frac{v^2}{\c^2}}\\
            M &= \frac{1}{\sqrt{1 - \frac{v^2}{\c^2}}} = \gamma\\
            G &= \frac{M \, v}{\c^2} = \frac{\gamma \, v}{\c^2}
          \end{align*}

        \paragraph{The final form of the Lorentz transformation}
          \begin{align*}
            t_A &= G \, x_B + \gamma \, t_B\\
            t_A &= \frac{\gamma \, v}{\c^2} \, x_B + \gamma \, t_B\\
            x_A &= M \, x_B + \gamma \, v \, t_B\\
            x_A &= \gamma \, x_B + \gamma \, v \, t_B
          \end{align*}
          \begin{align*}
            \Aboxed{t_A &= \gamma \, \left(t_B + \frac{v}{\c^2} \, x_B\right)}\\
            \Aboxed{x_A &= \gamma \, \left(x_B + v \, t_B\right)}
          \end{align*}

    \section{Exploring the Lorentz transformation}
      \subsection{Part 1}
        \paragraph{Summary of results so far}
          Rewriting the transformation equations in order to avoid confusion:
          \begin{align*}
            \Aboxed{t_{rest} &= \gamma \, \left(t_{moving} + \frac{v}{\c^2} \, x_{moving}\right)}\\
            \Aboxed{x_{rest} &= \gamma \, \left(x_{moving} + v \, t_{moving}\right)}
          \end{align*}

        \paragraph{What happens when $v = 0$?}
          \begin{samepage}
            Let $v = 0$ → $\gamma = 1$, then:
            \begin{align*}
              t_{rest} &= \cancel{\gamma} \, 1 \, \left(t_{moving} + \cancel{\frac{\cancel{v} \, 0}{\c^2} \, x_{moving}}\right)\\
              t_{rest} &= t_{moving}\\
              x_{rest} &= \cancel{\gamma} \, 1 \, \left(x_{moving} + \cancel{\cancel{v} \,  0 \, t_{moving}}\right)\\
              x_{rest} &= x_{moving}
            \end{align*}
            \emph{[Observers not moving relative to each other agree on times and locations.]}
          \end{samepage}
        \paragraph{What happens when $v$ is much less than \c{} (the speed of light)?}
          Let $v \ll \c$ → $\gamma \approx 1$ and $\frac{v}{\c^2} \approx 0$, then:
          \begin{align*}
            t_{rest} &= \cancel{\gamma} \, 1 \, \left(t_{moving} + \cancel{\frac{\cancel{v} \, 0}{\c^2} \, x_{moving}}\right)\\
            t_{rest} &= t_{moving}\\
            x_{rest} &= \cancel{\gamma} \, 1 \, \left(x_{moving} + v \, t_{moving}\right)\\
            x_{rest} &= x_{moving} + v \, t_{moving}
          \end{align*}
          \emph{This is the Galilean transformation.}
        \paragraph{What happens when $v$ is large?}
          At $t_{moving} = 0$, we get \emph{length contraction}:
          \begin{align*}
            x_{rest} &= \gamma \, \left(x_{moving} + \cancel{\cancel{v} \, 0 \, t_{moving}}\right)\\
            x_{rest} &= \gamma \, x_{moving}\\
            x_{moving} &= \frac{x_{rest}}{\gamma}
          \end{align*}

      \subsection{Part 2: Exploring the time transformation equation}
        \paragraph{What happens when $x = 0$ (as measured in the moving frame of reference)?}
          If $x_{moving} = 0$, we get \emph{time dilation}:
          \begin{align*}
            t_{rest} &= \gamma \, t_{moving} + \cancel{\gamma \, \frac{v}{\c^2} \, 0 \, \cancel{x_{moving}}}\\
            t_{rest} &= \gamma \, t_{moving}\\
            t_{moving} &= \frac{t_{rest}}{\gamma}
          \end{align*}
        \paragraph{What happens in general (Alice’s lattice of clocks vs. Bob’s lattice of clocks)?}
          If $x_{moving} = 0$, we get \emph{time dilation}:
          \begin{align*}
            t_{rest} &= \gamma \, t_{moving} + \cancel{\gamma \, \frac{v}{\c^2} \, \cancel{x_{moving}} \, 0}\\
            t_{rest} &= \gamma \, t_{moving}\\
            t_{moving} &= \frac{t_{rest}}{\gamma}
          \end{align*}
        \paragraph{The result (leading clocks lag)}
          From
          \begin{align*}
            t_{rest} &= \gamma \, t_{moving} + \gamma \, \frac{v}{\c^2} \, x_{moving}
          \end{align*}
          also follows: For the same time $t_{rest}$ measured on the clocks
          at rest, as $x_{moving}$ is increasing, $t_{moving}$ has to decrease.
          IOW, the farther away the moving clocks are, the more are they behind
          the clocks at rest: \emph{Leading clocks lag.}

      \subsection{Part 3: The inverse transformation (movement to the left, or negative $x$ direction)}
        [This is easier to understand with Alice at rest and Bob moving,
         hence the old indexes here.]
        \begin{align*}
          t_A &= \gamma \, \left(t_B + \frac{v}{\c^2} \, x_B\right)\\
          x_A &= \gamma \, \left(x_B + v \, t_B\right)
        \end{align*}
        From Bob's perspective, he is at rest and Alice is moving in the negative
        $x$ direction.

        If the direction of the velocity is reversed, we just have to reverse
        the sign of $v$:
        \begin{align*}
          t_B &= \gamma \, \left(t_A + \frac{-v}{\c^2} \, x_A\right)\\
          t_B &= \gamma \, \left(t_A - \frac{v}{\c^2} \, x_A\right)\\
          x_B &= \gamma \, \left(x_A + (-v) \, t_A\right)
          x_B &= \gamma \, \left(x_A - v \, t_A\right)
          \intertext{$\gamma$ also contains $v$, but:}
          \gamma &= \frac{1}{\sqrt{1 - \frac{v^2}{c^2}}} = \frac{1}{\sqrt{1 - \frac{{(-v)}^2}{c^2}}}
        \end{align*}

    \section{Leading clocks lag, revisited}
      \subsection{Quantitative analysis}
        $L_A$~-- Length of Bob's ship as measured by Alice
        \paragraph{Rearward light beam}
          When it hits the rear clock on Bob's ship, Alice
          reads ${(T_A)}_{rear}$ on her lattice of clocks.
          \begin{align*}
            \text{Distance covered} &= \frac{L_A}{2} - v \, {(T_A)}_{rear}\\
            \text{Elapsed time } {(T_A)}_{rear} &= \frac{\text{Distance covered}}{\c}\\
            {(T_A)}_{rear} &= \frac{\frac{L_A}{2} - v \, {(T_A)}_{rear}}{\c}\\
            \c \, {(T_A)}_{rear} &= \frac{L_A}{2} - v \, {(T_A)}_{rear}\\
            \c \, {(T_A)}_{rear} + v \, {(T_A)}_{rear} &= \frac{L_A}{2}\\
            {(T_A)}_{rear} (\c + v) &= \frac{L_A}{2}\\
            {(T_A)}_{rear} &= \left(\frac{L_A}{2}\right)\left(\frac{1}{\c + v}\right)
          \end{align*}

        \paragraph{Frontward light beam}
          When it hits the front clock on Bob's ship, Alice
          reads ${(T_A)}_{front}$ on her lattice of clocks. She has to wait
          longer for that than ${(T_A)}_{rear}$, because the light beam also
          has to cover the distance that Bob's ship has moved in the meantime.
          \begin{align*}
            \text{Distance covered} &= \frac{L_A}{2} + v \, {(T_A)}_{front} \\
            \text{Elapsed time } {(T_A)}_{front} &= \frac{\text{Distance covered}}{\c}\\
            {(T_A)}_{front} &= \frac{\frac{L_A}{2} + v \, {(T_A)}_{rear}}{\c}\\
            \c \, {(T_A)}_{front} &= \frac{L_A}{2} + v \, {(T_A)}_{front}\\
            \c \, {(T_A)}_{front} - v \, {(T_A)}_{front} &= \frac{L_A}{2}\\
            {(T_A)}_{front} (\c - v) &= \frac{L_A}{2}\\
            {(T_A)}_{front} &= \left(\frac{L_A}{2}\right)\left(\frac{1}{\c - v}\right)\\
            {(T_A)}_{front} &>^? {(T_A)}_{rear}\\
            \left(\frac{L_A}{2}\right)\left(\frac{1}{\c - v}\right) &>^? \left(\frac{L_A}{2}\right)\left(\frac{1}{\c + v}\right)\\
            \frac{1}{\c - v} &>^! \frac{1}{\c + v}. \text{ \emph{Leading clocks lag.}} \qed
          \end{align*}

        \paragraph{\emph{By how much time} do leading clocks lag?}
          \begin{align*}
            \mdelta{T_A} &= {(T_A)}_{front} - {(T_A)}_{rear}\\
              &= \left(\frac{L_A}{2}\right)\left(\frac{1}{\c - v}\right)
                - \left(\frac{L_A}{2}\right)\left(\frac{1}{\c + v}\right)\\
              &= \left(\frac{L_A}{2}\right)
                 \left(\frac{1}{\c - v} - \frac{1}{\c + v}\right)\\
              &= \left(\frac{L_A}{2}\right)
                 \left(\frac{\cancel{\c} + v - (\cancel{\c} - v)}{(\c - v)(\c + v)}\right)\\
              &= \left(\frac{L_A}{\cancel{2}}\right)
                 \left(\frac{\cancel{2} \, v}{\c^2 - v^2}\right)\\
              &= L_A \,
                 \left(\frac{v}{\c^2 \left(1 - \frac{v^2}{\c^2}\right)}\right)\\
              &= \left(\frac{L_A \, v}{\c^2}\right)
                 \left(\frac{1}{1 - \frac{v^2}{\c^2}}\right)\\
            \Aboxed{\mdelta{T_A} &= \left(\frac{L_A \, v}{\c^2}\right)
              \, {\gamma}^2}
            \intertext{Alice sees Bob's ship length-contracted:}
            L_A &= \frac{L_B}{\gamma}\\
            \mdelta{T_A} &= \left(\frac{L_B \, v}{\gamma \, \c^2}\right)
              \, {\gamma}^2\\
            \mdelta{T_A} &= \left(\frac{L_B \, v}{\c^2}\right)
              \, \gamma
            \intertext{$\mdelta{T_A}$ is the time difference that Alice sees on her clocks.
              She sees time on Bob's clocks time-dilated:}
            \mdelta{T_B} &= \frac{\mdelta{T_A}}{\gamma}\\
            \Aboxed{\mdelta{T_B} &= \frac{L_B \, v}{\c^2}
              \qquad \text{Time that Bob's front clock lags behind his rear clock}}
            \intertext{This can be generalized to}
            \Aboxed{\mdelta{T_B} &= \frac{D \, v}{\c^2}
              \qquad \text{Time that the front clock lags behind the rear clock}}
            \intertext{where $D$ is the distance between any two clocks in Bob's frame of reference.}
          \end{align*}

      \subsection{Using the Lorentz transformation}
        \begin{samepage}
          Proper length of Bob's ship: L\\
          Bob's front clock: B1\\
          Bob's rear clock: B2\\
          Alice's clock aligned with Bob's front clock: A1\\
          Alice's clock aligned with Bob's rear clock: A2
          \begin{align*}
            t_A &= \gamma \, \left(t_B + \frac{v}{\c^2} \, x_B\right)\\
            t_B &= \gamma \, \left(t_A - \frac{v}{\c^2} \, x_A\right)
            \intertext{Assume}
            t_{A1} &= t_{A2} = 0 \qquad \text{(Alice's clocks are synchronized)}\\
            t_{B1} &= 0\\
            x_{A1} &= 0\\
            x_{A2} &= x_{A1} - \frac{L}{\gamma} = -\,\frac{L}{\gamma} \text{ due to length contraction,}
            \intertext{then:}
            t_{B2} &= \gamma \, \left(t_{A2} - \frac{v}{\c^2} \, x_{A2}\right)\\
              &= \gamma \, \left(0 - \frac{v}{\c^2} \, \left(-\,\frac{L}{\gamma}\right)\right)\\
              &= \gamma \, \left(\frac{L \, v}{\gamma \, \c^2}\right)\\
            t_{B2} &= \frac{L \, v}{\c^2} \qed
          \end{align*}
        \end{samepage}

    \section{Star Tours, part 2}
      \paragraph{“Star Tours” set-up:}
        
      \paragraph{Analysis done from the perspective of the observer on Earth (the Earth-Star reference frame)}
      \paragraph{Analysis done from the perspective of the observer on the rocket (the rocket reference frame)}
      \paragraph{The answer to the puzzle (via “leading clocks lag”)}

    \section{The ultimate speed limit}
    \section{Combining velocities}

  % References
  \begingroup
  %     \raggedright
      %\sloppy
    \printbibliography[heading=lit]
  \endgroup
\end{document}
