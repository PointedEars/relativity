% vim:set fileencoding=utf-8 tabstop=2 shiftwidth=2 softtabstop=2 expandtab:
%% Page layout
% Scientific report, European style (A4) (from KOMA-Script)
% BCOR: binding correction
% DIV=calc: Calculate page spread now (recalculated below)
% pagesize: Add page info to (PDF/PS) output
% parskip=half: Do not indent paragraphs, add one half line spacing instead
%   (FFHS requirement)
% Default font size: 10pt (FFHS requirement)
% BCOR=1cm too much?
\documentclass[pagesize,headsepline,10pt,parskip=half]{scrreprt}

% Line spacing should be 1.5 times the line height (FFHS requirement)
\usepackage{setspace}
\onehalfspacing{}

% float compatibility for KOMA-Script
\usepackage{scrhack}

% Use English orthography and hyphenation by default
\usepackage[latin,italian,english]{babel}

% Allow using non-ASCII characters verbatim; disable for LuaLaTeX
% \usepackage[utf8]{inputenc}

% Special characters
\usepackage{textcomp}

% Use fonts having non-ASCII characters; disable fontenc for LuaLaTeX
% \usepackage[T1]{fontenc}
\usepackage{fontspec}
\newfontfamily\DejaSans{DejaVu Sans}

% User-defined English hyphenation
\hyphenation{InfoWorld}

% Use language-specific quotes
\usepackage[autostyle,english=american]{csquotes}
\DeclareQuoteAlias{italian}{latin}

% Advanced Computer Modern fonts
\usepackage{lmodern}

% Allow strike-through with \sout, keep italic for \emph
\usepackage[normalem]{ulem}

% Use medieval numbers except in math mode
% \usepackage{hfoldsty}

% Use sans-serif font ('Arial') by default for headings and normal text;
% disable helvet for LuaLaTeX
% (FFHS requirement)
\renewcommand{\familydefault}{\sfdefault}
\usepackage{mathptmx}
%\usepackage[scaled=.90]{helvet}
\usepackage{courier}

% Improved typography, like hyphenation in words with non-ASCII characters,
% see also http://homepage.ruhr-uni-bochum.de/georg.verweyen/pakete.html
\usepackage[babel]{microtype}

% Number also \subsubsection, but not \paragraph and below
\setcounter{secnumdepth}{3}

% Page heading and footer
\usepackage{scrpage2}
\pagestyle{scrheadings}
\automark[section]{chapter}
% heading on the top inner margin only
\ohead[]{\headmark}
\chead[]{}
% page number on the bottom outer margin only
\ofoot[\pagemark]{\pagemark}
\cfoot[]{}

% Support for list of acronyms
\usepackage[footnote,nohyperlinks,withpage]{acronym}

% References: can use section names
\usepackage{nameref}

% References: generate hyperlinks
\usepackage[plainpages=false]{hyperref}

% References style (default: 'numerical')
\usepackage[
  style=authoryear-ibid,
  maxcitenames=1,
  maxbibnames=3
]{biblatex}
\defbibheading{lit}{\chapter*{References}\markboth{References}{References}}
% References: bibliography database
\addbibresource{main.bib}

% prints author names as small caps
\renewcommand{\mkbibnamefirst}[1]{\textsc{#1}}
\renewcommand{\mkbibnamelast}[1]{\textsc{#1}}
\renewcommand{\mkbibnameprefix}[1]{\textsc{#1}}
\renewcommand{\mkbibnameaffix}[1]{\textsc{#1}}

% References: use English ordinal numbers
\usepackage[super]{nth}

% Automatically use teletype for \url argument, use content verbatim
\usepackage{url}
\urlstyle{tt}

\usepackage{makeidx}
\makeindex

% Less vertical spacing between list items (in `compactitem' environment)
\usepackage{paralist}

% Multi-row table cells
\usepackage{multirow}

% Support for horizontal rules in tables (professional style)
\usepackage{booktabs}

% Footnotes in tables
\usepackage{threeparttable}

% Tables across pages (for results)
\usepackage{longtable}

% Word wrap in table columns (calculate p width)
\usepackage{calc}

% Automatic column stretching
% \usepackage{tabularx}

% Stretched tables across pages (for results); requires longtable
% tabularx
\usepackage{ltxtable}

% Improved table formatting
\usepackage{array}

% Support for including PDFs
\usepackage{pdfpages}

% Support for figures
\usepackage{graphicx}

\usepackage{amsthm}
\newtheorem{mydef}{Definition}

\usepackage[fleqn]{amsmath}
\newlength{\normalparindent}
\AtBeginDocument{\setlength{\normalparindent}{\parindent}}
\newcommand{\longintertext}[1]{%
  \intertext{%
    \parbox{\linewidth}{%
      \setlength{\parindent}{\normalparindent}
      \noindent#1%
    }%
  }%
}

% cancel terms in equations
\usepackage[makeroom]{cancel}
\usepackage{color}
%\renewcommand{\CancelColor}{\green}

% for boxes around equations, with \Aboxed
\usepackage{mathtools}

\usepackage{esint}
\usepackage{amssymb}
% \usepackage{commath}
\allowdisplaybreaks{}

% degree symbol etc.
\usepackage{gensymb}

% dingbats
\usepackage{pifont}
\newcommand{\cmark}{\, \text{\ding{51}}}

% SI units
\usepackage{siunitx}
\sisetup{per-mode = fraction,%
  math-micro = \text{µ}, text-micro = µ%
}
\DeclareSIUnit\year{a}
\DeclareSIUnit\lightyear{ly}

% Margin notes
\usepackage{marginnote}

% FSM graphs
\usepackage{tikz}
\usetikzlibrary{arrows,automata}

% Captions
\usepackage{caption}

% Recalculate page spread based on the definitions above
\recalctypearea{}

%% User commands
% Formatting and languages
\newcommand{\strong}[1]{\textbf{#1}}
\newcommand{\code}[1]{\texttt{#1}}
\newcommand{\var}[1]{\textit{#1}}
\newcommand{\en}[1]{\foreignlanguage{english}{#1}}

% Abbreviations
\usepackage{xspace}
\newcommand{\ao}{\mbox{u.\,a.}\xspace}
\newcommand{\cf}{\mbox{vgl.}\xspace}
\newcommand{\ie}{\mbox{d.\,h.}\xspace}
\newcommand{\eg}{\mbox{e.g.}\xspace}
\newcommand{\ital}{\mbox{ital.}\xspace}

% Common terms
\newcommand{\sectionname}{Section}
\newcommand{\eq}{equation\xspace}

% Commands for common math expressions
\newcommand{\abs}[1]{\lvert#1\rvert}
\renewcommand\d[1]{\:\textrm{d}#1}
\newcommand*\diff{\mathop{}\!\mathrm{d}}
\newcommand*\mdelta[1]{\ensuremath{\mathrm{\Delta\,}#1}}
\renewcommand{\qedsymbol}{\ensuremath{\blacksquare}}

% Chemistry
\newcommand*\chem[1]{\ensuremath{\mathrm{#1}}}

% alignment in \cases
\makeatletter
\renewcommand{\env@cases}[1][@{}l@{\quad}l@{}]{%
 \let\@ifnextchar\new@ifnextchar{}
 \left\lbrace{}
 \def\arraystretch{1.2}%
 \array{#1}%
}
\makeatother

% asides
\usepackage{mdframed}
\newenvironment{aside}
{\begin{mdframed}[style=0,%
  leftline=false,rightline=false,leftmargin=2em,rightmargin=2em,%
  innerleftmargin=0pt,innerrightmargin=0pt,linewidth=0.75pt,%
  skipabove=7pt,skipbelow=7pt]\small}
{\end{mdframed}}

%% Shortcuts
%%% arrays (vectors, matrixes, tensors)
\newcommand{\vecb}[1]{\mathbf{#1}}
\newcommand{\parray}[2]{\left(\begin{array}{#1}#2\end{array}\right)}
%% Vector norm
\newcommand{\norm}[1]{\left\|{#1}\right\|}
%% Constants
\newcommand{\const}[1]{\ensuremath{\mathrm{#1}}}
% Speed of light
\renewcommand{\c}{\const{c}}

\begin{document}
  % User-defined language-specific hyphenation
  %  \hyphenation{bezüg-lich einer Da-ten-bank-ope-ra-ti-onen
  %  effizienz-stei-gernd ECMA ECMAScript Firefox Google JavaScript JavaScript-Core
  %  Kom-pa-ti-bi-li-täts-matrix lauf-fähig lenny Linux MySQL proto-typ-ba-sier-ter
  %  robusteren SquirrelFish Wine}
  \hyphenation{nanosecond}

  \begin{titlepage}
    \title{\href{https://www.coursera.org/learn/einstein-relativity/}{Understanding Einstein:\\The Special Theory of Relativity}}
    \subtitle{Notes made in the Stanford University Online Course}
    \author{Larry Randles Lagerstrom, Instructor\\\href{http://PointedEars.de/}{Thomas Lahn}, Student}
    \maketitle
  \end{titlepage}

  \clearpage
  \pagenumbering{Roman}
  \begin{spacing}{1}
    % Print TOC
    \tableofcontents
    \thispagestyle{empty}
  \end{spacing}

  %   \clearpage
  %   \begin{spacing}{1}
  %     \chapter*{List of acronyms} \label{chapter:acronyms}
  %     \begin{acronym}[]
  %       \setlength{\itemsep}{-\parsep}
  %        \acro{RFC}{Request for Comments (Internet-Standard)}
  %     \end{acronym}
  %   \end{spacing}

  \clearpage
  \pagenumbering{arabic}
  \chapter{Week 1}
    \section{Introduction to the Course}
      \subsection{Why take this course?}
        \begin{enumerate}
          \item \href{http://content.time.com/time/magazine/article/0,9171,993017,00.html?iid=sr-link1}{Albert Einstein: Time Magazine’s “Person of the century”}
          \item Miracle year of 1905
          \item Albert Einstein:
            \begin{quote}
              “The important thing is not to stop questioning.
              Curiosity has its own reason for existing.
              One cannot help but be in awe when one contemplates the mystery of eternity,
              of life, of the marvelous structure of reality.
              It is enough if one tries to comprehend only a little of this mystery every day.”
            \end{quote}
        \end{enumerate}

      \subsection{How to succeed in the course?}
        \begin{enumerate}
          \item
            \begin{samepage}
              \emph{Cultivate a “growth mindset”} (see~\cite{dweck2017mindset}):\\
              How do you deal with a bad day – are you discouraged (“fixed mindset”)
              or \emph{working towards a better one} (“growth mindset”)?

              See also: neuroplasticity\\
              \begin{aside}
                As a parent, praise the \emph{effort} of your child, \underline{not}
                its success or intelligence.  Otherwise your praise encourages
                the \emph{fixed} mindset, and is actually \emph{counterproductive}:
                If your child hears you saying to it “you are smart”, it implies
                to it that it will always be smart, and that no effort by it
                would be required.  Encouraging the \emph{fixed} mindset
                will cause self-doubt in your child as things \emph{will} become
                more difficult for it later.
              \end{aside}
            \end{samepage}
          \item \emph{Knowledge is constructed, not received}:\\
            It is not sufficient to take in new information.
          \item \emph{Embrace the struggle}:\\
            \textquote[Albert Einstein]{It is not the result of scientific research
            that ennobles humans and enriches their nature, but \emph{the struggle
            to understand} while performing creative and open-minded intellectual
            work.}
          \item Practical tips:
            \begin{enumerate}[a)]
              \item \emph{Take notes}: The writing process helps with understanding
              \item \emph{Visualize}: Depictions are easier to understand than text
              \item \emph{Repeat}:\\
                Understanding improves the more often you are exposed to the same ideas
              \item \emph{Testing is better than rehearsing}:\\
                Do not learn things by heart, but \emph{test yourself} so that you can know
                if you have \emph{understood} what you have learned.
              \item \emph{Don’t multitask}: this reduces concentration (see~\cite{medina2009brain}).
                \begin{aside}
                  The \emph{Pomodoro Technique}\footnote{\ital\foreignquote{italian}{pomodoro} “tomato”
                  because kitchen timers often come in such a shape} can help you to focus:
                  Set a timer for 20 minutes, then focus your attention on whatever
                  you need to do, for that time.  Then take a break for 5 minutes.
                  Then set the timer for 20 minutes again and continue working, or
                  start working on something else.  And so on.
                \end{aside}
            \end{enumerate}
          \item \foreignquote{latin}{\emph{Festina lente}} – “Make haste slowly” (classical adage and oxymoron):\\
            \textquote[Wikipedia]{[…] activities should be performed with a proper balance of urgency and diligence.
            If tasks are overly rushed, mistakes are made and good long-term results are not achieved.}
        \end{enumerate}

        \paragraph{Other good websites re.\ tips ``How to succeed in this course''}
          \begin{itemize}
            \item \url{http://duckstop.stanford.edu/}
            \item \url{http://resilience.stanford.edu/}
          \end{itemize}

      \subsection{Rules of Engagement}
        \begin{enumerate}
          \item Course rules and etiquette
            \begin{enumerate}[a)]
              \item Do your own work\\
                \textquote[Albert~Einstein]{I have little patience for a scientist
                who’d take a board of wood, look for its thinnest part, and drill
                a great number of holes where drilling is easy.}
              \item Assume the best of people – this is a NSZ (No Snark Zone)
            \end{enumerate}
          \item Strengths and limitations of a course like this
            \begin{itemize}
              \item Strengths: Videos can be paused, fast-forwarded, and watched again
              \item Limitations: Not all questions can be answered
            \end{itemize}
          \item What this course is not about
            \begin{itemize}
              \item We are not covering Einstein’s whole life and work,
                like (topics of) General Relativity.
              \item It’s not about why “Einstein was wrong”,
                not about alternative theories and fringe science.
            \end{itemize}
        \end{enumerate}

      \clearpage
      \subsection{Math review}
        \begin{enumerate}
          \item How exponents work\\
            Examples (using powers of 2):
            \begin{align*}
              2^4 &= 2 \times 2 \times 2 \times 2 = 16\\
              2^1 &= 2\\
              2^0 &= 1\\
              2^{-2} &= \frac{1}{2^2} = \frac{1}{4}\\
              2^3 \times 2^5 &= 2^{3+5} = 2^8 = 256
            \end{align*}
          \item Square roots:
            \begin{align*}
              \sqrt{3} \sqrt{3} &= 3\\
              \sqrt{a} \sqrt{a} &= a\\
              \sqrt{a} &= a^{\frac{1}{2}}\\
              \left(a^{\frac{1}{2}}\right)\left(a^{\frac{1}{2}}\right) &= a^{\frac{1}{2}+\frac{1}{2}} = a^1 = a
            \end{align*}
          \item Factoring out:
            \begin{align*}
              a + b &= \left(a\right)\left(1 + \frac{b}{a}\right)\\
              a^2 + b^2 &= \left(a^2\right)\left(1 + \frac{b^2}{a^2}\right)
            \end{align*}
          \item Creating a common denominator in order to add two fractions:
            \begin{align*}
              \frac{a}{b} + \frac{c}{d}
              = \left(\frac{a}{b}\right)\left(\frac{d}{d}\right) + \left(\frac{c}{d}\right)\left(\frac{b}{b}\right)
              = \frac{ad + bc}{bd}
            \end{align*}
          \item Basic plotting of y vs.\ x \\
            \\
            Examples of $y = x^2$ (parabola) and $y = Ax + B$ (a line with slope A and y-intercept B) \\
            \\
            Graphical meaning of larger vs.\ smaller values for the slope of a line, and a negative slope
        \end{enumerate}

    \clearpage
    \section{Einstein in Context}
      \subsection{To the Miracle Year}
        \paragraph{Youth in Munich}
        \paragraph{Stateless}
        \paragraph{Polytechnic on second try}
        \paragraph{Herr Professor Weber}
        \paragraph{Mileva}
        \paragraph{Graduation}
        \paragraph{Career anxieties}
        \paragraph{A friend in need}
        \paragraph{Lasting memories}
        \paragraph{Zigzagging}
        \paragraph{The Patent Office}
        \paragraph{The Olympia Academy}
        \paragraph{Lieserl}
        \paragraph{The miracle year}

      \subsection{The Miracle Year}
        \paragraph{March 1905: The light quantum idea, a “heuristic proposal”}
        \paragraph{April 1905: The size of molecules (doctoral dissertation)}
        \paragraph{May 1905: The existence of atoms (Brownian motion)}
        \paragraph{June 1905: On the electrodynamics of moving bodies (special relativity)}
        \paragraph{September 1905: $E = mc^2$}

  \chapter{Week 2: Events, Clocks, and Reference Frames}
    \section{Introduction}
      \subsection{Reading “The Electrodynamics of Moving Bodies”}
        \subsubsection{Introduction}
          \begin{itemize}
            \item Problems:
              \begin{enumerate}
                \item Asymmetries in Maxwell’s electrodynamics:
                  For example, contrary to the theory, no matter if the conductor is considered
                  moving or the magnet, an electric current is induced in the conductor by the magnet.
                \item Impossibility to detect the luminiferous ether
              \end{enumerate}
            \item Postulates, supported by previous experimental results:
              \begin{enumerate}
                \item Principle of relativity – laws of physics are the same in all inertial frames of reference
                \item Principle of light constancy – The speed of light is the same and constant in all inertial frames of reference
              \end{enumerate}
            \item Assumption: This solves both problems
            \item All electrodynamics deals with the kinematics of rigid bodies only, so Einstein’s too.
          \end{itemize}

        \subsubsection{Kinematics}
          \paragraph{§ 1. Definition of Simultaneity}
            \begin{itemize}
              \item To describe motion, one needs a measure of space and time.
              \item What is time?  Its definition depends on simultaneity,
                defined only by how light travels from an event to an observer.
              \item Conclusion: One can use light to measure times and lengths.
                Synchronize clocks by sending light from a clock to another and back.
                This way we can define and measure the speed of light.
            \end{itemize}

          \paragraph{§ 2. The Relativity of Lengths and Times}
            \begin{itemize}
              \item Clarification of the two postulates
              \item Rod moving along the x-axis.  What is the length of the moving rod?
                \begin{enumerate}[a)]
                  \item as measured using a meter by an observer moving with the rod (in his rest frame);
                  \item as measured using synchronized, stationary clocks;
                    the length of the rod at certain times is the distance
                    between the (imagined) clocks on the ends of the rod.
                \end{enumerate}
              \item Length (a) is the length of the rod (principle of relativity);
                length (b) is \emph{classically} assumed to be equal to length (a), but \emph{actually} is not.
              \item Ends of the rod are labeled “A” and “B”.
              \item Send light from A to B\@: this is related to the length of the rod, the relative speed of the rod, and the speed of light
              \item Observation: Observer in clocks’ rest frame would see clocks being synchronous, moving observer would not.
              \item Conclusion: Simultaneity is not an \emph{absolute} concept, but depends on the \emph{relative} motion.
            \end{itemize}

            \paragraph{§ 3. Coordinate Transformation From Rest Frame to Motion Frame}
              \begin{itemize}
                \item TODO
              \end{itemize}

    \clearpage
    \section{Events, Clocks, and Observers}

    \clearpage
    \section{Spacetime Diagrams}

    \clearpage
    \section{Frames of Reference}

    \clearpage
    \section{A Few More Words on World Lines}

    \clearpage
    \section{The Galilean Transformation}
      \paragraph{Priming the brain}
        Bob’s position at any given time $t$: $x(t) = \SI{3}{\meter\per\second} \times t$ \\
        Bob’s position with a head start of 4 m: $x(t) = \SI{4}{\meter} + \SI{3}{\meter\per\second} \times t$ \\
        In general: $ x(t) = x_0 + v \, t $

      \paragraph{Lecture}
        Question: Given a location and time of an event in Bob’s frame of reference,
        what is the corresponding location and time in Alice’s frame of reference?
        (Assuming that at $t = 0$, Alice and Bob are side-by-side at $x_A = 0, x_B = 0$.)

        \begin{align*}
          {(x_A)}_{flash} &= {(x_B)}_{flash} + v \; t \\
          {(x_B)}_{flash} &= {(x_A)}_{flash} - v \; t \\
          t_A &= t_B
        \end{align*}

    \section{Week 2 Summary}
      \begin{enumerate}
        \item Spacetime location of event → (x, y, z, t) or (x, t)
        \item How to specify the time? “Photo clock principle”
        \item How to synchronize clocks? Slow distribution or setting distant clocks ahead
        \item Spacetime diagrams
        \item Frames of reference, inertial frame of reference, velocities addition (classical)
        \item Galilean transformation
          \begin{align*}
            x_{Lab} &= x_{Rocket} + v \; t \qquad \text{Rocket is moving to the right}\\
            x_{Lab} &= x_{Rocket} - v \; t \qquad \text{Rocket is moving to the left}
          \end{align*}
      \end{enumerate}

  \chapter{Week 3: Ethereal Problems and Solutions}
    \section{Einsteins Starting Point: The Two Postulates}
      \paragraph{The June 1905 paper and the context leading up to it}
        \begin{itemize}
          \item Einstein already worked 8 to 10 years at least on the problem
          \item in April 1905, he went on a walk and talk with friend Michèle Besso,
            and shortly after said that “time is suspect”
          \item Other physicists working on the problem: Hendrik Antoon Lorentz, Henri Poincaré
        \end{itemize}
      \paragraph{The magnet and the coil}
      \paragraph{The two parts of the paper}
      \paragraph{The principle of relativity}
      \paragraph{The luminiferous ether}
      \paragraph{The principle of light constancy}

    \section{A Few Words About Waves}
      \subsection{Part 1}
        \subsubsection{Diagram}
        \subsubsection{Key terms}
          \paragraph{Medium}
            thing of which a wave is a propagating disturbance
          \paragraph{Periodic}
            something with a regularly repeated pattern; synonymous: harmonic wave
          \paragraph{Transverse}
            disturbance is perpendicular to the propagation of the wave
          \paragraph{Longitudinal}
            (density) disturbance propagates parallel to the direction of the wave
          \paragraph{Amplitude}
            absolute of the maximum value (for a transversal wave, the vertical distance to the rest position)
          \paragraph{Wavelength}
            $\lambda$: distance between two peaks or two troughs of the wave,
            or between two points with the same value and same ascent
          \paragraph{Period}
            $T$: time from one peak to the next one, measured \eg{} in seconds ($\mathrm{s}$)
          \paragraph{Frequency}
            $f$ or $\nu$ (Greek small letter Nu): how frequent are the up and down movements of the wave per unit time;
            $T = \frac{1}{f}$
          \paragraph{Velocity}
            \begin{itemize}
              \item Distance that the wave travels per unit time
              \item If we are told that there is a simple relationship between the velocity of the wave,
                its period and its wavelength, and we know the units in which they are measured
                ($[x] = \mathrm{y}$ means “the quantity $x$ is measured in the unit $\mathrm{y}$”),
                we can infer it thus:
                \begin{align*}
                  [v] &= \frac{\mathrm{m}}{\mathrm{s}}\\
                  [T] &= \mathrm{s}\\
                  T &= \frac{1}{f} \rightarrow~f = \frac{1}{T} \rightarrow~[f] = s^{-1}\\
                  [\lambda] &= \mathrm{m}\\
                  \rightarrow~v &= \lambda f = \frac{\lambda}{T}
                \end{align*}
            \end{itemize}

      \subsection{Part 2}
        \subsubsection{Diagram}
        \subsubsection{Key terms}
          \paragraph{Phase}
          \paragraph{Constructive interference}
            If waves are in phase, peaks and troughs add, respectively;
            the amplitude of the resulting wave is greater than each of the individual waves
          \paragraph{Destructive interference}
          If waves are out of phase, peaks and troughs subtract, respectively;
          the amplitude of the resulting wave is lower than each of the individual waves;
          worst case: they cancel out
          \paragraph{In phase}
            Two waves are “in phase” if they are in sync with each other
            (their peaks and troughs are aligned, respectively)
          \paragraph{Out of phase}
            Two waves are “out of phase” if they are not in sync with each other
            (their peaks and troughs are not aligned, respectively; worst case:
            peaks are aligned with troughs [180° out of phase])

      \subsection{Part 3}
        \begin{aside}
          Einstein and Picasso were contemporaries \parencite{miller2008einstein}.
        \end{aside}
        \begin{itemize}
          \item Three key facts about the speed of waves\footnote{Speed is the magnitude of velocity, but we are not making that distinction here}:
            \begin{enumerate}
              \item it depends on the medium
              \item moving source \rightarrow~no change in wave speed
                (because more/less peaks are generated per unit time
                and $v = \lambda \, f$), but change in wavelength (Doppler effect)\\
                example: 24 = 4 \times \, 6 = 3 \times \, 8
              \item moving medium \rightarrow~wave speed changes
            \end{enumerate}
        \end{itemize}

    \clearpage
    \section{The Michelson–Morley Experiment}
      \subsection{Part 1}
        \subsubsection{The goal}
          Albert Michelson (1881), Edward Morley (1887, now Case Western University, Ohio; more precise experiment):
          \emph{Detect the “ether wind”}

        \subsubsection{Math reminders}
          [more basic stuff already covered in the math review]
          \begin{align*}
            \sqrt{a^2 - b^2}
              &= \sqrt{a^2 \left(1 - \frac{b^2}{a^2}\right)}
               = \sqrt{a^2} \sqrt{1 - \frac{b^2}{a^2}}
               = a \sqrt{1 - \frac{b^2}{a^2}}
             \end{align*}
          \paragraph{Binomial expansion}\label{eq:binomial-expansion}
            (\emph{“physicists are using this all the time”})
            \begin{align*}
              {\left(1 + a\right)}^n &\approx 1 + n \, a\\
              {\left(1 - a\right)}^n &\approx 1 - n \, a\\
              {\left(1 + a\right)}^{-n} &\approx 1 - n \, a\\
              \intertext{if $a \ll 1$. For example, $a = \frac{v}{\c}$ and $v \ll c$.}
            \end{align*}
        \subsubsection{The idea behind the experiment (Earth’s motion through the luminiferous ether)}
          \begin{itemize}
            \item Assumption: The ether is a stationary medium, so as the Earth moves around, it moves through the ether
              \rightarrow~“ether wind”.
            \item Send light beam against and with direction of the “ether wind”
              \rightarrow~there should be a difference in observed frequency
          \end{itemize}

      \subsection{Part 2}
        \emph{Note}: The diagrams showing the A to B path of the airplane (and back again) are from the perspective of
        an observer looking down from above (like a map). In other words, imagine up is North, down is South,
        right is East, and left is West.

        \subsubsection{The airplane example}
          Plane travels from A and B and back\\
          D – distance between A and B\\
          $v_P$ – velocity of plane\\
          $v_W$ – velocity of wind blowing from B towards A\\

        \subsubsection{The “no wind” case}\label{sec:no-wind}
          \begin{align*}
            \text{Total time required by plane} = \frac{D}{v_P} + \frac{D}{v_P} = \frac{2D}{v_P}
          \end{align*}

        \subsubsection{The “headwind/tailwind” case}
          \begin{align*}
            \text{A to B trip (headwind): Time} &= \frac{D}{v_P - v_W}\\
            \text{B to A trip (tailwind): Time} &= \frac{D}{v_P + v_W}\\
            \text{Total time} &= \frac{D}{v_P - v_W} + \frac{D}{v_P + v_W}\\
            &= \frac{D\left(v_P + v_W\right) + D\left(v_P - v_W\right)}{\left(v_P - v_W\right) \left(v_P + v_W\right)}\\
            &= \frac{2 D v_P}{\left(v_P - v_W\right) \left(v_P + v_W\right)}\\
            &= \frac{2 D v_P}{{v_P}^2 + \cancel{v_P v_W} - \cancel{v_W v_P} + {v_W}^2}\\
            &= \frac{2 D v_P}{{v_P}^2 - {v_W}^2}\\
            &= \frac{2 D \cancel{v_P}}{{v_P}^{\cancel{2}} \left(1 - \frac{{v_W}^2}{{v_P}^2}\right)}\\
            &= \frac{2 D}{{v_P} \left(1 - \frac{{v_W}^2}{{v_P}^2}\right)}\\
            \text{Total time with headwind/tailwind} &= \left(\frac{2 D}{v_P}\right) \left(\frac{1}{1 - \frac{{v_W}^2}{{v_P}^2}}\right)\\
            \text{Total time without wind} &= \frac{2 D}{v_P} \qquad \text{(see~\ref{sec:no-wind})}
          \end{align*}

          The result of the “headwind/tailwind” case is \emph{obviously not equal}
          to that of the “no~wind” case.  So the time ``gained'' by flying with
          the wind does \emph{not} compensate for the time ``lost'' by flying
          against the wind earlier!  (To double-check the result, set $v_W = 0$
          \rightarrow~$\text{Total time} = \frac{2D}{v_P}$. QED.)

          \begin{samepage}
            Extreme example to demonstrate this more intuitively:
            \begin{align*}
              v_P &= \SI{300}{\kilo\meter\per\hour}\\
              v_W &= \SI{299}{\kilo\meter\per\hour}\\
              v_P - v_W &= \SI{1}{\kilo\meter\per\hour}\\
              v_P + v_W &= \SI{599}{\kilo\meter\per\hour} \qquad \text{returns 599 times as fast, \emph{not} twice as fast}
            \end{align*}
          \end{samepage}

          To do this with light and the Earth moving in the either, Michelson and Morley
          thought that because the speed of light is so large compared to the speed
          of Earth relative to the ether the increase in total traveling time could not
          be~measured.  As an alternative, they considered the “crosswind” case.

      \subsection{Part 3}
        \paragraph{The “crosswind” case}
        \paragraph{The flowing river analogy}
          If you try to swim across a river from A to B, you have to swim against
          the direction of the river flow to cross in a straight line ($v_P$ is
          diagonal to the flow $v_W$).
          \begin{align*}
            % \vec{v}_{actual} &= \vec{v}_P + \vec{v}_W\\
            \vec{v}_{actual} &\perp \vec{v}_W\\
            {v_P}^2 &= {v_W}^2 + {v_{actual}}^2 \qquad \text{Pythagorean theorem}\\
            {v_{actual}}^2 &= {v_P}^2 - {v_W}^2\\
            v_{actual} &= \sqrt{{v_P}^2 - {v_W}^2}
          \end{align*}

        \paragraph{The total time for the plane’s round trip}
          The river is always flowing/the wind is always blowing
          in the same direction.  So it does not matter if you are
          going from A to B or B to A: the total time required
          for the round trip is twice the time for each leg:
          \begin{align*}
            \text{Time} &= \frac{D}{v_{actual}}\\
            &= \frac{D}{\sqrt{{v_P}^2 - {v_W}^2}}\\
            \text{Total time} &= \frac{2D}{\sqrt{{v_P}^2 - {v_W}^2}}\\
            &= \frac{2D}{\sqrt{{v_P}^2\,\left(1 - \frac{{v_W}^2}{{v_P}^2}\right)}}\\
            &= \frac{2D}{v_P\,\sqrt{1 - \frac{{v_W}^2}{{v_P}^2}}}\\
            \text{Total time with crosswind} &= \left(\frac{2D}{v_P}\right)\left(\frac{1}{\sqrt{1 - \frac{{v_W}^2}{{v_P}^2}}}\right)
          \end{align*}

      \clearpage
      \subsection{Part 4}
        \paragraph{The experimental set-up}
          \begin{itemize}
            \item Path~A: Light beam crossing the ether wind (if any);
              then being reflected by a half-silvered mirror in the center
              towards the ether wind;
              reflected back by a full mirror, with the ether wind,
              through the half-silvered mirror,
              then going with the ether wind, onto the detector;
            \item Path~B: Light beam going through the half-silvered mirror,
              crossing the ether wind; being reflected back by full mirror,
              crossing the wind again;
              then reflected by the half-silvered mirror,
              then going with the ether wind, onto the detector.
          \end{itemize}
        \paragraph{Different travel times}
          \begin{align*}
            \text{Path~A time} &= \left(\frac{2D}{\c}\right)\left(\frac{1}{1 - \frac{{v_W}^2}{\c^2}}\right)\\
            \text{Path~B time} &= \left(\frac{2D}{\c}\right)\left(\frac{1}{\sqrt{1 - \frac{{v_W}^2}{\c^2}}}\right)\\
            \text{Path~A time} - \text{Path~B time}
              = \mdelta{t} &= \left(\frac{2D}{\c}\right)
                \left(
                  \frac{1}{\left(1 - \frac{{v_W}^2}{\c^2}\right)}
                  - \frac{1}{{\left(1 - \frac{{v_W}^2}{\c^2}\right)}^{\frac{1}{2}}}
                \right)\\
            \mdelta{t}  &= \left(\frac{2D}{\c}\right)
                \left(
                  {\left(1 - \frac{{v_W}^2}{\c^2}\right)}^{-1}
                  - {{\left(1 - \frac{{v_W}^2}{\c^2}\right)}}^{-\frac{1}{2}}
                \right)
            \intertext{Binomial expansion (\ref{eq:binomial-expansion}),
            because with ${v_W \approx \SI{29.78}{\kilo\meter\per\second}}$
            (orbital speed of Earth), ${\frac{{v_W}^2}{\c^2} \ll 1}$:}
            \mdelta{t} &\approx \left(\frac{2D}{\c}\right)
              \left(
                \left(1 - \left(-1\right)\left(\frac{{v_W}^2}{\c^2}\right)\right)
                - \left(1 - \left(-\frac{1}{2}\right)\left(\frac{{v_W}^2}{\c^2}\right)\right)
              \right)\\
            &= \left(\frac{2D}{\c}\right)
              \left(
                \left(1 + \frac{{v_W}^2}{\c^2}\right)
                - \left(1 + \frac{1}{2} \frac{{v_W}^2}{\c^2}\right)
              \right)\\
            &= \left(\frac{2D}{\c}\right)
              \left(
                1 + \frac{{v_W}^2}{\c^2}
                - 1 - \frac{1}{2} \frac{{v_W}^2}{\c^2}
              \right)\\
            &= \left(\frac{2D}{\c}\right)
              \left(\frac{1}{2} \frac{{v_W}^2}{\c^2}\right)\\
            \mdelta{t} &\approx \left(\frac{D}{\c}\right)
              \left(\frac{{v_W}^2}{\c^2}\right).
          \end{align*}

        \paragraph{Some actual numbers}
          For the experiment, $D \approx \SI{11}{\meter}$, so
          $\mdelta{t} \approx \SI{3.67e-16}{\second}$.

          If you rotate the table by 90°, you can get differences of
          ca. \SI{7e-16}{\second}.\textsuperscript{[why?]} But this is still
          too short a time to measure with clocks.

        \paragraph{The key technique Michelson and Morley used, and the “null result”}
          \begin{itemize}
            \item \begin{samepage}Key technique: Interference patterns.
              If there is no time difference, the light waves of the beams
              of the two paths are in phase \rightarrow~constructive interference.

              They used sodium (\chem{Na}) light:
              \begin{align*}
                \lambda &= \SI{508}{\nano\meter}\\
                f &= \frac{\c}{\lambda}\\
                T &= \frac{1}{f} = \frac{\lambda}{\c} = \SI{2e-15}{\second}
              \end{align*}
              The time difference \mdelta{t} to be expected because of
              the ether wind (if any) is smaller than that, so if one wave
              is out of phase (\emph{“fringe shift”}) by only
              \begin{align*}
                \frac{\mdelta{t}}{T} &= \frac{\SI{7e-16}{\second}}{\SI{2e-15}{\second}} = 0.35
              \end{align*}
              of a wavelength, you can see a different interference pattern
              (\emph{“fringe pattern”}) in the detector.\end{samepage}
            \item But they observed, with different orientations, at different
              times of the year, at most a \strong{fringe shift of
              only 0.005, well within experimental error
              \rightarrow~“null result”.}
            \item Physicists of the time were surprised.  Interpretation:
              Is there no ether at all {--} or has it properties that make it
              undetectable?
            \item Possible explanation: Earth could drag the ether with it
              when it is rotating, so you could not measure the ether wind on
              the surface of Earth (ether dragging).  But this had been shown
              \emph{not} to be the case \emph{either}, by
              \rightarrow~stellar aberration.
          \end{itemize}

    \section{Stellar Aberration}
      \paragraph{Case 1: Ether dragged by Earth as it travels around the Sun}
        Light gets straight into the telescope as it Earth is rotating,
        because the ether is dragged along by the rotation of Earth.
        No correction necessary.

      \paragraph{Case 2: Ether not dragged by Earth as it travels around the Sun}
        Light does not get straight into the telescope as it Earth is rotating,
        bcause the ether is not dragged along by the rotation of Earth, and
        the light travels in a straight line.  Before it reaches the observer,
        he and his telescope have already changed position because
        of the moving, rotating Earth.  The telescope must be slightly tilted
        ahead so that the light gets in straight.

    \section{Ethereal Solutions}
      \paragraph{G.F. Fitzgerald}
        \begin{itemize}
          \item Assumption: Compression effect of the measuring apparatus when moving against the ether
          \item sent letter to American journal, not noticed
        \end{itemize}

      \paragraph{H.A. Lorentz}
        \begin{itemize}
          \item he had the same idea as Fitzgerald, and gave credit to Fitzgerald
          \rightarrow~Lorentz–Fitzgerald contraction
        \end{itemize}

      \paragraph{The Lorentz-Fitzgerald contraction hypothesis}

      \paragraph{Henri Poincaré}
      \begin{itemize}
        \item built on the Lorentz-Fitzgerald contraction hypothesis
        \item Idea: Time may be changing
      \end{itemize}

      \paragraph{Einstein’s approach}
        Not certain that he was influenced by the MM experiment
        \subparagraph{Combining the principle of light constancy with the principle of relativity}

        \subparagraph{Einstein’s conclusion about the speed of light}
        \subparagraph{Einstein’s conclusion about the ether}

  \chapter{Week 4: The Weirdness Begins}
    \section{Introduction}
      \paragraph{Our journey up to this point}
      \paragraph{Quotes of the week}
      \paragraph{\textquote{Time is suspect}}

    \section{\textquote{Time is suspect}: The Relativity of Simultaneity}
      \subsection{Diagram 1: Alice and Bob stationary}
      \subsection{Diagram 2: Paintball experiment (Alice moving, Bob observing)}
      \subsection{Diagram 3: Light pulse experiment (Alice moving, Bob observing)}
      \subsection{Summary}
        \textquote{Leading clocks lag}

    \section{The Light Clock, and Exploring the Lorentz Factor}
      \subsection{The light clock, part 1}
      \subsection{The light clock, part 2}
        \begin{align*}
          \intertext{One tick of Alice’s clock as she sees it (light pulse up and down again):}
          \mdelta{t_A} &= \frac{2\,L}{\c}
          \intertext{One tick of Bob’s clock as he sees it:}
          \mdelta{t_B} &= \frac{2\,L}{\c}
          \intertext{One tick of Alice’s clock as Bob sees it:}
          \mdelta{t_B} &= \frac{2\,D}{\c}\\
          \text{qualitatively: }\mdelta{t_B} &> \mdelta{t_A}\\
          \\
          D^2 &= L^2 + {\left(\frac{x}{2}\right)}^2 \\
          D &= \frac{\c\,\mdelta{t_B}}{2}\\
          L &= \frac{\c\,\mdelta{t_A}}{2}\\
          x &= v \, \mdelta{t_B}\\
          \\
          \frac{\c^2\,{\mdelta{t_B}}^2}{4} &= \frac{\c^2\,{\mdelta{t_A}}^2}{4} + \frac{v^2\,{\mdelta{t_B}}^2}{4}\\
          \c^2\,{\mdelta{t_B}}^2 &= \c^2\,{\mdelta{t_A}}^2 + v^2\,{\mdelta{t_B}}^2\\
          \c^2\,{\mdelta{t_B}}^2 - v^2\,{\mdelta{t_B}}^2 &= \c^2\,{\mdelta{t_A}}^2\\
          {\mdelta{t_B}}^2 \left(\c^2 - v^2\right) &= \c^2\,{\mdelta{t_A}}^2\\
          {\mdelta{t_B}}^2 &= \frac{\c^2\,{\mdelta{t_A}}^2}{\c^2 - v^2}\\
          &= \frac{\c^2}{\c^2 - v^2} \, {\mdelta{t_A}}^2\\
          {\mdelta{t_B}}^2 &= \frac{1}{1 - \frac{v^2}{\c^2}} \, {\mdelta{t_A}}^2\\
          \mdelta{t_B} &= \sqrt{\frac{1}{1 - \frac{v^2}{\c^2}}} \, \mdelta{t_A}\\
          \mdelta{t_B} &= \frac{1}{\sqrt{1 - \frac{v^2}{\c^2}}} \, \mdelta{t_A}\\
          \gamma &= \frac{1}{\sqrt{1 - \frac{v^2}{\c^2}}} \geq 1 \qquad \text{Lorentz factor}\\
          \mdelta{t_B} &>^? \mdelta{t_A} \cmark \qquad \text{Time dilation}
        \end{align*}

        \begin{samepage}
          \begin{aside}
            My approach:
            \begin{align*}
              D &= \sqrt{L^2 + {\left(\frac{x}{2}\right)}^2} \\
              \mdelta{t_B} &= \frac{2\,\sqrt{L^2 + {\left(\frac{x}{2}\right)}^2} }{\c}\\
              L &= \frac{\mdelta{t_A}\,\c}{2}\\
              L^2 &= \frac{{\mdelta{t_A}}^2\,\c^2}{4}\\
              \mdelta{t_B} &= \frac{2\,\sqrt{\frac{{\mdelta{t_A}}^2\,\c^2}{4} + {\left(\frac{x}{2}\right)}^2} }{\c}\\
                &= \frac{\sqrt{\frac{4\,{\mdelta{t_A}}^2\,\c^2}{4} + 4\,{\left(\frac{x}{2}\right)}^2} }{\c}\\
                &= \frac{\sqrt{{\mdelta{t_A}}^2\,\c^2 + 4\,{\left(\frac{x}{2}\right)}^2} }{\c}\\
                &= \frac{\sqrt{{\mdelta{t_A}}^2\,\c^2 + 4\,\left(\frac{x^2}{4}\right)} }{\c}\\
                &= \frac{\sqrt{{\mdelta{t_A}}^2\,\c^2 + x^2} }{\c}\\
                &= \frac{\c^2\,\sqrt{{\mdelta{t_A}}^2\, + \frac{x^2}{\c^2}} }{\c}\\
                &= \c\,\sqrt{{\mdelta{t_A}}^2\, + \frac{x^2}{\c^2}}\\
                \mdelta{t_B} &= \c\,\sqrt{{\mdelta{t_A}}^2\, + \frac{v^2\,{\mdelta{t_B}}^2}{\c^2}}\\
                \mdelta{t_B} &= \c\,\sqrt{{\mdelta{t_A}}^2\, + \frac{v^2\,{\mdelta{t_B}}^2}{\c^2}}\\
            \end{align*}
          \end{aside}
        \end{samepage}

    \section{Time Dilation}
      \paragraph{Duration of one click tick on moving clock compared to clock at rest}
        \begin{align*}
          {\left(\mdelta{t}\right)}_{\text{moving}} &= \gamma\,{\left(\mdelta{t}\right)}_{\text{rest}}
        \end{align*}
      \paragraph{Elapsed time on moving clock compared to clock at rest}
        \begin{align*}
          {\left(\text{Elapsed time}\right)}_{\text{moving}} &= \frac{1}{\gamma}\,{\left(\text{Elapsed time}\right)}_{\text{rest}}\\
          {\left(\mdelta{T}\right)}_{\text{moving}} &= \frac{1}{\gamma}\,{\left(\mdelta{T}\right)}_{\text{rest}}
        \end{align*}

        \emph{Note: This does \underline{not} mean that time slows down \underline{for you} if you are moving.}

      \paragraph{1. Is the light clock simply a special case?
        2. Does time dilation occur for all types of clocks?}
        1. No; 2. Yes.

        Because if Alice had a normal clock, and the light clock were special,
        the former would get out of sync with her light clock if, and only if,
        she were moving (because that is what Bob would have to observe, too).

        But this would violate the special principle of relativity (which says
        that if you are moving uniformly the laws of physics apply exactly as if
        you were at rest; IOW, there is no experiment that allows you
        to determine if you are moving uniformly or not; IYOW, there is no
        absolute velocity; velocity depends on the frame of reference).

    \section{Measuring Length (length contraction)}
      Alice’s result for the length of Bob’s ship:
      \begin{align*}
        {\left(L_B\right)}_{Alice} &= v \left(T_{A2} - T_{A1}\right) = v\,\mdelta{T_A}
      \end{align*}

      Bob’s result for the length of his ship:
      \begin{align*}
        {\left(L_B\right)}_{Bob} &= v \left(T_{B2} - T_{B1}\right) = v\,\mdelta{T_B}
      \end{align*}

      For Bob:
      \begin{align*}
        \mdelta{T_A} &= \frac{1}{\gamma} \mdelta{T_B}\\
        {\left(L_B\right)}_{Alice} &= v \, \mdelta{T_A}\\
        &= v\,\frac{1}{\gamma} \mdelta{T_B}\\
        &= \frac{1}{\gamma} \, \mdelta{T_B}\,v\\
        {\left(L_B\right)}_{Alice} &= \frac{1}{\gamma} {\left(L_B\right)}_{Bob}
      \end{align*}

      \subsubsection{Summary}
        Time dilation:
        \begin{align*}
          {\left(\mdelta{T}\right)}_{moving} &= \frac{1}{\gamma} {\left(\mdelta{T}\right)}_{rest} \qquad  \text{Time dilation}\\
          \mdelta{T}\text{ – \underline{elapsed} time}\\
          \mdelta{T}_{rest}\text{ – Proper time}
        \end{align*}

        Time dilation:
        \begin{align*}
          L_{moving} &= \frac{1}{\gamma} \, L_{rest} \qquad \text{Length contraction}\\
          L_{rest}\text{ – Proper length}
        \end{align*}

    \section{What is Not Suspect, and the Invariant Interval}
      \subsection{What is not suspect (invariants)}
        \strong{Lengths not in the direction of motion stay are not shortened.}

        If it were different,
        \begin{itemize}
          \item as to \strong{width}, Alice would observe the wheels of Bob’s
            moving train cars to fall off as its width reduces, while Bob
            would see the width of the tracks get narrower;
          \item as to \strong{height}, supposed there would be a tunnel through
            which Bob’s train car would just fit while at rest, then he would
            observe that his car would into that tunnel as it its height were
            reduced, while Alice would observe that he has more room to get
            through the tunnel.
        \end{itemize}

        Either case would be a contradiction; if you take a photo while on
        the car, both Alice and Bob must agree as to its status.

      \subsection{The invariant interval}\label{sec:invariant-interval}
        \begin{samepage}
          Consider the time and distance between two events in, as observed
          in different frames of reference:

          [see handout]
          \begin{align*}
            h &\text{ {--} height of the light clocks}\\
            x_A &\text{ {--} distance between Bob's events for Alice}\\
            x_K &\text{ {--} distance between Bob's events for Kris}\\
            t_A &\text{ {--} one tick of Bob's clock from Alice's perspective}\\
            t_K &\text{ {--} one tick of Bob's clock from Kris's perspective}\\
            t_B &\text{ {--} one tick of Bob's clock according to him}\\
            \\
            h^2 + {\left(\frac{x_A}{2}\right)}^2 &= {\left(\frac{\c\,t_A}{2}\right)}^2\\
            h^2 + \frac{{x_A}^2}{4} &= \frac{\c^2\,{t_A}^2}{4}\\
            4\,h^2 + {x_A}^2 &= \c^2\,{t_A}^2\\
            4\,h^2 &= \c^2\,{t_A}^2 - {x_A}^2\\
            4\,h^2 &= \c^2\,{t_K}^2 - {x_K}^2\\
            2\,h &= \c \, \frac{t_B}{2} + \c \, \frac{t_B}{2}\\
            &= \c\,t_B\\
            4\,h^2 &= \c^2\,{t_B}^2
            \intertext{or}
            4\,h^2 &= \c^2\,{t_B}^2 - {x_B}^2
            \intertext{but $x_B = 0$.}
          \end{align*}
        \end{samepage}

    \clearpage
    \section{A Real-Life Example: The Muon}
      \begin{align*}%
        \text{Average muon (proper) lifetime: } \tau &= \SI{2.2}{\micro\second}\\
        \text{Average velocity: } v &= \SI{0.998}{\c}\\
        \text{\rightarrow~Distance traveled: } s &\approx \SI{660}{\meter}\\
        \text{Altitude where they are generated by cosmic rays: } h &\approx \SI{10}{\kilo\meter}
      \end{align*}
      But a lot of atmospheric muons are detected near
      Terra's surface.  How is that possible?

      \emph{Special relativity!}
      \begin{align*}
        \gamma &= \frac{1}{\sqrt{1 - \frac{v^2}{\c^2}}} \approx 15
        \intertext{From our perspective, time dilation:}
        \tau_{observer} &= \gamma \, \tau \\
          &= 15 \times \SI{2.2}{\micro\second}\\
          &= \SI{33}{\micro\second}\\
        s_{observer} &= v \, \tau_{observer}\\
          &= \SI{0.998}{\c} \times \SI{33}{\micro\second}\\
          &\approx \SI{9.873}{\kilo\meter} \approx h. \qed
        \intertext{From the muon's perspective, length contraction:}
        h_{\mu} &= \frac{1}{\gamma} \, h\\
          &= \frac{\SI{10}{\kilo\meter}}{15} \\
          &\approx \SI{667}{\meter}
          \approx s. \qed
      \end{align*}

  \chapter{Week 5: Spacetime Switches}
    \section{Quotations of the week}
    \section{Convenient units for the speed of light}
      \paragraph{Calculating the Lorentz factor when writing the velocity in terms of \c{} (e.g.,~$v~=~\SI{0.9}{\c}$)}
        \begin{align*}
          \gamma &= \frac{1}{\sqrt{1 - \frac{v^2}{\c^2}}}\\
          \gamma(v = \SI{0.9}{\c}) &= \frac{1}{\sqrt{1 - \frac{{\left(\SI{0.9}{\c}\right)}^2}{\c^2}}}\
            = \frac{1}{\sqrt{1 - \frac{{\left(0.9\right)}^2{\c}^2}{\c^2}}}\
            = \frac{1}{\sqrt{1 - {\left(0.9\right)}^2}}\\
        \end{align*}
        1~light-year~[\SI{1}{\lightyear}] is the distance that light travels in vacuum
        in 1~year [\SI{1}{\year}].
        in 1~year [\SI{1}{a}].
        \begin{align*}
          \c_0 &= \SI{1}{\frac{\lightyear}{\year}}
          \c_0 &= 1\,\frac{\text{ly}}{\text{a}}
        \end{align*}
      \paragraph{The speed of light in meters/second, km/second, light-years/year, feet/nanosecond, and so on}
        \begin{align*}
          \c_0 &\approx \SI{1}{\frac{ft}{\nano\second}}
          \c_0 &\approx 1\,\frac{\text{ft}}{\text{ns}}
        \end{align*}

    \section{Exploring time dilation and length contraction (Star Tours, part 1)}
      \paragraph{“Star Tours” set-up:} A trip to a star 5 light-years away.
        \begin{align*}
          v &= \SI{0.943}{\c} \rightarrow \gamma = 3
        \end{align*}

      \paragraph{Analysis done from the perspective of the observer on Earth (the Earth-Star reference frame)}
        \begin{align*}
          \text{Travel time } \mdelta{T}_{rest} &= \frac{\SI{5}{\lightyear}}{\SI{0.943}{\c}} = \SI{5.3}{\year}\\
          \mdelta{T}_{moving} &= \frac{1}{\gamma} \mdelta{T}_{rest} = \frac{\SI{5.3}{\year}}{\gamma} \approx \SI{1.77}{\year}
          \mdelta{T}_{moving} &= \frac{1}{\gamma} \mdelta{T}_{rest} = \frac{\SI{5.3}{a}}{\gamma} \approx \SI{1.77}{a}
        \end{align*}

      \paragraph{Analysis done from the perspective of the observer on the rocket (the rocket reference frame)}
        Observes that the Earth{--}Star distance is contracted
        \begin{align*}
          L_{moving} &= \frac{1}{\gamma} L_{rest} = \frac{\SI{5}{\lightyear}}{\gamma} \approx \SI{1.67}{\lightyear}\\
          \text{Time for star to reach Bob } \mdelta{T}_{Rocket} &= \frac{\SI{1.67}{\lightyear}}{\SI{0.943}{\c}} = \SI{1.77}{\year}(!)\\
          \text{Time for star to reach Bob } \mdelta{T}_{Rocket} &= \frac{\SI{1.67}{ly}}{\SI{0.943}{\c}} = \SI{1.77}{a}(!)\\
        \end{align*}

      \paragraph{The puzzle/conundrum}
        \begin{align*}
            &= \frac{\SI{1.77}{\year}}{\gamma} \approx \SI{0.59}{\year} \text{ (???)}
            &= \frac{\SI{1.77}{a}}{\gamma} \approx \SI{0.59}{a} \text{ (???)}
        \end{align*}
        (Hint: relativity of simultaneity, ``leading clocks lag'')

    \section{Deriving the Lorentz transformation}
      \subsection{Part 1}
        Bob in a spaceship moving relative to Alice

        At start:
        \begin{align*}
          t_A = 0 \qquad t_B = 0\\
          x_A = 0 \qquad x_B = 0
        \end{align*}
        \paragraph{The question/goal}
          Given $(x_B, \, y_B, \, z_B, \, t_B)$,
          what is $(x_A, \, y_A, \, z_A, \, t_A)$?

          Or: $(x_A, \, y_A, \, z_A, \, t_A) \leftrightarrow (x_B, \, y_B, \, z_B, \, t_B)$

        \paragraph{Reminder of the Galilean transformation}
          \begin{align*}
            x_A &= x_B + v \, t_B\\
            y_A &= y_B\\
            z_A &= z_B\\
            t_A &= t_B
          \end{align*}

        \paragraph{Bob and Alice and the flash of light}
          Later, at time $t_B$, there is a flash of light in Bob's cockpit.

          We know:
          \begin{align*}
            t_A &= \gamma \, t_B, \, \gamma = \frac{1}{\sqrt{1 - \frac{v^2}{\c^2}}}
          \end{align*}
          \emph{But note:} This was derived for $x_B = 0$.

        \paragraph{Applying the invariant interval equation}
          We also know:
          \begin{align*}
            \c^2 \, {t_A}^2 - {x_A}^2 &= \c^2 \, {t_B}^2 - {x_B}^2 \qquad \text{Invariant interval}
          \end{align*}

          Consider first the case when the flash of light occurs at $x_B = 0$:
          \begin{align*}
             {\left(x_B\right)}_{flash} &= 0\\
             {\left(x_A\right)}_{flash} &= v \, t_A
             \intertext{Because $x_B = 0$, we can apply the known time dilation equation:}
             {\left(x_A\right)}_{flash} &= \gamma \, v \, t_B
              \c^2 \, {t_A}^2 - {x_A}^2 &= \c^2 \, {t_B}^2 - \cancelto{0}{{x_B}^2}\\
              \c^2 \, {t_A}^2 - {x_A}^2 &= \c^2 \, {t_B}^2 - \cancel{{x_B}^2} 0\\
              \c^2 \, {t_A}^2 - {\left(v \, t_A\right)}^2 &= \c^2 \, {t_B}^2\\
              \c^2 \, {t_A}^2 - v^2 \, {t_A}^2 &= \c^2 \, {t_B}^2\\
              {t_A}^2 \left(\c^2 - v^2\right) &= \c^2 \, {t_B}^2\\
              {t_A}^2 &= \frac{\c^2}{\c^2 - v^2} \, {t_B}^2\\
                &= \frac{\cancel{\c^2}}{\cancel{\c^2} \left(1 - \frac{v^2}{\c^2}\right)} \, {t_B}^2\\
              {t_A}^2 &= \frac{1}{1 - \frac{v^2}{\c^2}} \, {t_B}^2\\
              t_A &= \frac{1}{\sqrt{1 - \frac{v^2}{\c^2}}} \, t_B\\
              \Aboxed{t_A &= \gamma \, t_B}\\
              x_A &= v \, t_A\\
              \Aboxed{x_A &= \gamma \, v \, t_B.}
          \end{align*}

      \subsection{Part 2}
        \paragraph{Review of Part 1 and the overall goal}
          Goal: For an event at $(x_B, \, t_B)$, what are $x_A$ and $t_A$?
          That is
          \begin{align*}
            x_A &= \text{some formula involving $x_B$ and $t_B$}\\
            t_A &= \text{some formula involving $x_B$ and $t_B$}
          \end{align*}

        \paragraph{Restrictions on the possible formula}
          \begin{enumerate}
            \item The units have to match, so e.g., $x_A = {x_B}^2 + \gamma \, t_B$ cannot
              be correct.
            \item It has to work for $x_B = 0$
          \end{enumerate}

        \paragraph{Why the clock ticks have to be uniform}
          \begin{align*}
            t_A = \frac{{t_B}^2}{\frac{x_B}{v}} \text{ could work dimensionally.}
            \intertext{Why can there not be a ${t_B}^2$ on the right-hand side?  Assume}
            t_A &= {t_B}^2,
            t_B &= 0, \, 1, \, 2, \, 3 \rightarrow 1, \, 2, \, 3, \, 4\\
            t_A &= 0, \, 1, \, 4, \, 9 \rightarrow 1, \, 4, \, 9, \, 16
            t_A &= 0, \, 1, \, 4, \, 9 \rightarrow 1, \, 4 \, 9, \, 16
          \end{align*}
          \emph{The clock ticks have to be uniform, but would not be for later times.}

        \paragraph{A linear form for the desired equations}
          \begin{samepage}
            Therefore, the transformation has to have a linear form:
            \begin{align*}
              t_A &= G \, x_B + H \, t_B\\
              x_A &= M \, x_B + N \, t_B
            \end{align*}
          \end{samepage}

        \paragraph{Finding two of the four unknown coefficients (G, H, M, N)}
          If $x_B = 0$, $t_A = \gamma \, t_B$ and $x_A = \gamma \, v \, t_B$ and
          \begin{align*}
            t_A &= G \, \cancelto{0}{x_B} + H \, t_B = \gamma \, t_B \rightarrow~H = \gamma{}\\
            x_A &= M \, \cancelto{0}{x_B} + N \, t_B = \gamma \, v \, t_B \rightarrow~N = \gamma \, v
            x_A &= M \, \cancel{x_B} 0 + N \, t_B = \gamma \, v \, t_B \rightarrow~N = \gamma \, v
            \intertext{so:}
            t_A &= G \, x_B + \gamma \, t_B\\
            x_A &= M \, x_B + \gamma \, v \, t_B
          \end{align*}

      \subsection{Part 3}
        \paragraph{Finding the remaining coefficients (G and M) using the invariant interval equation}
          \begin{align*}
            \c^2 \, {t_A}^2 - {x_A}^2 &= \c^2 \, {t_B}^2 - {x_B}^2
            \intertext{Insert the terms for $x_A$ and $t_A$:}
            \c^2 \, {(G \, x_B + \gamma \, t_B)}^2 - {(M \, x_B + \gamma \, v \, t_B)}^2 &= \c^2 \, {t_B}^2 - {x_B}^2\\
            \c^2 \, (G^2 \, {x_B}^2 + 2 \, G \, x_B \gamma \, t_B + {\gamma}^2 {t_B}^2)
              - (M^2 \, {x_B}^2 + 2 \, M \, \gamma \, v \, x_B \, t_B + {\gamma}^2 \, v^2 \, {t_B}^2)
              &= \c^2 \, {t_B}^2 - {x_B}^2\\
            \c^2 \, (G^2 \, {x_B}^2 + 2 \, G \, \gamma \, x_B \, t_B + {\gamma}^2 {t_B}^2)
              - M^2 \, {x_B}^2 - 2 \, M \, \gamma \, v \, x_B \, t_B  - {\gamma}^2 \, v^2 \, {t_B}^2
              &= \c^2 \, {t_B}^2 - {x_B}^2\\
            (\c^2 \, {\gamma}^2 {t_B}^2 - {\gamma}^2 v^2 {t_B}^2)
              + (2 \, G \, \gamma \, \c^2 \, x_B \, t_B - 2 \, M \, \gamma \, v \, x_B \, t_B)
              + (\c^2 \, G^2 {x_B}^2 - M^2 \, {x_B}^2) &= \c^2 \, {t_B}^2 - {x_B}^2\\
            (\c^2 \, {\gamma}^2 - {\gamma}^2 v^2) \, {t_B}^2
              + (2 \, G \, \gamma \, \c^2 - 2 \, M \, \gamma \, v) \, x_B \, t_B
              + (\c^2 \, G^2 - M^2) \, {x_B}^2 &= \c^2 \, {t_B}^2 - {x_B}^2\\
          \end{align*}
          Therefore:
          \begin{align*}
            \c^2 \, {\gamma}^2 - {\gamma}^2 v^2 &= c^2\\
            {\gamma}^2 \, (\c^2 - v^2) &= c^2\\
            {\gamma}^2 &= \frac{c^2}{\c^2 - v^2}\\
            {\gamma}^2 &= \frac{1}{1 - \frac{v^2}{c^2}}\\
            {\gamma} &= \frac{1}{\sqrt{1 - \frac{v^2}{c^2}}} \qed\\
            2 \, G \, \gamma \, \c^2 - 2 \, M \, \gamma &= 0\\
            (2 \, \gamma) (G \, \c^2 - \, M \, v) &= 0\\
            \gamma \neq 0 \rightarrow~G \, \c^2 - \, M \, v &= 0\\
            G \, \c^2 &= M \, v\\
            G &= \frac{M \, v}{\c^2}\\
            \c^2 \, G^2 - M^2 &= -1\\
            -\c^2 \, G^2 + M^2 &= 1\\
            M^2 - \c^2 \, G^2 &= 1
            \intertext{Insert $G$:}
            M^2 - \c^2 \, {\left(\frac{M \, v}{\c^2}\right)}^2 &= 1\\
            M^2 - \c^2 \, \frac{M^2 \, v^2}{\c^4} &= 1\\
            M^2 - \frac{M^2 \, v^2}{\c^2} &= 1\\
            M^2 \, \left(1 - \frac{v^2}{\c^2}\right) &= 1\\
            M &= \frac{1}{\sqrt{1 - \frac{v^2}{\c^2}}} = \gamma{}\\
            M &= \frac{1}{\sqrt{1 - \frac{v^2}{\c^2}}} = \gamma{}\\
            G &= \frac{M \, v}{\c^2} = \frac{\gamma \, v}{\c^2}
          \end{align*}

        \paragraph{The final form of the Lorentz transformation}
          \begin{align*}
            t_A &= G \, x_B + \gamma \, t_B\\
            t_A &= \frac{\gamma \, v}{\c^2} \, x_B + \gamma \, t_B\\
            x_A &= M \, x_B + \gamma \, v \, t_B\\
            x_A &= \gamma \, x_B + \gamma \, v \, t_B
          \end{align*}
          \begin{align*}
            \Aboxed{t_A &= \gamma \, \left(t_B + \frac{v}{\c^2} \, x_B\right)}\\
            \Aboxed{x_A &= \gamma \, \left(x_B + v \, t_B\right)}
          \end{align*}

    \section{Exploring the Lorentz transformation}
      \subsection{Part 1}
        \paragraph{Summary of results so far}
          Rewriting the transformation equations in order to avoid confusion:
          \begin{align*}
            \Aboxed{t_{rest} &= \gamma \, \left(t_{moving} + \frac{v}{\c^2} \, x_{moving}\right)}\\
            \Aboxed{x_{rest} &= \gamma \, \left(x_{moving} + v \, t_{moving}\right)}
          \end{align*}

        \paragraph{What happens when $v = 0$?}
          \begin{samepage}
            Let $v = 0$ → $\gamma = 1$, then:
            \begin{align*}
              t_{rest} &= \cancelto{1}{\gamma} \, \left(t_{moving} + \cancelto{0}{\frac{\cancel{v} \, 0}{\c^2} \, x_{moving}}\right)\\
              t_{rest} &= \cancel{\gamma} \, 1 \, \left(t_{moving} + \cancel{\frac{\cancel{v} \, 0}{\c^2} \, x_{moving}}\right)\\
              x_{rest} &= \cancelto{1}{\gamma} \, \left(x_{moving} + \cancelto{0}{\cancel{v} \, t_{moving}}\right)\\
              x_{rest} &= \cancel{\gamma} \, 1 \, \left(x_{moving} + \cancel{\cancel{v} \,  0 \, t_{moving}}\right)\\
              x_{rest} &= x_{moving}
            \end{align*}
            \emph{[Observers not moving relative to each other agree on times and locations.]}
          \end{samepage}
        \paragraph{What happens when $v$ is much less than \c{} (the speed of light)?}
          Let $v \ll \c$ → $\gamma \approx 1$ and $\frac{v}{\c^2} \approx 0$, then:
          \begin{align*}
            t_{rest} &= \cancelto{1}{\gamma} \, \left(t_{moving} + \cancelto{0}{\frac{\cancel{v}}{\c^2} \, x_{moving}}\right)\\
            t_{rest} &= \cancel{\gamma} \, 1 \, \left(t_{moving} + \cancel{\frac{\cancel{v} \, 0}{\c^2} \, x_{moving}}\right)\\
            x_{rest} &= \cancelto{1}{\gamma} \, \left(x_{moving} + v \, t_{moving}\right)\\
            x_{rest} &= \cancel{\gamma} \, 1 \, \left(x_{moving} + v \, t_{moving}\right)\\
            x_{rest} &= x_{moving} + v \, t_{moving}
          \end{align*}
          \emph{This is the Galilean transformation.}
        \paragraph{What happens when $v$ is large?}
          At $t_{moving} = 0$, we get \emph{length contraction}:
          \begin{align*}
            x_{rest} &= \gamma \, \left(x_{moving} + \cancelto{0}{\cancel{v} \, t_{moving}}\right)\\
            x_{rest} &= \gamma \, \left(x_{moving} + \cancel{\cancel{v} \, 0 \, t_{moving}}\right)\\
            x_{rest} &= \gamma \, x_{moving}\\
            x_{moving} &= \frac{x_{rest}}{\gamma}
          \end{align*}

      \subsection{Part 2: Exploring the time transformation equation}
        \paragraph{What happens when $x = 0$ (as measured in the moving frame of reference)?}
          If $x_{moving} = 0$, we get \emph{time dilation}:
          \begin{align*}
            t_{rest} &= \gamma \, t_{moving} + \cancelto{0}{\gamma \, \frac{v}{\c^2} \, \cancel{x_{moving}}}\\
            t_{rest} &= \gamma \, t_{moving} + \cancel{\gamma \, \frac{v}{\c^2} \, 0 \, \cancel{x_{moving}}}\\
            t_{rest} &= \gamma \, t_{moving}\\
            t_{moving} &= \frac{t_{rest}}{\gamma}
          \end{align*}
        \paragraph{The result (leading clocks lag)}
          From
          \begin{align*}
            t_{rest} &= \gamma \, t_{moving} + \gamma \, \frac{v}{\c^2} \, x_{moving}
          \end{align*}
          also follows: For the same time $t_{rest}$ measured on the clocks
          at rest, as $x_{moving}$ is increasing, $t_{moving}$ has to decrease.
          IOW, the farther away the moving clocks are, the more are they behind
          the clocks at rest: \emph{Leading clocks lag.}

      \subsection{Part 3: The inverse transformation (movement to the left, or negative $x$ direction)}
        [This is easier to understand with Alice at rest and Bob moving,
         hence the old indexes here.]
        \begin{align*}
          t_A &= \gamma \, \left(t_B + \frac{v}{\c^2} \, x_B\right)\\
          x_A &= \gamma \, \left(x_B + v \, t_B\right)
        \end{align*}
        From Bob's perspective, he is at rest and Alice is moving in the negative
        $x$ direction.

        If the direction of the velocity is reversed, we just have to reverse
        the sign of $v$:
        \begin{align*}
          t_B &= \gamma \, \left(t_A + \frac{-v}{\c^2} \, x_A\right)\\
          t_B &= \gamma \, \left(t_A - \frac{v}{\c^2} \, x_A\right)\\
          x_B &= \gamma \, \left(x_A + (-v) \, t_A\right)
          x_B &= \gamma \, \left(x_A - v \, t_A\right)
          \intertext{$\gamma$ also contains $v$, but:}
          \gamma &= \frac{1}{\sqrt{1 - \frac{v^2}{c^2}}} = \frac{1}{\sqrt{1 - \frac{{(-v)}^2}{c^2}}}
        \end{align*}

      \subsection{Quantitative analysis}
        $L_A$~-- Length of Bob's ship as measured by Alice
        \paragraph{Rearward light beam}
          When it hits the rear clock on Bob's ship, Alice
          reads ${(T_A)}_{rear}$ on her lattice of clocks.
          \begin{align*}
            \text{Distance covered} &= \frac{L_A}{2} - v \, {(T_A)}_{rear}\\
            \text{Elapsed time } {(T_A)}_{rear} &= \frac{\text{Distance covered}}{\c}\\
            {(T_A)}_{rear} &= \frac{\frac{L_A}{2} - v \, {(T_A)}_{rear}}{\c}\\
            \c \, {(T_A)}_{rear} &= \frac{L_A}{2} - v \, {(T_A)}_{rear}\\
            \c \, {(T_A)}_{rear} + v \, {(T_A)}_{rear} &= \frac{L_A}{2}\\
            {(T_A)}_{rear} (\c + v) &= \frac{L_A}{2}\\
            {(T_A)}_{rear} &= \left(\frac{L_A}{2}\right)\left(\frac{1}{\c + v}\right)
          \end{align*}

        \paragraph{Frontward light beam}
          When it hits the front clock on Bob's ship, Alice
          reads ${(T_A)}_{front}$ on her lattice of clocks. She has to wait
          longer for that than ${(T_A)}_{rear}$, because the light beam also
          has to cover the distance that Bob's ship has moved in the meantime.
          \begin{align*}
            \text{Distance covered} &= \frac{L_A}{2} + v \, {(T_A)}_{front} \\
            \text{Elapsed time } {(T_A)}_{front} &= \frac{\text{Distance covered}}{\c}\\
            {(T_A)}_{front} &= \frac{\frac{L_A}{2} + v \, {(T_A)}_{rear}}{\c}\\
            \c \, {(T_A)}_{front} &= \frac{L_A}{2} + v \, {(T_A)}_{front}\\
            \c \, {(T_A)}_{front} - v \, {(T_A)}_{front} &= \frac{L_A}{2}\\
            {(T_A)}_{front} (\c - v) &= \frac{L_A}{2}\\
            {(T_A)}_{front} &= \left(\frac{L_A}{2}\right)\left(\frac{1}{\c - v}\right)\\
            {(T_A)}_{front} &>^? {(T_A)}_{rear}\\
            \left(\frac{L_A}{2}\right)\left(\frac{1}{\c - v}\right) &>^? \left(\frac{L_A}{2}\right)\left(\frac{1}{\c + v}\right)\\
            \frac{1}{\c - v} &>^! \frac{1}{\c + v}. \text{ \emph{Leading clocks lag.}} \qed
          \end{align*}

        \paragraph{\emph{By how much time} do leading clocks lag?}
          \begin{align*}
            \mdelta{T_A} &= {(T_A)}_{front} - {(T_A)}_{rear}\\
              &= \left(\frac{L_A}{2}\right)\left(\frac{1}{\c - v}\right)
                - \left(\frac{L_A}{2}\right)\left(\frac{1}{\c + v}\right)\\
              &= \left(\frac{L_A}{2}\right)
                 \left(\frac{1}{\c - v} - \frac{1}{\c + v}\right)\\
              &= \left(\frac{L_A}{2}\right)
                 \left(\frac{\cancel{\c} + v - (\cancel{\c} - v)}{(\c - v)(\c + v)}\right)\\
              &= \left(\frac{L_A}{\cancel{2}}\right)
                 \left(\frac{\cancel{2} \, v}{\c^2 - v^2}\right)\\
              &= L_A \,
                 \left(\frac{v}{\c^2 \left(1 - \frac{v^2}{\c^2}\right)}\right)\\
              &= \left(\frac{L_A \, v}{\c^2}\right)
                 \left(\frac{1}{1 - \frac{v^2}{\c^2}}\right)\\
            \Aboxed{\mdelta{T_A} &= \left(\frac{L_A \, v}{\c^2}\right)
              \, {\gamma}^2}
            \intertext{Alice sees Bob's ship length-contracted:}
            L_A &= \frac{L_B}{\gamma}\\
            \mdelta{T_A} &= \left(\frac{L_B \, v}{\gamma \, \c^2}\right)
              \, {\gamma}^2\\
            \mdelta{T_A} &= \left(\frac{L_B \, v}{\c^2}\right)
              \, \gamma
            \intertext{$\mdelta{T_A}$ is the time difference that Alice sees on her clocks.
              She sees time on Bob's clocks time-dilated:}
            \mdelta{T_B} &= \frac{\mdelta{T_A}}{\gamma}\\
            \Aboxed{\mdelta{T_B} &= \frac{L_B \, v}{\c^2}
              \qquad \text{Time that Bob's front clock lags behind his rear clock}}
            \intertext{This can be generalized to}
            \Aboxed{\mdelta{T_B} &= \frac{D \, v}{\c^2}
              \qquad \text{Time that the front clock lags behind the rear clock}}
            \intertext{where $D$ is the distance between any two clocks in Bob's frame of reference.}
          \end{align*}

      \subsection{Using the Lorentz transformation}
        \begin{samepage}
          Proper length of Bob's ship: L\\
          Bob's front clock: B1\\
          Bob's rear clock: B2\\
          Alice's clock aligned with Bob's front clock: A1\\
          Alice's clock aligned with Bob's rear clock: A2
          \begin{align*}
            t_A &= \gamma \, \left(t_B + \frac{v}{\c^2} \, x_B\right)\\
            t_B &= \gamma \, \left(t_A - \frac{v}{\c^2} \, x_A\right)
            \intertext{Assume}
            t_{A1} &= t_{A2} = 0 \qquad \text{(Alice's clocks are synchronized)}\\
            t_{B1} &= 0\\
            x_{A1} &= 0\\
            x_{A2} &= x_{A1} - \frac{L}{\gamma} = -\,\frac{L}{\gamma} \text{ due to length contraction,}
            \intertext{then:}
            t_{B2} &= \gamma \, \left(t_{A2} - \frac{v}{\c^2} \, x_{A2}\right)\\
              &= \gamma \, \left(0 - \frac{v}{\c^2} \, \left(-\,\frac{L}{\gamma}\right)\right)\\
              &= \gamma \, \left(\frac{L \, v}{\gamma \, \c^2}\right)\\
            t_{B2} &= \frac{L \, v}{\c^2} \qed
          \end{align*}
        \end{samepage}

    \section{Leading clocks lag, revisited}
      \paragraph{“Star Tours” set-up:}

      \paragraph{Analysis done from the perspective of the observer on Earth (the Earth-Star reference frame)}
        \begin{align*}
          \text{Time for Bob to reach star } &= \frac{\SI{5}{\lightyear}}{\SI{0.943}{\c}}
            = \SI{5.30}{\year}
        \end{align*}
      \paragraph{Analysis done from the perspective of the observer on the rocket (the rocket reference frame)}
        \begin{align*}
          \text{Time for star to reach Bob } &= \frac{\SI{1.67}{\lightyear}}{\SI{0.943}{\c}}
            = \SI{1.77}{\year}\\
          \text{Bob's observation of elapsed time on Alice's clock}
            &= \frac{\SI{1.77}{\year}}{\gamma}
            = \SI{0.59}{\year} \text{ (???)}
        \end{align*}
      \paragraph{The answer to the puzzle (via “leading clocks lag”)}
        In addition to time dilation and length contraction, one also has to
        consider the \emph{relativity of simultaneity: Alice's clocks are not
        synchronized in Bob's frame of reference.}

        In Bob's frame of reference, Alice's clocks are moving towards him.
        To him, her clock on Earth, which shows $t_{A2} = 0$, is the
        leading clock, which lags the one at the star.
        \begin{align*}
          \text{Time by which Earth clock lags star clock}
            &= \frac{D \, v}{\c^2}\\
            &= \frac{\SI{5}{\lightyear} \times \SI{0.943}{\c}}{\c^2}\\
            &= \frac{\SI{5}{\lightyear} \times 0.943}{\c} \qquad \c = \SI{1}{\frac{\lightyear}{\year}}\\
            &= \SI{5}{\lightyear} \times \SI{0.943}{\year\per\lightyear}\\
            &= \SI{4.715}{\year}
        \end{align*}
        IOW: \emph{The clock at the star}, from Bob's perspective, \emph{is
        \SI{4.715}{\year} ahead of the Earth clock.}  So when Bob reaches it after
        \SI{1.77}{\year}, it correctly shows \SI{5.3}{\year}.

    \section{The ultimate speed limit}
      \paragraph{Reminder regarding combining velocities in the pre-relativity world}
        In the Galilean transformation, velocities just add: For example, if
        a ball is shot from a moving car, the velocity of the ball is
        the velocity of the car as observed from the ground plus the velocity
        of the ball as observed from the car.
      \paragraph{The “\mdelta{}” form of the Lorentz transformation}
        Situation: Alice (in the \strong{L}ab frame) observes Bob in his
        \strong{R}ocket moving at speed $v$, from which an escape pod is
        ejected whose speed is $u_R$ as measured from the rocket.
    \section{Combining velocities}
        \begin{align*}
          \text{Lorentz transformation:}\\
          x_L &= \gamma \, \left(x_R + v \, t_R\right)\\
          t_L &= \gamma \, \left(t_R + \frac{v}{\c^2} \, x_R\right)\\
          \\
          \mdelta{x_L}
            &= {(x_2)}_L - {(x_1)}_L\\
            &= \gamma \, \left({(x_2)}_R + v \, {(t_2)}_R\right)
              -\gamma \, \left({(x_1)}_R + v \, {(t_1)}_R\right)\\
            &= \gamma \, {(x_2)}_R - \gamma \, {(x_1)}_R
              + \gamma \, v \, {(t_2)}_R - \gamma \, v \, {(t_1)}_R\\
            &= \gamma \, \left({(x_2)}_R - {(x_1)}_R\right)
              + \gamma \, v \, \left({(t_2)}_R - {(t_1)}_R\right)\\
            &= \gamma \, \mdelta{x_R} + \gamma \, v \, \mdelta{t_R}\\
          \mdelta{x_L}
            &= \gamma \, \left(\mdelta{x_R} + v \, \mdelta{t_R}\right)
          \intertext{Similarly,}
          \mdelta{t_L}
            &= \gamma \, \left(\mdelta{t_R} + \frac{v}{\c^2} \mdelta{x_R}\right).
        \end{align*}

      \paragraph{Reminder regarding the definition of velocity}
        \begin{align*}
          \text{Velocity}
            &= \frac{\text{Distance}}{\text{Time}}
            = \frac{\mdelta{x}}{\mdelta{t}}\\
        \end{align*}

      \paragraph{Using the Lorentz transformation to find how to combine velocities according to the theory of relativity}
        \begin{align*}
          u_R &= \frac{\mdelta{x_R}}{\mdelta{t_R}}\\
          u_L
            &= \frac{\mdelta{x_L}}{\mdelta{t_L}}\\
            &= \frac{\cancel{\gamma} \, \left(\mdelta{x_R} + v \, \mdelta{t_R}\right)}
                    {\cancel{\gamma} \, \left(\mdelta{t_R} + \frac{v}{\c^2} \mdelta{x_R}\right)}\\
            &= \frac{\mdelta{x_R} + v \, \mdelta{t_R}}
                    {\mdelta{t_R} + \frac{v}{\c^2} \mdelta{x_R}}\\
            &= \frac{\cancel{\mdelta{t_R}} \, \left(\frac{\mdelta{x_R}}{\mdelta{t_R}} + v\right)}
                    {\cancel{\mdelta{t_R}} \, \left(1 + \frac{v}{\c^2} \, \frac{\mdelta{x_R}}{\mdelta{t_R}}\right)}\\
          u_L &= \frac{u_R + v}{1 + \frac{v}{\c^2} \, u_R}\\
              &= \frac{u_R + v}{1 + \frac{v \, u_R}{\c^2}}
        \end{align*}

      \paragraph{Checking the result to see if it makes sense}
        \begin{samepage}
          \begin{align*}
            \intertext{If $v \ll \c$ and $u_R \ll \c$,}
            \frac{v \, u_R}{\c^2} &\approx 0\\
            1 + \frac{v \, u_R}{\c^2} &\approx 1\\
            u_L
              &= \frac{u_R + v}{1 + \frac{v \, u_R}{\c^2}}\\
              \approx u_R + v
            \intertext{\emph{This is the addition of velocities according
              to the Galilean transformation.}}
            \intertext{If $u_R = 0$ (the escape pod is still onboard Bob's ship),}
            u_L
              &= \frac{\cancelto{0}{u_R} + v}{1 + \cancelto{0}{\frac{v \, \cancel{u_R}}{\c^2}}}
              = v. \qed
            \intertext{If $v = 0$ (Bob is not moving relative to Alice),}
            u_L
              &= \frac{u_R \, + \cancelto{0}{v}}{1 + \cancelto{0}{\frac{\cancel{v} \, u_R}{\c^2}}}
              = u_R. \qed
            \intertext{If $v = \SI{0.9}{\c}$ and $u_R = \SI{0.7}{\c}$,}
            u_L
              &= \frac{\SI{0.7}{\c} + \SI{0.9}{\c}}{1 + \frac{\SI{0.9}{\c} \times \SI{0.7}{\c}}{\c^2}}\\
              &= \frac{\SI{1.6}{\c}}{1 + \frac{0.63 \, \cancel{\c^2}}{\cancel{\c^2}}}\\
              &= \frac{\SI{1.6}{\c}}{1.63} \approx \SI{0.98}{\c} < \c\text{(!)}
            \intertext{If $v = \SI{0.9}{\c}$ and $u_R = \c$ (Bob points a light beam in the direction of travel),}
            u_L
              &= \frac{\c + \SI{0.9}{\c}}{1 + \frac{\SI{0.9}{\c} \times \c}{\c^2}}\\
              &= \frac{\SI{1.9}{\c}}{1 + \frac{0.9 \, \cancel{\c^2}}{\cancel{\c^2}}}\\
              &= \frac{\SI{1.9}{\c}}{1.9} = \c. \qed
            \intertext{\emph{Both Bob and Alice see the light beam travel at \c.}}
          \end{align*}
        \end{samepage}

    \section{The ultimate speed limit}
      \paragraph{What the Lorentz factor equation implies about an ultimate speed limit}
        \begin{align*}
          \gamma &= \frac{1}{\sqrt{1 - \frac{v^2}{\c^2}}}
        \end{align*}
        As $v \rightarrow~\c$, $\gamma \rightarrow~\infty$, so nothing but light
        can go at the speed of light.

        If $v > \c$, $\gamma = \frac{1}{\sqrt{a}}$ with $a < 0$, so nothing
        can go faster than the speed of light.

      \paragraph{What happens to a light clock traveling at the speed of light (as observed by a stationary observer)}
        Elapsed time on a moving clock is less than the elapsed time on an
        identical stationary clock (``time dilation''):
        \begin{align*}
          {(\mdelta{T})}_{Lab} &= \gamma \, {(\mdelta{T})}_{Rocket}\\
          \text{or } {(\mdelta{T})}_{Rocket} &= \frac{1}{\gamma} \, {(\mdelta{T})}_{Lab}
        \end{align*}
        As $\gamma \rightarrow~\infty$, ${(\mdelta{T})}_{Lab} \rightarrow~\infty$
        and ${(\mdelta{T})}_{Rocket} \rightarrow~0$.

        For a light clock moving at \c{} this would mean that it would not tick at all.
        Physically it would mean that the top mirror of the clock would have moved
        away before the light beam can hit it.

    \clearpage
    \section{Perpendicular Velocities}
      \paragraph{The case of the escape pod with a velocity perpendicular to
        the spaceship’s direction of motion}
        \begin{align*}
          \text{Lorentz transformation:}\\
          x_L &= \gamma \, \left(x_R + v \, t_R\right)\\
          t_L &= \gamma \, \left(t_R + \frac{v}{\c^2} \, x_R\right)\\
          y_L &= y_R, \, z_L = z_R\\
          {(u_L)}_x &= \frac{{(u_R)}_x + v}{1 + \frac{v \, {(u_R)}_x}{\c^2}}\\
          \\
          {(u_R)}_x &= \frac{\mdelta{x_R}}{\mdelta{t_R}}\\
          {(u_R)}_y &= \frac{\mdelta{y_R}}{\mdelta{t_R}}\\
          {(u_L)}_y
            &= \frac{\mdelta{y_L}}{\mdelta{t_L}}
            = \frac{\mdelta{y_R}}{\mdelta{t_L}}\\
            &= \frac{\mdelta{y_R}}{\gamma \, \left(\mdelta{t_R} + \frac{v}{\c^2} \, \mdelta{x_R}\right)}\\
            &= \frac{\mdelta{y_R}}{\gamma \, \mdelta{t_R} \, \left(1 + \frac{v}{\c^2} \, \frac{\mdelta{x_R}}{\mdelta{t_R}}\right)}\\
            &= \frac{\mdelta{y_R}}{\gamma \, \mdelta{t_R} \, \left(1 + \frac{v}{\c^2} \, {(u_R)}_x\right)}
          \intertext{For Bob, there is no velocity of the pod in the $x$-direction:}
          \mdelta{x_R} &= 0 \rightarrow~{(u_R)}_x = 0,
          \intertext{therefore:}
          {(u_L)}_y &= \frac{\mdelta{y_R}}{\gamma \, \mdelta{t_R} \, \left(1 + \cancelto{0}{\frac{v}{\c^2} \, \cancel{{(u_R)}_x}}\right)}\\
            &= \frac{\mdelta{y_R}}{\gamma \, \mdelta{t_R}}\\
          {(u_L)}_y &= \frac{{(u_R)}_y}{\gamma}
          \intertext{To show this, consider again}
          {(u_L)}_x &= \frac{\cancelto{0}{{(u_R)}_x} + v}{1 + \frac{v \, {(u_R)}_x}{\c^2}}\\
            &= \frac{v}{1 + \cancelto{0}{\frac{v \, \cancel{{(u_R)}_x}}{\c^2}}}\\
            &= v\\
          {(u_L)}_y &= \frac{{(u_R)}_y}{\gamma} \rightarrow~0 \text{ as } v \rightarrow~\c (\gamma \rightarrow~\infty)
          \longintertext{\emph{This describes the case of a light clock whose speed approaches
            and reaches the speed of light, which for a stationary observer
            would tick ever slower and eventually stop when it moves at the
            speed of light.}}
        \end{align*}

  \chapter{Week 6: Breaking the Spacetime Limit?}
    \section{Spacetime diagrams, revisited}
      \subsection{Part 1}
        \paragraph{Review of spacetime diagrams}
        \paragraph{Lines of simultaneity and lines of same location}
        \paragraph{Plotting a light beam}
          If you measure the speed of light in light-years per year,
          you can measure distance ($x$) in light-years and time ($t$) in years,
          so that the worldline of a beam of light becomes a straight
          line at \SI{45}{\degree} (slope: 1).

          \emph{This is equivalent to plotting the vertical axis as $\c{}\,t$.}

      \subsection{Part 2}
        ``Cool'' plots (``Don't be a square'' ;-))
        \paragraph{“Normal” plot (90 degree angles)}
        \paragraph{Plot with horizontal x axis and y axis at a non-90-degree angle}
        \paragraph{Plot with both axes at non-90-degree angles}

      \subsection{Part 3}
        \paragraph{Spacetime diagrams for Alice and Bob (separate)}
        \paragraph{Plotting Bob’s time axis ($x_B = 0$) and lines of
          same location on Alice’s diagram}
          Bob’s time axis is where his worldline is in Alice’s diagram
          \begin{align*}
            x_B &= 0\\
            \gamma \, \left(x_A - v \, t_A\right) &= x_B\\
            \gamma \, \left(x_A - v \, t_A\right) &= 0\\
            \gamma  &\neq 0\\
            x_A - v \, t_A &= 0\\
            x_A &= v \, t_A\\
            t_A &= \frac{1}{v} \, x_A \qquad \text{Equation for Bob's worldline in Alice's diagram}\\
            \intertext{Other lines of same location in Bob's frame of reference in Alices diagram:}
            x_B &= 1\\
            \gamma \, \left(x_A - v \, t_A\right) &= 1\\
            x_A - v \, t_A &= \frac{1}{\gamma}\\
            v \, t_A &= x_A - \frac{1}{\gamma}\\
            t_A &= \frac{1}{v} \, x_A - \frac{1}{\gamma \, v}
          \end{align*}

        \paragraph{Plotting Bob’s x axis ($t_B = 0$) and lines of simultaneity on Alice’s diagram}
          \begin{align*}
            t_B &= 0\\
            \gamma \, \left(t_A - \frac{v}{\c^2} \, x_A\right) &= t_B\\
            \gamma \, \left(t_A - \frac{v}{\c^2} \, x_A\right) &= 0\\
            \gamma  &\neq 0\\
            t_A - \frac{v}{\c^2} \, x_A &= 0\\
            t_A &= \frac{v}{\c^2} \, x_A \qquad \text{Equation for Bob's x-axis in Alice's diagram}
            \intertext{Other lines of simultaneity in Bob's frame of reference in Alice's diagram:}
            t_B &= 1\\
            \gamma \, \left(t_A - \frac{v}{\c^2} \, x_A\right) &= 1\\
            t_A - \frac{v}{\c^2} \, x_A &= \frac{1}{\gamma}\\
            t_A &= \frac{v}{\c^2} \, x_A + \frac{1}{\gamma}
          \end{align*}

        \emph{Bob's axes are skewed in Alice's diagram, so the same event has
          different coordinates in each frame of reference, and Alice and Bob
          do not agree on locations and times.}

      \subsection{Part 4}
        \begin{samepage}
          \begin{align*}
           v &= \SI{0.5}{\c} \rightarrow~\gamma = 1.2, \, \c = 1
          \end{align*}

          \paragraph{Completing the combined spacetime diagrams with scale markings for Bob’s axes}
            \begin{align*}
              x_A &= \gamma \, \left(x_B + v \, t_B\right)\\
              t_A &= \gamma \, \left(t_B + \frac{v}{\c^2} x_B\right)\\
              \\
              x_B &= 0, \, t_B = 1\\
              x_A &= \gamma \, v = 1.2 \times 0.5 = 0.6\\
              t_A &= \gamma = 1.2\\
              (x_B &= 0, \, t_B = 1) \rightarrow~(x_A = 0.6, \, t_A = 1.2)\\
              (x_B &= 0, \, t_B = 1) \rightarrow~(x_A = 1.2, \, t_A = 2.4)\\
              (x_B &= 1, \, t_B = 0) \rightarrow~(x_A = 1.2, \, t_A = 0.6)\\
              (x_B &= 2, \, t_B = 0) \rightarrow~(x_A = 2.4, \, t_A = 1.2)
            \end{align*}

          \paragraph{Verifying time dilation and length contraction using the combined diagram}
        \end{samepage}

      \subsection{Part 5}
        \paragraph{Verifying the relativity of simultaneity and
          “leading clocks lag” using the combined spacetime diagram
          (series of flashes that are simultaneous in Alice’s frame of reference
          but not in Bob’s)}

    \section{Regions of spacetime}
      \paragraph{Combined spacetime diagrams for Bob’s plot on Alice’s regular (90‐degree angles) plot, and vice versa.}

      \paragraph{The light cone}
        \begin{align*}
          \c^2 \, t^2 - x^2 = \text{const.} \qquad \text{Invariant interval, see~\ref{sec:invariant-interval}}
        \end{align*}

      \paragraph{Timelike interval}
        ``Enough time to make the distance''
        \begin{align*}
          \c^2 \, t^2 - x^2 &> 0\\
          \c^2 \, t^2 &> x^2\\
          \c \, t &> x
        \end{align*}
        The distance to be traveled is less than light could travel
        in the same time, so one can get there if those events are
        in the observer's future; also, events from there,
        if in the observers past, can affect the observer.

      \paragraph{Lightlike interval}
        \begin{align*}
          \c^2 \, t^2 - x^2 &= 0\\
          \c^2 \, t^2 &= x^2\\
          \c \, t &= x
        \end{align*}
        The distance to be traveled is equal to that light could travel
        in the same time, so only something traveling at the speed of light
        can get there if it in in the observer's future; or, if it is in
        the observer's past, can affect the observer.

      \paragraph{Spacelike interval}
        \begin{samepage}
          ``Not enough time to make the distance''
          \begin{align*}
            \c^2 \, t^2 - x^2 &< 0\\
            \c^2 \, t^2 &< x^2\\
            \c \, t &< x
          \end{align*}
          The distance to be traveled is greater than light could travel
          in the same time, so only something traveling faster than
          the speed of light can get there.  As nothing can go faster than
          the speed of light, nothing can get there in time nor, if those
          events are in the past, can they affect the observer.
        \end{samepage}

    \section{Faster than light?}
        \emph{Note: This example is an edited version of a problem
          from~\citetitle[pp.~89--90]{taylor1992spacetime},
          by~\cite{taylor1992spacetime}.}
        \paragraph{Rocket trip from San~Francisco to St.~Louis to New~York City}
        \paragraph{Diagram}
        \paragraph{Actual distance between where each flash was created
          (the distance the rocket covered between San~Francisco and St.~Louis):}
          \begin{samepage}
            \begin{align*}
              \text{Distance San~Francisco---St.~Louis}
              &= v \, \left(t_2 - t_1\right)\\
              &= v \, \mdelta{t}
            \end{align*}
        \end{samepage}

        \paragraph{Distance between flash~\#1 (taken at San~Francisco) and
          flash~\#2 (taken at St.~Louis) when the flashes are traveling
          toward New~York, one flash behind the other:}
          \begin{samepage}
            \begin{align*}
              \text{Distance between the two flashes}
                &= \text{Distance light has traveled to St.~Louis}\\
                  &- \text{Distance spaceship has traveled to St.~Louis}\\
                &= \c \, \mdelta{t} - v \, \mdelta{t}\\
                &= (\c - v) \, \mdelta{t}
            \end{align*}
          \end{samepage}

        \paragraph{Time between when flash~\#1 is seen by the New~York observer
          and when flash~\#2 is seen:}
          \begin{align*}
            \text{Time difference between the two flashes}
              &= \frac{(\c - v) \, \mdelta{t}}{\c}
          \end{align*}

        \paragraph{The apparent speed of the rocket as seen by the observer
          in New~York:}
          \begin{samepage}
            \begin{align*}
              \text{Apparent speed of spaceship}
                &= \frac{\text{Distance San~Francisco---St.~Louis}}
                        {\text{Time difference between the two flashes}}\\
                &= \frac{v \, \cancel{\mdelta{t}}}
                        {\frac{(\c - v) \, \cancel{\mdelta{t}}}{\c}}\\
                &= \frac{\c \, v}{\c - v}\\
                &= \frac{v}{1 - v} \qquad \text{($\c := 1$)}\\
                &= \frac{\SI{0.8}{\c}}{1 - \SI{0.8}{\c}} \qquad \text{$v := \SI{0.8}{\c}$}\\
                &= \frac{\SI{0.8}{\c}}{\SI{0.2}{\c}}\\
                &= 4 \, \text{(\c!)}
            \end{align*}
          \end{samepage}

    \section{Cause and effect, or vice versa?}
    \emph{Note: This example is a modified version of a problem
      from~\citetitle[pp.~108--109]{taylor1992spacetime},
      by~\cite{taylor1992spacetime}.}

      \paragraph{The situation:}
        \begin{enumerate}
          \item Year 0: Peace treaty signed, good guys' negotiation team
            leaves bad guys' planet at $v = \SI{0.6}{\c}$.
          \item Year 4: Bad guys invent faster-than-light ship and take off
            in pursuit at $v = \SI{3}{\c}$.
          \item Year 5: Bad guys catch up and launch sneak attack.
        \end{enumerate}
      \paragraph{Diagram}
      \paragraph{Spacetime location of the sneak attack in the frame of reference
        of the bad guys’ planet:}
        $\left({\left(x_{attack}\right)}_B = 4, \, {\left(t_{attack}\right)}_B = 5\right)$

      \paragraph{Spacetime location of the sneak attack in the frame of reference
        of the good guys’ spaceship:}
        \begin{align*}
          v &= \SI{0.6}{\c} \rightarrow~\gamma = 1.25\\
          {\left(x_{attack}\right)}_G
            &= \gamma \, \left({\left(x_{attack}\right)}_B - v \, {\left(t_{attack}\right)}_B\right)\\
            &= 1.25 \times \left(3 - 0.6 \times 5\right)\\
            &= 0\\
          {\left(t_{attack}\right)}_G
            &= \gamma \, \left({\left(t_{attack}\right)}_B - \frac{v}{\c^2} \, {\left(x_{attack}\right)}_B\right)\\
            &= 1.25 \times \left(5 - 0.6 \times 4\right)\\
            &= 4\\
          \rightarrow~\left({\left(x_{attack}\right)}_B = 4, \, {\left(t_{attack}\right)}_B = 5\right)
            &\rightarrow~\left({\left(x_{attack}\right)}_G = 0, \, {\left(x_{attack}\right)}_G = 4\right)
        \end{align*}

      \paragraph{Time of the launch of the bad guys’ faster‐than‐light spaceship
        according to the frame of reference of the good guys’ spaceship:}
        \begin{align*}
          \left({\left(x_{launch}\right)}_B = 0, \, {\left(t_{launch}\right)}_B = 4\right)\\
          {\left(x_{launch}\right)}_G
            &= \gamma \, \left({\left(x_{launch}\right)}_B - v \, {\left(t_{launch}\right)}_B\right)\\
            &= 1.25 \times \left(0 - 0.6 \times 4\right)\\
            &= -3\\
          {\left(t_{launch}\right)}_G
            &= \gamma \, \left({\left(t_{launch}\right)}_B - \frac{v}{\c^2} \, {\left(x_{launch}\right)}_B\right)\\
            &= 1.25 \times \left(4 - 0.6 \times 0\right)\\
            &= 5\\
          \rightarrow~\left({\left(x_{launch}\right)}_B = 0, \, {\left(t_{launch}\right)}_B = 4\right)
            &\rightarrow~\left({\left(x_{launch}\right)}_G = -3, \, {\left(t_{launch}\right)}_G = 5\right)\\
          {(t_{attack})}_G &< {(t_{launch})}_G \, \text(!)
        \end{align*}

        In the frame of reference of the good guys, the bad guys attack
        the good guys \underline{one year before} the former invented and
        launched their FTL starship for the attack.  This is a
        \emph{violation of causality} that strongly indicates that
        \emph{nothing can go faster than the speed of light}.

    \section{Summary}

  \chapter{Week 7: Paradoxes to Ponder}
    \section{Cause and effect: spacetime diagram}
      The diagram (note the significance of the fact that the world line of
      the faster-than-light spaceship has a slope that is less than
      the slopes of the lines of simultaneity for the good guys’
      frame of reference):

      Worldlines for movements faster than the speed of light cross lines
      of simultaneity in the other frame of reference in the backwards
      direction.
    \section{The pole in the barn paradox}
      \paragraph{The situation and the paradox}
        Alice moves with her pole into Bob's barn at $v = \SI{0.6}{\c}$, so
        $\gamma(v) = 1.25$.  There are synchronized clocks on each end of
        Bob's barn: \#1 at the front, \#2 at the back.  The front of the pole
        enters the barn at $t_A = t_B = 0$.

      \paragraph{Bob’s (barn) perspective and analysis}
        \begin{align*}
          {(L_{barn})}_B &= \SI{8}{\meter}\\
          {(L_{pole})}_B &= \frac{1}{\gamma} {(L_{pole})}_A = \SI{8}{\meter}
        \end{align*}
        Conclusion: The pole will fit into the barn.

        Analysis:
        Time it takes the pole to reach back door and back of pole to reach front door:
        \begin{align*}
          \frac{\SI{8}{\meter}}{\SI{0.6}{c} \frac{\text{m}}{\text{s}}}
          &= \SI{44.4}{\nano\second} \qquad \text{(two photos taken)}
        \end{align*}

      \paragraph{Alice’s (pole) perspective and analysis}
        \begin{align*}
          {(L_{barn})}_A &= \frac{1}{\gamma} {(L_{barn})}_B = \SI{6.4}{\meter}\\
          {(L_{pole})}_A &= \SI{10}{\meter}
        \end{align*}
        Conclusion: The pole will \underline{not} fit into the barn.

        Analysis:
        \begin{enumerate}
          \item
            The back door takes
            $\frac{\SI{6.4}{\meter}}{\SI{0.6}{\c}} = \SI{35.6}{\nano\second}$
            to reach the front of the pole.

            During this time, Alice observes Bob's clock to tick off
            $\frac{\SI{35.6}{\nano\second}}{\gamma} = \SI{28.4}{\nano\second}$.

            Also, at $t_A = 0$, Bob's clock \#2 was reading ahead
            $\frac{D \, v}{\c} = \frac{\SI{8}{\meter} \times \SI{0.6}{\c}}{\c}
            = \SI{16.0}{\nano\second}$ (because clock \#1 ``lagged'').

            So the when the back door reaches the pole, she observes Bob's
            clock \#2 at $\SI{16.0}{\nano\second} + \SI{28.4}{\nano\second}
            = \SI{44.4}{\nano\second}$ (\emph{as does he}).

          \item
            Alice sees the barn to be only \SI{6.4}{\meter} long due to length
            contraction.  So when the back of the barn has reached the front
            of the pole, she would have to think that the pole would stick out
            the front door by
            $\SI{10}{\meter} - \SI{6.4}{\meter} = \SI{3.6}{\meter}$.
        \end{enumerate}

      \paragraph{Resolving the paradox}
        ``Leading clocks lag'': For Alice, when clock \#2 shows
        \SI{44.4}{\nano\second}, clock \#1 (at the front door of the barn) only
        shows \SI{28.4}{\nano\second}.  It has to tick off further
        \SI{16}{\nano\second} to reach the \SI{44.4}{\nano\second} as observed
        on the photos (\emph{to which both Alice and Bob have to agree}).

        But due to time dilation, this happens within
        $\gamma \times \SI{16}{\nano\second} = \SI{20}{\nano\second}$
        for Alice.  And during that time, the barn covers the remaining
        $\SI{20}{\nano\second} \times \SI{0.6}{\c} = \SI{3.6}{\meter}$. (!)

      \paragraph{The combined spacetime diagram}

    \section{How objects contract, and spaceships on a rope}
      \subsection{How objects contract}
        \paragraph{The situation (Bob on a metal rod moving at velocity $v$,
          Alice observing)}
          Metal rod of length $L_B$ (measured in Bob's frame).
          Alice observes it to have length $L_A = \frac{1}{\gamma} \, L_B$.
          Let the rod be accelerated slightly to new velocity $v + \mdelta{v}$.
          How does that occur?
        \paragraph{What has to happen for all parts of the rod to accelerate
          at the same time (according to Bob)}
          Consider rod sliced in several small parts, with two clocks
          synchronized in Bob's frame, attached to the rear part and
          the front part.

          If Bob can push all parts of the rod by the same amount at
          the same time then the rod will remain intact but travel
          faster/slower.
        \paragraph{Alice’s observation of Bob’s supposedly simultaneous
          acceleration of the parts of the rod (the key role of
          the relativity of simultaneity)}
          When Bob's clocks show the same time to Bob, they do not for Alice.
          To her, the leading clock lags, i.e.\ the front part will receive
          the push some time later than the rear part.  This can be explained
          as the rod contracting in her frame of reference.

        \emph{Note: This is just one possible explanation.  If and how
          length contraction physically occurs is still being debated
          by physicists.}

      \subsection{Spaceships on a rope}
        Paradox developed independently by several people in the 1950s.
        One of them was J.S. Bell (Bell’s inequality) at CERN

        Book recommendation: ``Speakable and Unspeakable in Quantum Mechanics''
        \paragraph{The situation (spaceships B and C connected by a rope,
          with observer A in the middle between them)}
          On signal from A, spaceships B and C accelerate
          \underline{at the same time} such that the distance $D$ between them
          remains constant to A.

        \paragraph{What has to happen for the distance D to remain constant
          with respect to observer~A?}
          \emph{Statements like ``at the same time'' have to be thought
            about with regard to the relativity of simultaneity.}

          The leading and the trailing spaceship have to accelerate at the
          same time.

          But: As soon as B and C are moving relative to A, they are in a different
          reference frame than A.  For A, if B and C have synchronized clocks,
          the clock in the leading spaceship C always lags the one in the trailing
          spaceship B.  E.g., when clock~B reads \SI{10}{\second}, clock~C reads
          \SI{7}{\second}.

          Then, if B and C are to accelerate at \SI{10}{\second}, B has already
          started accelerating while C is not at \SI{10}{\second} yet.  But then
          B would get closer to C which violates the precondition.

        \paragraph{What happens from the perspective of the reference frame
          of the spaceships (B and C)?}
          The only way that the distance can remain constant for A, is if
          B starts at \SI{10}{\second} and C at \SI{7}{\second}.

        \paragraph{Will the rope break?}
          Yes.

        \paragraph{Observer A’s perspective and answer}
          The distance between B and C remains constant, but the rope is
          length-contracted:
          $L_A = \frac{1}{\gamma} \, L_{BC} = \frac{1}{\gamma} \, D$.

        \paragraph{Spaceships B and C perspective and answer}
          As C has to start before B, the rope is pulled apart by C's forward
          motion.

    \section{The twin paradox}
      \subsection{Part 1}
        \paragraph{The situation and diagram}
          Alice goes at \SI{0.6}{\c} ($\gamma = 1.25$) to a star
          \SI{3}{\lightyear} away from Bob's home planet, and returns
          at the same speed.

        \paragraph{Bob’s time dilated observation of Alice’s clock}
          \begin{align*}
            \mdelta{t_A} &= \frac{1}{\gamma} \, \mdelta{t_B}
              = \frac{1}{1.25} \times 5
              = \SI{4}{\year}
          \end{align*}
          Bob's conclusion: When Alice returns, she should be \emph{younger} than Bob.

        \paragraph{Alice’s time dilated observation of Bob’s clock}
          \begin{align*}
            \mdelta{t_B} &= \frac{1}{\gamma} \, \mdelta{t_A}
              = \frac{1}{1.25} \times 4
              = \SI{3.2}{\year}
          \end{align*}
          Alice's conclusion: When Alice returns, \underline{Bob} should be
            younger than her, she should be \underline{older} than him.

      \subsection{Part 2}
        \paragraph{The outbound diagram}
        \paragraph{The inbound diagram}
        \paragraph{Where the extra/missing time for Bob (from Alice’s perspective) comes from}
          When Alice turns around, she changes her frame of reference, and
          Bob's clock have to tick from 3.2 to 6.8 years.

      \subsection{Part 3}
        \paragraph{Bob’s analysis}
          \begin{align*}
            \intertext{Alice travels}
            \frac{\SI{3}{\lightyear}}{\SI{0.6}{\c}} &= \SI{5}{\year}
            \intertext{to the star, while he observes Alice's clock to tick}
            \frac{1}{\gamma} \times \SI{5}{\year} &= \SI{4}{\year}.
          \end{align*}

        \paragraph{Alice’s analysis (including the “leading clocks lag” effect)}
          \begin{align*}
            \intertext{To Alice, the distance to the star is length-contracted:}
            \text{Distance to star} &= \frac{\SI{3}{\lightyear}}{\gamma}
              = \SI{2.4}{\lightyear}\\
            \frac{\SI{2.4}{\lightyear}}{\SI{0.6}{\c}} &= \SI{4}{\year}.
            \intertext{During her journey to the star, she observes Bob's~clocks
              to tick}
            \frac{\SI{4}{\year}}{\gamma} &= \SI{3.2}{\year}.
            \longintertext{But as she moves towards the star, Bob's~clock
            at the star is ahead because the leading clock at his planet,
            lags by}
            \frac{D \, v}{\c^2}
              &= \frac{\SI{3}{\lightyear} \times \SI{0.6}{\c}}{\c^2}
              = \SI{1.8}{\year}
            \intertext{and she observes Bob's star clock to read}
            \SI{1.8}{\year} + \SI{3.2}{\year} &= \SI{5}{\year}
            \intertext{when she arrives (as does Bob).}
            \longintertext{When she turns around, now \emph{the star~clock} is
              the leading one, and \emph{lags} by the same amount.}
            \longintertext{Therefore, when the star clock shows \SI{5}{\year},
              the clock on Bob's planet reads}
            \SI{5}{\year} + \SI{1.8}{\year} &= \SI{6.8}{\year}.
            \longintertext{After ticking \SI{3.2}{\year} more while Alice is
              returning, it shows \SI{10}{\year} (as she and Bob can observe).}
          \end{align*}

      \subsection{Part 4}
        \paragraph{Summary of previous analysis}
        \paragraph{Comment on asymmetrical role of acceleration}
        \paragraph{Using the Lorentz transformation to get the same results}
          \subparagraph{A. Calculating Alice’s results using Bob’s value}
            \begin{align*}
              \intertext{Outbound: $(x_L = 3, \, t_L = 5)$}
              x_R &= \gamma \, \left(x_L - v \, t_L\right) = 0\\
              t_R &= \gamma \, \left(t_L - \frac{v}{\c^2} \, x_L\right) = 4
              \intertext{Inbound: $(x_L = -3, \, t_L = 5)$ (negative, because origin has moved)}
              x_R &= \gamma \, \left(x_L + v \, t_L\right) = 0\\
              t_R &= \gamma \, \left(t_L + \frac{v}{\c^2} \, x_L\right) = 4
            \end{align*}
          \subparagraph{B. Calculating Bob’s results using Alice’s value}
            \begin{align*}
              \intertext{Outbound: $(x_R = 0, \, t_R = 4)$}
              x_L &= \gamma \, \left(x_R + v \, t_R\right) = 3\\
              t_L &= \gamma \, \left(t_R + \frac{v}{\c^2} \, x_R\right) = 5
              \intertext{Inbound: $(x_R = 0, \, t_R = 4)$}
              x_L &= \gamma \, \left(x_R - v \, t_R\right) = -3\\
              t_L &= \gamma \, \left(t_R - \frac{v}{\c^2} \, x_R\right) = 5
            \end{align*}

          \paragraph{Using the invariant interval to get the same results}
            \begin{align*}
              \c^2 \, {(\mdelta{t_L})}^2 - {(\mdelta{x_L})}^2
                &= \c^2 \, {(\mdelta{t_R})}^2 - {(\mdelta{x_R})}^2
              \intertext{Outbound:}
              \mdelta{x_L} &= 3\\
              \mdelta{t_L} &= 5\\
              \mdelta{x_R} &= 0\\
              \c &= 1\\
              \cancel{\c^2} \, {(\mdelta{t_L})}^2 - {(\mdelta{x_L})}^2
                &= \cancel{\c^2} \, {(\mdelta{t_R})}^2 - \cancel{{(\mdelta{x_R})}^2}\\
                &= 4
              \intertext{Inbound:}
              \mdelta{x_L} &= -3\\
              \mdelta{t_L} &= 5\\
              \mdelta{x_R} &= 0\\
              \c &= 1\\
              \cancel{\c^2} \, {(\mdelta{t_L})}^2 - {(\mdelta{x_L})}^2
                &= \cancel{\c^2} \, {(\mdelta{t_R})}^2 - \cancel{{(\mdelta{x_R})}^2}\\
                &= 4
            \end{align*}

        \paragraph{Experimental verifications of time dilation}
          \begin{itemize}
            \item Atmospheric muons
            \item Muons at rest vs. Muons traveling in particle accelerator ring:
              muons at rest decay faster
            \item 1971: Two atomic clocks on ground, two in air in opposite directions:
              Westbound clock prediction: $\mdelta{t} = 275 \pm 25 ns$
              Westbound clock result: $\mdelta{t} = 273 \pm 7 ns$

              Repeated in 1996:
              Prediction: $\mdelta{t} = 39.8 ns$
              Result: $\mdelta{t} = 39 \pm 2 ns$

              2010: NIST laboratory experiment
              Very accurate clock, slow speeds required
          \end{itemize}

    \section{summary}

  \chapter{To the Center of the Galaxy and Back}
    \section{Traveling the galaxy}
    \section{The famous equation}
    \section{Traveling the galaxy, part 2}
    \section{The happiest thought}
    \section{The bending of light}
    \section{Final comments}

  % References
  \begingroup
  %     \raggedright
      %\sloppy
    \printbibliography[heading=lit]
  \endgroup
\end{document}
