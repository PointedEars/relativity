% vim:set fileencoding=utf-8 tabstop=2 shiftwidth=2 softtabstop=2 expandtab:
%% Page layout
% Scientific report, European style (A4) (from KOMA-Script)
% BCOR: binding correction
% DIV=calc: Calculate page spread now (recalculated below)
% pagesize: Add page info to (PDF/PS) output
% parskip=half: Do not indent paragraphs, add one half line spacing instead
%   (FFHS requirement)
% Default font size: 10pt (FFHS requirement)
% BCOR=1cm too much?
\documentclass[pagesize,headsepline,10pt,parskip=half]{scrreprt}

% Line spacing should be 1.5 times the line height (FFHS requirement)
\usepackage{setspace}
\onehalfspacing{}

% float compatibility for KOMA-Script
\usepackage{scrhack}

% Use English orthography and hyphenation by default
\usepackage[latin,italian,english]{babel}

% Allow using non-ASCII characters verbatim; disable for LuaLaTeX
% \usepackage[utf8]{inputenc}

% Special characters
\usepackage{textcomp}

% Use fonts having non-ASCII characters; disable fontenc for LuaLaTeX
% \usepackage[T1]{fontenc}
\usepackage{fontspec}
\newfontfamily\DejaSans{DejaVu Sans}

% User-defined English hyphenation
\hyphenation{InfoWorld}

% Use language-specific quotes
\usepackage[autostyle,english=american]{csquotes}
\DeclareQuoteAlias{italian}{latin}

% Advanced Computer Modern fonts
\usepackage{lmodern}

% Allow strike-through with \sout, keep italic for \emph
\usepackage[normalem]{ulem}

% Use medieval numbers except in math mode
% \usepackage{hfoldsty}

% Use sans-serif font ('Arial') by default for headings and normal text;
% disable helvet for LuaLaTeX
% (FFHS requirement)
\renewcommand{\familydefault}{\sfdefault}
\usepackage{mathptmx}
%\usepackage[scaled=.90]{helvet}
\usepackage{courier}

% Improved typography, like hyphenation in words with non-ASCII characters,
% see also http://homepage.ruhr-uni-bochum.de/georg.verweyen/pakete.html
\usepackage[babel]{microtype}

% Number also \subsubsection, but not \paragraph and below
\setcounter{secnumdepth}{3}

% Page heading and footer
\usepackage{scrpage2}
\pagestyle{scrheadings}
\automark[section]{chapter}
% heading on the top inner margin only
\ohead[]{\headmark}
\chead[]{}
% page number on the bottom outer margin only
\ofoot[\pagemark]{\pagemark}
\cfoot[]{}

% Support for list of acronyms
\usepackage[footnote,nohyperlinks,withpage]{acronym}

% References: can use section names
\usepackage{nameref}

% References: generate hyperlinks
\usepackage[plainpages=false]{hyperref}

% References style (default: 'numerical')
\usepackage[backend=biber,
style=authoryear-ibid,
maxcitenames=1,
maxbibnames=3]{biblatex}
\defbibheading{lit}{\chapter*{References}\markboth{References}{References}}
% References: bibliography database
\addbibresource{main.bib}

% prints author names as small caps
\renewcommand{\mkbibnamefirst}[1]{\textsc{#1}}
\renewcommand{\mkbibnamelast}[1]{\textsc{#1}}
\renewcommand{\mkbibnameprefix}[1]{\textsc{#1}}
\renewcommand{\mkbibnameaffix}[1]{\textsc{#1}}

% References: use English ordinal numbers
\usepackage[super]{nth}

% Automatically use teletype for \url argument, use content verbatim
\usepackage{url}
\urlstyle{tt}

% Less vertical spacing between list items (in `compactitem' environment)
\usepackage{paralist}

% Multi-row table cells
\usepackage{multirow}

% Support for horizontal rules in tables (professional style)
\usepackage{booktabs}

% Footnotes in tables
\usepackage{threeparttable}

% Tables across pages (for results)
\usepackage{longtable}

% Word wrap in table columns (calculate p width)
\usepackage{calc}

% Automatic column stretching
% \usepackage{tabularx}

% Stretched tables across pages (for results); requires longtable
% tabularx
\usepackage{ltxtable}

% Improved table formatting
\usepackage{array}

% Support for including PDFs
\usepackage{pdfpages}

% Support for figures
\usepackage{graphicx}

\usepackage{amsthm}
\newtheorem{mydef}{Definition}

\usepackage[fleqn]{amsmath}
\newlength{\normalparindent}
\AtBeginDocument{\setlength{\normalparindent}{\parindent}}
\newcommand{\longintertext}[1]{%
  \intertext{%
    \parbox{\linewidth}{%
      \setlength{\parindent}{\normalparindent}
      \noindent#1%
    }%
  }%
}

% cancel terms in equations
\usepackage[makeroom]{cancel}

% for boxes around equations, with \Aboxed
\usepackage{mathtools}

\usepackage{esint}
\usepackage{amssymb}
% \usepackage{commath}
\allowdisplaybreaks{}

% degree symbol etc.
\usepackage{gensymb}

% dingbats
\usepackage{pifont}
\newcommand{\cmark}{\, \text{\ding{51}}}

% SI units
\usepackage{siunitx}
\sisetup{per-mode = fraction, math-micro = \text{µ}, text-micro = µ}
\DeclareSIUnit\year{a}
\DeclareSIUnit\lightyear{ly}

% Margin notes
\usepackage{marginnote}

% FSM graphs
%\usepackage{tikz}
%\usetikzlibrary{arrows,automata}

% Plots
\usepackage{pgfplots}
\pgfplotsset{compat=1.5}

% Captions
\usepackage{caption}

% Recalculate page spread based on the definitions above
\recalctypearea{}

%% User commands
% Formatting and languages
\newcommand{\strong}[1]{\textbf{#1}}
\newcommand{\code}[1]{\texttt{#1}}
\newcommand{\var}[1]{\textit{#1}}
\newcommand{\en}[1]{\foreignlanguage{english}{#1}}

% Abbreviations
\usepackage{xspace}
\newcommand{\ao}{\mbox{u.\,a.}\xspace}
\newcommand{\cf}{\mbox{vgl.}\xspace}
\newcommand{\ie}{\mbox{d.\,h.}\xspace}
\newcommand{\eg}{\mbox{e.g.}\xspace}
\newcommand{\ital}{\mbox{ital.}\xspace}

% Common terms
\newcommand{\sectionname}{Section}
\newcommand{\eq}{equation\xspace}

% Commands for common math expressions
\newcommand{\abs}[1]{\lvert#1\rvert}
\renewcommand\d[1]{\:\textrm{d}#1}
\newcommand*\diff{\mathop{}\!\mathrm{d}}
\newcommand*\mdelta[1]{\ensuremath{\mathrm{\Delta\,}#1}}
\renewcommand{\qedsymbol}{\ensuremath{\blacksquare}}

% Chemistry
\newcommand*\chem[1]{\ensuremath{\mathrm{#1}}}

% alignment in \cases
\makeatletter
\renewcommand{\env@cases}[1][@{}l@{\quad}l@{}]{%
 \let\@ifnextchar\new@ifnextchar{}
 \left\lbrace{}
 \def\arraystretch{1.2}%
 \array{#1}%
}
\makeatother

% asides
\usepackage{mdframed}
\newenvironment{aside}
{\begin{mdframed}[style=0,%
  leftline=false,rightline=false,leftmargin=2em,rightmargin=2em,%
  innerleftmargin=0pt,innerrightmargin=0pt,linewidth=0.75pt,%
  skipabove=7pt,skipbelow=7pt]\small}
{\end{mdframed}}

%% Shortcuts
%%% arrays (vectors, matrixes, tensors)
\newcommand{\vecb}[1]{\mathbf{#1}}
\newcommand{\parray}[2]{\left(\begin{array}{#1}#2\end{array}\right)}
%% Vector norm
\newcommand{\norm}[1]{\left\|{#1}\right\|}
%% Constants
\newcommand{\const}[1]{\mathrm{#1}}
% Speed of light
\renewcommand{\c}{\const{c}}

% Calculate numbers
\usepackage{fp}

% Use programming in commands
% \usepackage{etoolbox}

%% Relativistic calculations
% Define command #1 as Lorentz factor for a speed #2 in multiples of speed of light
\newcommand{\lorentz}[2]{\FPeval{#1}{1/root(2, 1 - #2^2)}}

% Speed of light
\FPeval{\sol}{299792458}

% Define command #1 as relative speed for a Lorentz factor #2
\newcommand{\lorentztospeed}[2]{\FPeval{#1}{sol * root(2, 1 - 1/(#2^2))}}

% Define command #1 as length contraction of length #2
% for a speed #3 in multiples of speed of light
\newcommand{\lencon}[3]{\FPeval{#1}{#2 * root(2, 1 - #3^2)}}

% Define command #1 as time dilation of time #2
% for a speed #3 in multiples of speed of light
\newcommand{\timedil}[3]{\FPeval{#1}{1/root(2, 1 - #3^2) * #2}}

\begin{document}
  % User-defined language-specific hyphenation
  %  \hyphenation{bezüg-lich einer Da-ten-bank-ope-ra-ti-onen
  %  effizienz-stei-gernd ECMA ECMAScript Firefox Google JavaScript JavaScript-Core
  %  Kom-pa-ti-bi-li-täts-matrix lauf-fähig lenny Linux MySQL proto-typ-ba-sier-ter
  %  robusteren SquirrelFish Wine}

  \begin{titlepage}
    \title{\href{https://www.coursera.org/learn/einstein-relativity/}{Understanding Einstein:\\The Special Theory of Relativity}}
    \subtitle{Own solutions for the problem sets from the Stanford University Online Course}
    \author{Larry Randles Lagerstrom, Instructor\\\href{http://PointedEars.de/}{Thomas Lahn}, Student}
    \maketitle
  \end{titlepage}

  \clearpage
  \pagenumbering{Roman}
  \begin{spacing}{1}
    % Print TOC
    \tableofcontents
    \thispagestyle{empty}
  \end{spacing}

  %   \clearpage
  %   \begin{spacing}{1}
  %     \chapter*{List of acronyms} \label{chapter:acronyms}
  %     \begin{acronym}[]
  %       \setlength{\itemsep}{-\parsep}
  %        \acro{RFC}{Request for Comments (Internet-Standard)}
  %     \end{acronym}
  %   \end{spacing}

  \clearpage
  \pagenumbering{arabic}
  \chapter{Week 4: The Weirdness Begins}
    \section{Problem 1: Velocity → Lorentz factor}
      Calculate the value of the Lorentz factor for the following velocities.
      \setsol{300000}
      \FPeval{velocitya}{0.01}
      \lorentzr{\resulta}{\velocitya * sol}
      \FPeval{velocityb}{0.1}
      \lorentzr{\resultb}{\velocityb * sol}
      \FPeval{velocityc}{0.25}
      \lorentzr{\resultc}{\velocityc * sol}
      \FPeval{velocityd}{0.5}
      \lorentzr{\resultd}{\velocityd * sol}
      \FPeval{velocitye}{0.75}
      \lorentzr{\resulte}{\velocitye * sol}
      \FPeval{velocityf}{0.9}
      \lorentzr{\resultf}{\velocityf * sol}
      \FPeval{velocityg}{0.99}
      \lorentzr{\resultg}{\velocityg * sol}
      \FPeval{velocityh}{0.999}
      \lorentzr{\resulth}{\velocityh * sol}
      \begin{align*}
        \gamma &= \frac{1}{\sqrt{1 - {\left(\frac{v}{c}\right)}^2}}\\
        \intertext{(a)}
          \gamma(v = \SI{\velocitya}{\c}) &= \frac{1}{\sqrt{1 - {\velocitya}^2}}\\
          &= \frac{1}{\sqrt{1 - 0.0001}}\\
          &= \frac{1}{\sqrt{0.9999}}\\
          &= \frac{1}{\sqrt{\frac{9999}{10000}}}\\
          &= \frac{1}{\frac{99.995}{100}}\\
          &= \frac{100}{99.995}\\
          &\approx \resulta \cmark
        \intertext{(b)}
          \gamma(v = \SI{\velocityb}{\c}) &\approx \resultb \cmark
        \intertext{(c)}
          \gamma(v = \SI{\velocityc}{\c}) &\approx \resultc \cmark
        \intertext{(d)}
          \gamma(v = \SI{\velocityd}{\c}) &\approx \resultd \cmark
        \intertext{(e)}
          \gamma(v = \SI{\velocitye}{\c}) &\approx \resulte \cmark
        \intertext{(f)}
          \gamma(v = \SI{\velocityf}{\c}) &\approx \resultf \cmark
        \intertext{(g)}
          \gamma(v = \SI{\velocityg}{\c}) &\approx \resultg \cmark
        \intertext{(h)}
          \gamma(v = \SI{\velocityh}{\c}) &\approx \resulth \cmark
      \end{align*}

    \section{Problem 2: Lorentz factor →~velocity}\label{sec:lorentztospeed}
      Calculate the value of the relative velocity in km/second between two frames of reference
      for the following values of the Lorentz factor. (Use 300,000 km/second for the speed of light.)
      \setsol{300000}
      \FPeval{gammaa}{1.1}
      \lorentztospeedr{\resulta}{\gammaa}
      \FPeval{gammab}{1.25}
      \lorentztospeedr{\resultb}{\gammab}
      \FPeval{gammac}{1.5}
      \lorentztospeedr{\resultc}{\gammac}
      \FPeval{gammad}{2}
      \lorentztospeedr{\resultd}{\gammad}
      \FPeval{gammae}{5}
      \lorentztospeedr{\resulte}{\gammae}
      \FPeval{gammaf}{10}
      \lorentztospeedr{\resultf}{\gammaf}
      \begin{align*}
        \gamma &= \frac{1}{\sqrt{1 - \frac{v^2}{c^2}}}\\
         \sqrt{1 - \frac{v^2}{c^2}} &= \frac{1}{\gamma}\\
         1 - \frac{v^2}{c^2} &= \frac{1}{\gamma^2}\\
         - \frac{v^2}{c^2} &= \frac{1}{\gamma^2} - 1\\
         \frac{v^2}{c^2} &= 1 - \frac{1}{\gamma^2}\\
         v^2 &= c^2 \, \left(1 - \frac{1}{\gamma^2}\right)\\
         v &= \sqrt{c^2 \, \left(1 - \frac{1}{\gamma^2}\right)}\\
         v &= c \, \sqrt{1 - \frac{1}{\gamma^2}}
         \intertext{(a)}
          v(\gamma = \gammaa) &\approx \SI{\resulta}{\kilo\meter\per\second} \cmark
         \intertext{(b)}
          v(\gamma = \gammab) &\approx \SI{\resultb}{\kilo\meter\per\second} \cmark
         \intertext{(c)}
          v(\gamma = \gammac) &\approx \SI{\resultc}{\kilo\meter\per\second} \cmark
         \intertext{(d)}
          v(\gamma = \gammad) &\approx \SI{\resultd}{\kilo\meter\per\second} \cmark
         \intertext{(e)}
          v(\gamma = \gammae) &\approx \SI{\resulte}{\kilo\meter\per\second} \cmark
         \intertext{(f)}
          v(\gamma = \gammaf) &\approx \SI{\resultf}{\kilo\meter\per\second} \cmark
      \end{align*}

    \section{Problem 3: length contraction}
      Alice and Bob have identical spaceships, each with a proper length of [$D_0 = $] 50 meters. Bob docks
      his ship at the spaceport while Alice flies by him to the right at a speed of [$v =$]~\SI{0.6}{\c}.
      \begin{enumerate}[(a)]
        \item As Alice flies by, Bob measures the length of Alice’s ship. What value does he get?
          \FPeval{shiplen}{50}
          \FPeval{valuea}{0.6}
          \lencon{\resulta}{\shiplen}{\valuea}
          \FPround{\resulta}{\resulta}{0}
          \begin{align*}
            {D_{Bob}}_{Alice} &= \frac{D_0}{\gamma} = D_0 \, \sqrt{1 - \frac{v^2}{c^2}}\\
            {D_{Bob}}_{Alice}\left(v = \SI{0.6}{\c}\right) &= \SI{\resulta}{\meter}. \cmark
          \end{align*}
        \item At the same time, Alice measures the length of Bob’s ship. What value does she get?
          \begin{align*}
            {D_{Alice}}_{Bob}\left(v = \SI{0.6}{\c}\right) &= {D_{Bob}}_{Alice}\left(v = \SI{0.6}{\c}\right) = \SI{\resulta}{\meter}. \cmark
          \end{align*}
        \item Alice flies by a second time and this time Bob measures the length of Alice’s ship
          and gets a value of [${D_{Bob}}_{Alice} = $] 20 meters. How fast is Alice going with respect to Bob?
          \FPeval{valuec}{20}
          \FPeval{gammac}{round(shiplen/valuec :1)}
          \lorentztospeedr{\resultc}{\gammac}
          \FPeval{resultcinc}{round(resultc / sol :3)}
          \begin{align*}
            v &= c \, \sqrt{1 - \frac{1}{\gamma^2}}\\
            {D_{Bob}}_{Alice} &= \frac{D_0}{\gamma} \rightarrow \gamma = \frac{D_0}{{L_{Bob}}_{Alice}} = \frac{\SI{\shiplen}{\meter}}{\SI{\valuec}{\meter}} = \gammac{}\\
            v(\gamma = \gammac) &= \SI{\resultc}{\kilo\meter\per\second} \approx \SI{\resultcinc}{\c}. \cmark
          \end{align*}
      \end{enumerate}

    \section{Problem 4: Time dilation}
      Alice and Bob have identical spaceships, each equipped with identical
      light clocks. Bob docks his ship at the spaceport while Alice flies by
      him to the right at a speed of [$v = $]~\SI{0.6}{c}.
      \setsol{1}
      \FPeval{velocitya}{0.6}
      \begin{enumerate}[(a)]
        \item
          With his light clock next to him, Bob measures the length of one tick
          of the clock to be [$\mdelta{t_{Bob}} =$] \SI{0.001}{seconds}.
          (This value isn’t necessarily very realistic, i.e., it would be
          much shorter, on the order of a nanosecond.  But use the value given
          as it makes the numbers a little easier to work with.)

          As Alice flies by, he compares one tick of his clock to one tick
          of her clock. According to his observation, what is the length of
          her clock tick?
          \FPeval{timea}{0.001}
          \timedil{\resulta}{\timea}{\velocitya}
          \FPround{\resulta}{\resulta}{5}
          \begin{align*}
            \mdelta{{t_{Bob}}_{Alice}} &= \gamma(v) \, \mdelta{t_{Bob}} = \SI{\resulta}{\second}. \cmark
          \end{align*}
        \item
          Bob observes that, according to his clock (or really, his lattice of
          synchronized clocks), Alice’s ship takes [$\mdelta{{T_{Bob}}Alice} =$]
          60 minutes to travel from his location to a distant space station.

          When Alice passes that space station, Bob records the time on
          his clock (located at the space station, i.e., it’s part of his lattice
          of synchronized clocks) and the time on Alice’s clock by taking a photo
          of them.

          According to Bob’s photo, how much time [$\mdelta{T_{Alice}}$]
          has elapsed on Alice’s clock between the time she passed him and
          the time she passes the space station?
          \FPeval{timeb}{60}
          \FPeval{resultb}{\timeb * root(2, 1 - \velocitya^2)}
          \FPround{\resultb}{\resultb}{0}
          \begin{align*}
            \mdelta{T_{Alice}} &= \frac{\mdelta{{T_{Bob}}_{Alice}}}{\gamma(v)} = \SI{\resultb}{\minute}. \cmark
          \end{align*}
        \item
          \begin{samepage}
            How does Alice, in her frame of reference, see and explain
            the lengths and times involved?

            \begin{itemize}
              \item For Alice, the \strong{length} to the space station is
                length-contracted, so she needs less time to travel to it
                than Bob measures. \cmark
              \item \strong{Time} is running normally for Alice in her frame of reference, so
                she has to conclude that Bob's clocks are not synchronized:
                his leading clock (the one at the space station) lags. \cmark
            \end{itemize}
          \end{samepage}
      \end{enumerate}

    \section{Problem 5: Relativistic particles}
      Imagine that you are a physicist studying particles created by cosmic rays
      colliding with atoms in the upper atmosphere at a height of [$h = $]~10 kilometers
      above the surface of the Earth. Using special equipment attached to a large
      weather balloon, you discover that a certain kind of particle is detected
      as low as [$ h_{min} =$]~1500 meters above the surface, but no lower.
      In other words, it is created at a height of 10 kilometers and travels down
      to a height of 1.5 kilometers, where it decays into other particles. You
      also measure the speed of the original particle and get a value of
      [$v =$]~\SI{0.9}{\c}.

      Given this information, what is the lifetime [$\tau_0$] of the particle
      (in seconds) in its own frame of reference (i.e., when it is at rest, not
      in motion)?  (The lifetime is the amount of time it exists between being
      created and decaying into other particles. Also note: The numbers given here
      do not correspond to any actual particle, but are just for the exercise.)
      \FPeval{height}{10}
      \FPeval{minheight}{1.5}
      \setsol{\solSI/1000}
      \FPround{\sol}{\sol}{0}
      \FPeval{speed}{0.9}
      \lorentz{\gammas}{\speed * sol}
      \FPeval{taume}{(height - minheight)/(speed * sol)}
      \FPeval{result}{(height - minheight)/(speed * sol * gammas)}
      \FPround{\gammas}{\gammas}{5}
      \FPround{\taume}{\taume}{7}
      \FPround{\result}{\result}{7}
      \begin{align*}
        h &= \SI{10}{\kilo\meter}\
          \qquad h_{min} = \SI{1500}{\meter}\
          \qquad v = \SI{0.9}{\c}\\
        \mdelta{h} &= h - h_{min} = \SI{8.5}{\kilo\meter}\\
        \tau_{me} &= \frac{\mdelta{h}}{v} \approx \SI{\taume}{\second} \cmark\\
        \tau_0 &= \frac{\tau_{me}}{\gamma(v)} = \frac{\mdelta{h}}{v \, \gamma(v)}\\
          &\approx \frac{\SI{\height}{\kilo\meter} - \SI{\minheight}{\kilo\meter}}{\speed \times \SI{\sol}{\kilo\meter\per\second} \times \gammas}\
          \approx \SI{\result}{\second}. \cmark
      \end{align*}

  \chapter{Week 5: Spacetime Switches}
    \section{Quiz: Leading clocks lag, revisited}
      \begin{enumerate}
        \item
          Imagine that Bob, at a spaceport, observes Alice flying by to the right
          in her spaceship at a velocity of [$v = $]\SI{0.9}{\c}. When Alice’s ship
          had been previously docked at the spaceport, Alice had measured its length as
          [$L = $]100~meters.  Alice has five clocks on her spaceship: one in front
          (at [$x_{A1}\,=\,$]0~meters), one at [$x_{A2}\,=\,-$]25~meters, one at
          [$x_{A3}\,=\,-$]50~meters, one at [$x_{A4}\,=\,-$]75~meters, and one at
          the rear at [$x_{A5}\,=\,-$]100~meters. All the clocks are synchronized in
          Alice’s frame of reference.  When Alice flies by Bob, Bob uses his
          lattice of synchronized clocks to take simultaneous photos of Alice’s
          clocks (simultaneous in Bob’s frame of reference
          [$t_{B1} = t_{B2} = \cdots = t_{B5}$]).

          If the photos show that Alice’s front clock reads $t_{[A1]} [= t_{B1}] = 0$
          at that instant, what equation below would give the time [$t_{A5}$]
          that Alice's clock at [$x_{A5}\,=\,-$]100~meters has in the photos?
          \begin{enumerate}[(a)]
            \item $t = \frac{90}{\c}$
            \item $t = 90 \, \c^2$
            \item $t = 0$
            \item $t = 0.9 \, \c^2$
          \end{enumerate}
          Solution:
          \begin{align*}
            v &= \SI{0.9}{\c}\\
            L &= \SI{100}{\meter}\\
            x_{A1} &= 0\\
            t_{A1} &= 0\\
            x_{A5} &= \SI{-100}{\meter}\\
            \\
            t_{A5} &= \gamma \, \left(t_{B5} - \frac{v}{\c^2} x_{B5}\right)\\
            t_{B5} &= t_{B1} = t_{A1} = 0\\
            x_{B5} &= \frac{x_{A5}}{\gamma}\\
            t_{A5} &= \gamma \, \left(t_{B5} - \frac{v}{\gamma \, \c^2} x_{A5}\right)\\
            &= \gamma \, \left(0 - \frac{v}{\gamma \, \c^2} \, x_{A5}\right)\\
            &= \cancel{\gamma} \, \left(-\frac{v}{\cancel{\gamma} \, \c^2} \, x_{A5}\right)\\
            t_{A5} &= -\frac{\SI{0.9}{\c}}{\c^2} \times (\SI{-100}{\meter})\\
            t_{A5} &= \frac{\SI{90}{\meter}}{\c} \rightarrow \text{(a) \cmark}
            \intertext{Or, in this case:}
            t_{A5} &= \frac{L \, v}{\c^2}\\
            L &= \left|x_{A5} - x_{A0}\right|\\
            t_{A5} &= \frac{\SI{100}{\meter} \times \SI{0.9}{\c}}{\c^2}\\
            t_{A5} &= \frac{\SI{90}{\meter}}{\c} \rightarrow \text{(a) \cmark}
          \end{align*}
      \end{enumerate}

    \section{Light emitted perpendicular to the direction of motion (forum question)}
      \blockquote[Antonio Rubio Rodríguez]{I am confused with the emision of ligth in perpendicular direction to the velocity of the pod. The velocity observed by rest frame, will be c/gamma, according algebra, but it is not consistent with postulate.}

      Question: Does this equation apply to light as well?
      \begin{align*}
        \vec{u_R} &= \left({(u_R)}_x, \, {(u_R)}_y, \, {(u_R)}_z\right)\\
        {(u_R)}_x &= \frac{\mdelta{x_R}}{\mdelta{t_R}}, \qquad
        {(u_R)}_y = \frac{\mdelta{y_R}}{\mdelta{t_R}}, \qquad
        {(u_R)}_z = \frac{\mdelta{z_R}}{\mdelta{t_R}}\\
        \\
        {(u_L)}_y &= \frac{\mdelta{y_L}}{\mdelta{t_L}}\\
          &= \frac{\mdelta{y_R}}{\mdelta{t_L}}\\
          &= \frac{\mdelta{y_R}}{\gamma \, \mdelta{t_R} \, \left(1 + \frac{v}{\c^2} \, \mdelta{x_R}\right)}\\
          &= \frac{\mdelta{y_R}}{\gamma \, \mdelta{t_R}}\\
          &= \frac{{(u_R)}_y}{\gamma}\\
        \vec{u_L}
          &= \left({(u_L)}_x, \, {(u_L)}_y, \, {(u_L)}_z\right)\\
          &= \left(\frac{{(u_R)}_x + v}{1 + \frac{v}{\c^2} \, {(u_R)}_x}, \, \frac{\c}{\gamma}, \, 0\right)\\
        \left|\vec{u_L}\right|
          &= \sqrt{{{(u_L)}_x}^2 + {{(u_L)}_y}^2 + {{(u_L)}_z}^2}\\
          &= \sqrt{v^2 + {\left(\frac{\c}{\gamma}\right)}^2}\\
        \left|\vec{u_L}\right|
          &= \sqrt{v^2 + \frac{\c^2}{{\gamma}^2}}\\
          &= \sqrt{v^2 + \frac{\c^2}{{\left(\frac{1}{\sqrt{1 - \frac{v^2}{\c^2}}}\right)}^2}}\\
          &= \sqrt{v^2 + \frac{\c^2}{\left(\frac{1}{1 - \frac{v^2}{\c^2}}\right)}}\\
          &= \sqrt{v^2 + \c^2 \, \left(1 - \frac{v^2}{\c^2}\right)}\\
          &= \sqrt{v^2 + \c^2 - \c^2 \, \frac{v^2}{\c^2}}\\
          &= \sqrt{v^2 + \c^2 - v^2}\\
          &= \sqrt{\c^2}\\
          &= \c
          \intertext{In general:}
          \vec{u_R} &= \left({(u_R)}_x, \, {(u_R)}_y, \, {(u_R)}_z \right)\\
          \vec{u_L} &= \left({(u_L)}_x, \, {(u_L)}_y, \, {(u_L)}_z \right)\\
            &= \left(\frac{\Delta{x}_L}{\Delta{t}_L}, \, \frac{\Delta{y}_L}{\Delta{t}_L}, \, \frac{\Delta{z}_L}{\Delta{t}_L} \right)\\
            &= \left(\frac{{(u_R)}_x + v}{1 + \frac{v}{\c^2} {(u_R)}_x}, \,
                 \frac{{(u_R)}_y}{\gamma \, \left(1 + \frac{v}{\c^2} \, {(u_R)}_x\right)}, \,
                 \frac{{(u_R)}_z}{\gamma \, \left(1 + \frac{v}{\c^2} \, {(u_R)}_x\right)} \right)\\
          \intertext{For a light beam parallel to the coordinate axes:}
          \vec{u_R}_x &:= \left(\c, \, 0, \, 0 \right)\\
          \vec{u_L}_x &= \left(\frac{\c + v}{1 + \frac{v}{c}}, \,
            0, \,0 \right)\\
          \left|\vec{u_L}_x\right|
            &= \sqrt{{\left(\frac{\c + v}{1 + \frac{v}{c}}\right)}^2}
             = \frac{\c + v}{1 + \frac{v}{c}}
             = \frac{\c \, (1 + \frac{v}{\c})}{1 + \frac{v}{c}} = \c{}\\
          \vec{u_R}_y &:= \left(0, \, \c, \, 0 \right)\\
          \vec{u_L}_y &= \left(v, \,
            \frac{\c}{\gamma}, \,0 \right)\\
          \left|\vec{u_L}_y\right|
            &= \sqrt{v^2 + {\left(\frac{\c}{\gamma}\right)}^2}
             = \sqrt{v^2 + {\left(\frac{\c}{\frac{1}{\sqrt{1 -  \frac{v^2}{\c^2}}}}\right)}^2}
             = \sqrt{v^2 + {\left(\c \, \sqrt{1 -  \frac{v^2}{\c^2}}\right)}^2}\\
          &= \sqrt{v^2 + {\c}^2 \, \left(1 -  \frac{v^2}{\c^2}\right)}\\
          &= \sqrt{v^2 + {\c}^2 - v^2}\\
          &= \sqrt{{\c}^2}\\
          &= \c{}\\
        \vec{u_R}_z &:= \left(0, \, 0, \, \c \right)\\
        \vec{u_L}_z &= \left(v, \,0, \,
          \frac{\c}{\gamma} \right)\\
        \left|\vec{u_L}_z\right| &= \c
        \intertext{For a light beam in any direction,}
        \left|\vec{u_R}\right| &= \sqrt{
            {\left({\left(u_R\right)}_x\right)}^2
          + {\left({\left(u_R\right)}_y\right)}^2
          + {\left({\left(u_R\right)}_z\right)}^2
        } = \c{}\\
        \intertext{must hold.  That is:}
        \left|\vec{u_L}\right|
        &= \sqrt{
            {\left(\frac{{(u_R)}_x + v}{1 + \frac{v}{\c^2} {(u_R)}_x}\right)}^2 +
            {\left(\frac{{(u_R)}_y}{\gamma \, \left(1 + \frac{v}{\c^2} \, {(u_R)}_x\right)}\right)}^2 +
            {\left(\frac{{(u_R)}_z}{\gamma \, \left(1 + \frac{v}{\c^2} \, {(u_R)}_x\right)}\right)}^2
          }\\
        &= \sqrt{
              {\left({\left(u_R\right)}_x\right)}^2
            + {\left({\left(u_R\right)}_y\right)}^2
            + {\left({\left(u_R\right)}_z\right)}^2
          }\\
        {\left(u_R\right)}_x
          &= \frac{{(u_R)}_x + v}{1 + \frac{v}{\c^2} {(u_R)}_x}\\
        {\left(u_R\right)}_x &= {(u_R)}_x + v\\
        1  &= 1 + \frac{v}{\c^2} {(u_R)}_x\\
          \intertext{[TODO]}
        {\left(u_R\right)}_y
          &= \frac{{(u_R)}_y}{\gamma \, \left(1 + \frac{v}{\c^2} \, {(u_R)}_x\right)}\\
          \intertext{[TODO]}
        {\left(u_R\right)}_z
          &= \frac{{(u_R)}_z}{\gamma \, \left(1 + \frac{v}{\c^2} \, {(u_R)}_x\right)}\\
          \intertext{[TODO]}
      \end{align*}

    \section{Lorentz transformation}
      \subsection{Problem 1: Simple calculations}
        \setsol{1}
        \lorentztrafo{\resultax}{\resultat}{0}{10}{0.1}
        \lorentztrafo{\resultbx}{\resultbt}{0}{20}{0.1}
        \lorentztrafo{\resultcx}{\resultct}{0}{10}{0.5}
        \lorentztrafo{\resultdx}{\resultdt}{0}{20}{0.5}
        \lorentztrafo{\resultex}{\resultet}{0}{10}{0.99}
        \lorentztrafo{\resultfx}{\resultft}{0}{20}{0.99}
        \lorentztrafo{\resultgx}{\resultgt}{100}{0}{0.5}
        \lorentztrafo{\resulthx}{\resultht}{100}{10}{0.5}
        \lorentztrafo{\resultix}{\resultit}{100}{20}{0.5}
        \lorentztrafo{\resultjx}{\resultjt}{1000}{0}{0.5}
        \lorentztrafo{\resultkx}{\resultkt}{1000}{10}{0.5}
        \lorentztrafo{\resultlx}{\resultlt}{1000}{20}{0.5}
        \lorentztrafo{\resultmx}{\resultmt}{100000}{0}{0.5}
        \lorentztrafo{\resultnx}{\resultnt}{100000}{10}{0.5}
        \lorentztrafo{\resultox}{\resultot}{100000}{20}{0.5}
        \lorentztrafo{\resultpx}{\resultpt}{100}{0}{0.99}
        \lorentztrafo{\resultqx}{\resultqt}{100}{10}{0.99}
        \lorentztrafo{\resultrx}{\resultrt}{100}{20}{0.99}
        \lorentztrafo{\resultsx}{\resultst}{1000}{0}{0.99}
        \lorentztrafo{\resulttx}{\resulttt}{1000}{10}{0.99}
        \lorentztrafo{\resultux}{\resultut}{1000}{20}{0.99}
        \lorentztrafo{\resultvx}{\resultvt}{100000}{0}{0.99}
        \lorentztrafo{\resultwx}{\resultwt}{100000}{10}{0.99}
        \lorentztrafo{\resultxx}{\resultxt}{100000}{20}{0.99}
        \begin{align*}
          x_A &= \gamma \, \left(x_B + v \, t_B\right)\\
          t_A &= \gamma \, \left(t_B + \frac{v}{\c^2} \, x_B\right)\\
          \c &= 1
          \intertext{(a) $v = \SI{0.1}{\c}$, $x_B = 0$, $t_B = 10$}
          x_A
            &= \gamma \, \left(x_B + v \, t_B\right)\\
            &= \frac{1}{\sqrt{1 - \frac{v^2}{\c^2}}} \, \left(x_B + v \, t_B\right)\\
            &= \frac{1}{\sqrt{1 - {(0.1)}^2}} \, \left(0 + 0.1 \times 10\right)\\
            &= \frac{1}{\sqrt{1 - 0.01}}\\
            &= \frac{1}{\sqrt{0.99}}\\
            &\approx \resultax{} \cmark{}\\
          t_A
            &= \gamma \, \left(t_B + \frac{v}{\c^2} \, x_B\right)\\
            &= \frac{1}{\sqrt{1 - \frac{v^2}{\c^2}}} \, \left(t_B + \frac{v}{\c^2} \, x_B\right)\\
            &= \frac{1}{\sqrt{1 - v^2}} \, \left(t_B + v \, x_B\right)\\
            &= \frac{1}{\sqrt{1 - {(0.1)}^2}} \, \left(10 + 0.1 \times 0\right)\\
            &= \frac{1}{\sqrt{1 - 0.01}} \times 10\\
            &= \frac{1}{\sqrt{0.99}} \times 10\\
            &\approx \resultat{} \cmark
          \intertext{(b) $v = \SI{0.1}{\c}$, $x_B = 0$, $t_B = 20$}
          x_A &\approx \resultbx{} \cmark\\
          t_A &\approx \resultbt{} \cmark
          \intertext{(c) $v = \SI{0.5}{\c}$, $x_B = 0$, $t_B = 10$}
          x_A &\approx \resultcx{} \cmark\\
          t_A &\approx \resultct{} \cmark
          \intertext{(d) $v = \SI{0.5}{\c}$, $x_B = 0$, $t_B = 20$}
          x_A &\approx \resultdx{} \cmark\\
          t_A &\approx \resultdt{} \cmark
          \intertext{(e) $v = \SI{0.99}{\c}$, $x_B = 0$, $t_B = 10$}
          x_A &\approx \resultex{} \cmark\\
          t_A &\approx \resultet{} \cmark
          \intertext{(f) $v = \SI{0.99}{\c}$, $x_B = 0$, $t_B = 20$}
          x_A &\approx \resultfx{} \cmark\\
          t_A &\approx \resultft{} \cmark
          \intertext{(g) $v = \SI{0.5}{\c}$, $x_B = 100$, $t_B = 0$}
          x_A &\approx \resultgx{} \cmark\\
          t_A &\approx \resultgt{} \cmark
          \intertext{(h) $v = \SI{0.5}{\c}$, $x_B = 100$, $t_B = 10$}
          x_A &\approx \resulthx{} \cmark\\
          t_A &\approx \resultht{} \cmark
          \intertext{(i) $v = \SI{0.5}{\c}$, $x_B = 100$, $t_B = 20$}
          x_A &\approx \resultix{} \cmark\\
          t_A &\approx \resultit{} \cmark
          \intertext{(j) $v = \SI{0.5}{\c}$, $x_B = 1000$, $t_B = 0$}
          x_A &\approx \resultjx{} \cmark\\
          t_A &\approx \resultjt{} \cmark
          \intertext{(k) $v = \SI{0.5}{\c}$, $x_B = 1000$, $t_B = 10$}
          x_A &\approx \resultkx{} \cmark\\
          t_A &\approx \resultkt{} \cmark
          \intertext{(l) $v = \SI{0.5}{\c}$, $x_B = 1000$, $t_B = 20$}
          x_A &\approx \resultlx{} \cmark\\
          t_A &\approx \resultlt{} \cmark
          \intertext{(m) $v = \SI{0.5}{\c}$, $x_B = 100000$, $t_B = 0$}
          x_A &\approx \resultmx{} \cmark\\
          t_A &\approx \resultmt{} \cmark
          \intertext{(n) $v = \SI{0.5}{\c}$, $x_B = 100000$, $t_B = 10$}
          x_A &\approx \resultnx{} \cmark\\
          t_A &\approx \resultnt{} \cmark
          \intertext{(o) $v = \SI{0.5}{\c}$, $x_B = 100000$, $t_B = 20$}
          x_A &\approx \resultox{} \cmark\\
          t_A &\approx \resultot{} \cmark
          \intertext{(p) $v = \SI{0.99}{\c}$, $x_B = 100$, $t_B = 0$}
          x_A &\approx \resultpx{} \cmark\\
          t_A &\approx \resultpt{} \cmark
          \intertext{(q) $v = \SI{0.99}{\c}$, $x_B = 100$, $t_B = 10$}
          x_A &\approx \resultqx{} \cmark\\
          t_A &\approx \resultqt{} \cmark
          \intertext{(r) $v = \SI{0.99}{\c}$, $x_B = 100$, $t_B = 20$}
          x_A &\approx \resultrx{} \cmark\\
          t_A &\approx \resultrt{} \cmark
          \intertext{(s) $v = \SI{0.99}{\c}$, $x_B = 1000$, $t_B = 0$}
          x_A &\approx \resultsx{} \cmark\\
          t_A &\approx \resultst{} \cmark
          \intertext{(t) $v = \SI{0.99}{\c}$, $x_B = 1000$, $t_B = 10$}
          x_A &\approx \resulttx{} \cmark\\
          t_A &\approx \resulttt{} \cmark
          \intertext{(u) $v = \SI{0.99}{\c}$, $x_B = 1000$, $t_B = 20$}
          x_A &\approx \resultux{} \cmark\\
          t_A &\approx \resultut{} \cmark
          \intertext{(v) $v = \SI{0.99}{\c}$, $x_B = 100000$, $t_B = 0$}
          x_A &\approx \resultvx{} \cmark\\
          t_A &\approx \resultvt{} \cmark
          \intertext{(w) $v = \SI{0.99}{\c}$, $x_B = 100000$, $t_B = 10$}
          x_A &\approx \resultwx{} \cmark\\
          t_A &\approx \resultwt{} \cmark
          \intertext{(x) $v = \SI{0.99}{\c}$, $x_B = 100000$, $t_B = 20$}
          x_A &\approx \resultxx{} \cmark\\
          t_A &\approx \resultxt{} \cmark
        \end{align*}

      \subsection{Problem 2: Star Tour to Gliese 581}
        Lorena is traveling to Gliese 581, observed by Cynthia on Earth.
        \begin{align*}
          D &= \SI{20.3}{\lightyear}\\
          v &= \SI{0.99}{\c}\\
          t_{Lorena} &= t_{Cynthia} = 0
        \end{align*}
        \FPeval{length}{20.3}
        \FPeval{speed}{0.99}
        \lorentz{\gammav}{\speed}
        \FPround{\gammav}{\gammav}{2}
        \begin{enumerate}[(a)]
          \item
            \FPeval{resulta}{length/speed}
            \FPround{\resulta}{\resulta}{2}
            \begin{align*}
              T_{Cynthia}
                &= \frac{D}{v}
                = \frac{\SI{\length}{\lightyear}}{\SI{\speed}{\c}}
                \approx \SI{\resulta}{\year} \cmark
            \end{align*}
          \item
            \FPeval{\resultb}{length/gammav}
            \FPround{\resultb}{\resultb}{2}
            \begin{align*}
              D_{Lorena}
                &= \frac{D}{\gamma(v)}
                \approx \frac{\SI{\length}{\lightyear}}{\gammav}
                \approx \SI{\resultb}{\lightyear} \cmark
            \end{align*}
          \item
            \FPeval{resultc}{resultb/speed}
            \FPround{\resultc}{\resultc}{2}
            \begin{align*}
              T_{Lorena}
                &= \frac{D_{Lorena}}{v}
                = \frac{\SI{\resultb}{\lightyear}}{\SI{\speed}{\c}}
                \approx \SI{\resultc}{\year} \cmark
            \end{align*}
          \item
            \begin{align*}
              T_{Cynthia} &\approx \SI{\resulta}{\year} \text{ [same as (a)]} \cmark
            \end{align*}
          \item
            \begin{align*}
              {(T_{Lorena})}_{Cynthia} \approx \SI{\resultc}{\year} \text{ [same as (c)]} \cmark
            \end{align*}
          \item
            \FPeval{resultf}{resultc/gammav}
            \FPround{\resultf}{\resultf}{2}
            \begin{align*}
              {(T_{Cynthia})}_{Lorena}
                &= \frac{T_{Lorena}}{\gamma(v)}
                \approx \frac{\SI{\resultc}{\year}}{\gammav}
                \approx \SI{\resultf}{\year} \cmark
            \end{align*}
          \item
            \FPeval{resultg}{resulta - resultf}
            \FPround{\resultg}{\resultg}{2}
            \begin{align*}
              {(\mdelta{t}_{Cynthia})}_{Lorena}
                &= T_{Cynthia} - {(T_{Cynthia})}_{Lorena}
                \approx \SI{\resulta}{\year} - \SI{\resultf}{\year}
                \approx \SI{\resultg}{\year} \cmark
            \end{align*}
        \end{enumerate}

      \subsection{Problem 3: Combining velocities}
        \setsol{1}
        \begin{enumerate}[(a)]
          \item
            \FPeval{speed}{0.2}
            \FPeval{speeda}{0.1}
            \combinespeeds{\resulta}{\speed}{\speeda}
            \FPround{\resulta}{\resulta}{2}
            \begin{align*}
              u_L
                &= \frac{u_R + v}{1 + \frac{v}{\c^2} \, u_R}
                = \frac{\SI{\speeda}{\c} + \SI{\speed}{\c}}{1 + \frac{\SI{\speed}{\c}}{\c^2} \times \SI{\speeda}{\c}}
                \approx \SI{\resulta}{\c} \cmark
            \end{align*}
          \item
            \FPeval{speed}{0.5}
            \FPeval{speeda}{0.3}
            \combinespeeds{\resulta}{\speed}{\speeda}
            \FPround{\resulta}{\resulta}{2}
            \begin{align*}
              u_L
                &= \frac{u_R + v}{1 + \frac{v}{\c^2} \, u_R}
                = \frac{\SI{\speeda}{\c} + \SI{\speed}{\c}}{1 + \frac{\SI{\speed}{\c}}{\c^2} \times \SI{\speeda}{\c}}
                \approx \SI{\resulta}{\c} \cmark
            \end{align*}
          \item
            \FPeval{speed}{0.8}
            \FPeval{speeda}{0.7}
            \combinespeeds{\resulta}{\speed}{\speeda}
            \FPround{\resulta}{\resulta}{2}
            \begin{align*}
              u_L
                &= \frac{u_R + v}{1 + \frac{v}{\c^2} \, u_R}
                = \frac{\SI{\speeda}{\c} + \SI{\speed}{\c}}{1 + \frac{\SI{\speed}{\c}}{\c^2} \times \SI{\speeda}{\c}}
                \approx \SI{\resulta}{\c} \cmark
            \end{align*}
          \item
            \FPeval{speed}{0.9}
            \FPeval{speeda}{0.9}
            \combinespeeds{\resulta}{\speed}{\speeda}
            \FPround{\resulta}{\resulta}{3}
            \begin{align*}
              u_L
                &= \frac{u_R + v}{1 + \frac{v}{\c^2} \, u_R}
                = \frac{\SI{\speeda}{\c} + \SI{\speed}{\c}}{1 + \frac{\SI{\speed}{\c}}{\c^2} \times \SI{\speeda}{\c}}
                \approx \SI{\resulta}{\c} \cmark
            \end{align*}
          \item
            \FPeval{speed}{0.999}
            \FPeval{speeda}{0.999}
            \combinespeeds{\resulta}{\speed}{\speeda}
            \FPround{\resulta}{\resulta}{7}
            \begin{align*}
              u_L
                &= \frac{u_R + v}{1 + \frac{v}{\c^2} \, u_R}
                = \frac{\SI{\speeda}{\c} + \SI{\speed}{\c}}{1 + \frac{\SI{\speed}{\c}}{\c^2} \times \SI{\speeda}{\c}}
                \approx \SI{\resulta}{\c}
            \end{align*}
        \end{enumerate}

      \subsection{Problem 4: Practical application}
        \setsol{1}
        \FPeval{speed}{0.8}
        \FPneg{\speed}{\speed}
        \lorentztrafo{\resultax}{\resultat}{4}{5}{\speed}
        \lorentztrafo{\resultbx}{\resultbt}{0}{4}{\speed}
        \FPround{\resultat}{\resultat}{0}
        \FPround{\resultbt}{\resultbt}{1}
        \begin{align*}
          v &= \SI{0.8}{\c}\\
          D &= \SI{6}{\lightyear}\\
          x_E &= \SI{4}{\lightyear}\\
          t_E &= \SI{5}{\year}
          \intertext{(a)}
            t_B
              &= \gamma \, \left(t_E - \frac{v}{\c^2} \, x_E\right)\\
              &= \frac{1}{\sqrt{1 - \frac{v^2}{\c^2}}} \, \left(t_E - \frac{v}{\c^2} \, x_E\right)\\
              &= \frac{1}{\sqrt{1 - {\left(0.8\right)}^2}} \, \left(\SI{5}{\year} - \SI{0.8}{\year\per\lightyear} \times \SI{4}{\lightyear}\right)\\
              &= \frac{1}{\sqrt{1 - 0.64}} \, \left(\SI{5}{\year} - \SI{3.2}{\year}\right)\\
              &= \frac{1}{\sqrt{0.36}} \times \SI{1.8}{\year}\\
              &= \frac{1}{0.6} \times \SI{1.8}{\year}\\
              &= \SI{\resultat}{\year} \cmark
          \intertext{(b) $t_E = \SI{4}{\year}$, $x_E = 0$}
            {(t_{Alice~start})}_B
              &= \gamma \, \left(t_E - \frac{v}{\c^2} \, x_E\right)
              \approx \SI{\resultbt}{\year} \cmark
        \end{align*}
        (c) Alice appears to come from Bob's future

  \chapter{Week 6: Breaking the Spacetime Limit?}
    \section{Problem 1: Combining spacetime diagrams}
      Bob moves at $v = \SI{0.4}{\c}$ to the right, observed by Alice.
      \setsol{1}
      \FPeval{speed}{0.4}
      \lorentz{\gammav}{\speed}
      \FPround{\gammav}{\gammav}{2}
      \FPeval{resulta}{1/speed}
      \FPround{\resulta}{\resulta}{1}
      \begin{enumerate}[(a)]
        \item Bob's $t_B$ axis ($x_B = 0$) in Alice's diagram:
          \begin{align*}
            \Aboxed{x_A &= \gamma \, \left(x_B + v \, t_B\right)}\\
            \frac{x_A}{\gamma} &= x_B + v \, t_B\\
            \frac{x_A}{\gamma} - x_B &= v \, t_B\\
            t_B &= \frac{x_A}{\gamma \, v} - \frac{x_B}{v}\\
            \Aboxed{t_A &= \gamma \, \left(t_B + \frac{v}{\c^2} \, x_B\right)}\\
              &= \gamma \, \left(\left(\frac{x_A}{\gamma \, v} - \frac{x_B}{v}\right) + \frac{v}{\c^2} \, x_B\right)\\
              &= \cancel{\gamma} \, \frac{1}{\cancel{\gamma} \, v} \, x_A - \gamma \, \frac{1}{v} \, x_B + \gamma \, \frac{v}{\c^2} \, x_B\\
              &= \frac{1}{\, v} \, x_A - \gamma \, \left(\frac{1}{v} \, x_B - \frac{v}{\c^2} \, x_B\right)\\
              &= \frac{1}{\, v} \, x_A - \gamma \, \left(\frac{1}{v} - \frac{v}{\c^2}\right) \, x_B\\
            \Aboxed{t_A &= \frac{1}{v} \, x_A - \gamma(v) \, \left(\frac{1}{v} - \frac{v}{\c^2}\right) \, x_B}\\
            t_A(x_B = 0) &= \frac{1}{v} \, x_A\\
            t_A(x_B = 0, \, v = \SI{\speed}{\c}, \, \c = 1) &= \frac{1}{\speed} \, x_A = \resulta \, x_A \cmark
          \end{align*}
          Easier:
          \begin{align*}
            x_B &= \gamma \, \left(x_A - v \, t_A\right)\\
              0 &= \gamma \, \left(x_A - 0.4 \times t_A\right)\\
            0.4 \, t_A &= x_A\\
            t_A &= \frac{x_A}{0.4}
          \end{align*}
        \item Bob's line of same location for $x_B = 3$ on Alice's diagram:
          \FPeval{xB}{3}
          \FPeval{resultb}{gammav * (1/speed - speed) * xB}
          \FPround{\resultb}{\resultb}{3}
          \begin{align*}
            \Aboxed{t_A &= \frac{1}{v} \, x_A - \gamma(v) \, \left(\frac{1}{v} - \frac{v}{\c^2}\right) \, x_B}\\
            t_A(x_B = \xB, \, v = \SI{\speed}{\c}, \, \c = 1)
              &\approx \frac{1}{\speed} \, x_A - \gammav \times \left(\frac{1}{\speed} - \speed\right) \times \xB{}\\
              &\approx \resulta \, x_A - \resultb \cmark
          \end{align*}
          Easier:
          \begin{align*}
            x_B &= \gamma \, \left(x_A - v \, t_A\right)\\
              3 &= \gamma \, \left(x_A - 0.4 \, t_A\right)\\
              0.4 \, t_A &= x_A -\frac{3}{\gamma}\\
              t_A &= \frac{x_A}{0.4} - \frac{3}{0.4 \, \gamma}\\
                &\approx \frac{x_A}{0.4} - 6.8738635
          \end{align*}
        \item Bob's $x_B$ axis [$t_B = 0$] on Alice's diagram:
          \begin{align*}
            \Aboxed{x_A &= \gamma \, \left(x_B + v \, t_B\right)}\\
            x_B &= \frac{x_A}{\gamma} - v \, t_B\\
            \Aboxed{t_A &= \gamma \, \left(t_B + \frac{v}{\c^2} \, x_B\right)}\\
              &= \gamma \, \left(t_B + \frac{v}{\c^2} \, \left(\frac{x_A}{\gamma} - v \, t_B\right)\right)\\
              &= \gamma \, t_B + \cancel{\gamma} \, \frac{v}{\cancel{\gamma} \, \c^2} \, x_A - \gamma \, \frac{v}{\c^2} v \, t_B\\
            \Aboxed{t_A &= \frac{v}{\c^2} \, x_A + \gamma \, \left(1 - \frac{v^2}{\c^2}\right) \, t_B}\\
            t_A(t_B = 0) &= \frac{v}{\c^2} \, x_A\\
            t_A(t_B = 0, \, v = \SI{\speed}{\c}, \, \c = 1) &= \speed \, x_A \cmark
          \end{align*}
          Easier:
          \begin{align*}
            t_B &= \gamma \, \left(t_A - \frac{v}{\c^2} \, x_A\right)\\
              0 &= \gamma \, \left(t_A - 0.4 \, x_A\right)\\
              0 &= t_A - 0.4 \, x_A\\
              t_A &= 0.4 \, x_A
          \end{align*}
        \item Bob's line of simultaneity for $t_B = 3$ on Alice's diagram:
          \FPeval{tB}{3}
          \FPeval{resultd}{gammav * (1 - speed^2) * tB}
          \FPround{\resultd}{\resultd}{3}
          \begin{align*}
            t_A(t_B = \tB, \, v = \SI{\speed}{\c}, \, \c = 1)
              &= \frac{v}{\c^2} \, x_A + \gamma \, \left(1 - \frac{v^2}{\c^2}\right) \, t_B\\
              &\approx \speed \, x_A + \gammav \, \left(1 - {(\speed)}^2\right) \times \tB{}\\
              &\approx \speed \, x_A + \resultd{} \cmark\\
          \end{align*}
          Easier:
          \begin{align*}
            t_B &= \gamma \, \left(t_A - \frac{v}{\c^2} \, x_A\right)\\
            3 &= \gamma \, \left(t_A - 0.4 \, x_A\right)\\
            t_A &= 0.4 \, x_A + \frac{3}{\gamma}
          \end{align*}
      \end{enumerate}

    \section{Problem 2: Regions of spacetime}
      \begin{enumerate}[(a)]
        \setsol{1}
        \FPeval{xa}{6}
        \FPeval{ta}{3}
        \FPeval{xb}{4}
        \FPeval{tb}{6}
        \item $x_1 = \xa$, $t_1 = \ta$ and $x_2 = \xb$, $t_2 = \tb$.
          \spacetimeintv{\result}{\xa}{\ta}{\xb}{\tb}
          \FPround{\result}{\result}{0}
          \begin{align*}
            \c^2 \, {(\mdelta{t})}^2 - {(\mdelta{x})}^2
              &= \c^2 \, {(t_2 - t_1)}^2 - {(x_2 - x_1)}^2\\
              &= {(t_2 - t_1)}^2 - {(x_2 - x_1)}^2 \qquad \c = \sol{}\\
              &= {(\tb - \ta)}^2 - {(\xb - \xa)}^2\\
              &= 3^2 - {(-2)}^2\\
              &= 9 - 4\\
              &= \result > 0 \rightarrow \text{timelike interval} \cmark
          \end{align*}

        \FPneg{\xa}{1}
        \FPround{\xa}{\xa}{0}
        \FPeval{ta}{3}
        \FPeval{xb}{4}
        \FPneg{\tb}{1}
        \FPround{\tb}{\tb}{0}
        \item $x_1 = \xa$, $t_1 = \ta$ and $x_2 = \xb$, $t_2 = \tb$.
          \spacetimeintv{\result}{\xa}{\ta}{\xb}{\tb}
          \FPround{\result}{\result}{0}
          \begin{align*}
            \c^2 \, {(\mdelta{t})}^2 - {(\mdelta{x})}^2
              &= \c^2 \, {(t_2 - t_1)}^2 - {(x_2 - x_1)}^2\\
              &= {(t_2 - t_1)}^2 - {(x_2 - x_1)}^2 \qquad \c = \sol{}\\
              &= {(\tb - \ta)}^2 - {(\xb - \xa)}^2\\
              &= \result < 0 \rightarrow\text{spacelike interval} \cmark
          \end{align*}

        \FPeval{xa}{7}
        \FPeval{ta}{3}
        \FPeval{xb}{4}
        \FPeval{tb}{0}
        \item $x_1 = \xa$, $t_1 = \ta$ and $x_2 = \xb$, $t_2 = \tb$.
          \spacetimeintv{\result}{\xa}{\ta}{\xb}{\tb}
          \FPround{\result}{\result}{0}
          \begin{align*}
            \c^2 \, {(\mdelta{t})}^2 - {(\mdelta{x})}^2
              &= \c^2 \, {(t_2 - t_1)}^2 - {(x_2 - x_1)}^2\\
              &= {(t_2 - t_1)}^2 - {(x_2 - x_1)}^2 \qquad \c = \sol{}\\
              &= {(\tb - \ta)}^2 - {(\xb - \xa)}^2\\
              &= \result \rightarrow \text{lightlike interval} \cmark
          \end{align*}

        \FPneg{\xa}{2}
        \FPround{\xa}{\xa}{0}
        \FPneg{\ta}{3}
        \FPround{\ta}{\ta}{0}
        \FPeval{\xb}{3}
        \FPneg{\tb}{2}
        \FPround{\tb}{\tb}{0}
        \item $x_1 = \xa$, $t_1 = \ta$ and $x_2 = \xb$, $t_2 = \tb$.
          \begin{samepage}
            \spacetimeintv{\result}{\xa}{\ta}{\xb}{\tb}
            \FPround{\result}{\result}{0}
            \begin{align*}
              \c^2 \, {(\mdelta{t})}^2 - {(\mdelta{x})}^2
                &= \c^2 \, {(t_2 - t_1)}^2 - {(x_2 - x_1)}^2\\
                &= {(t_2 - t_1)}^2 - {(x_2 - x_1)}^2 \qquad \c = \sol{}\\
                &= {\left(\tb - \left(\ta\right)\right)}^2 - {\left(\xb - \left(\xa\right)\right)}^2\\
                &= \result < 0 \rightarrow \text{spacelike interval} \cmark
            \end{align*}
          \end{samepage}

        \FPeval{xa}{1}
        \FPeval{ta}{3}
        \FPeval{xb}{3}
        \FPneg{\tb}{3}
        \FPround{\tb}{\tb}{0}
        \item $x_1 = \xa$, $t_1 = \ta$ and $x_2 = \xb$, $t_2 = \tb$.
          \begin{samepage}
            \spacetimeintv{\result}{\xa}{\ta}{\xb}{\tb}
            \FPround{\result}{\result}{0}
            \begin{align*}
              \c^2 \, {(\mdelta{t})}^2 - {(\mdelta{x})}^2
                &= \c^2 \, {(t_2 - t_1)}^2 - {(x_2 - x_1)}^2\\
                &= {(t_2 - t_1)}^2 - {(x_2 - x_1)}^2 \qquad \c = \sol{}\\
                &= {\left(\tb - \ta\right)}^2 - {\left(\xb - \xa\right)}^2\\
                &= \result > 0 \rightarrow \text{timelike interval} \cmark
            \end{align*}
          \end{samepage}

        \FPeval{xa}{2}
        \FPeval{ta}{5}
        \FPeval{xb}{5}
        \FPeval{tb}{8}
        \item $x_1 = \xa$, $t_1 = \ta$ and $x_2 = \xb$, $t_2 = \tb$.
          \begin{samepage}
            \spacetimeintv{\result}{\xa}{\ta}{\xb}{\tb}
            \FPround{\result}{\result}{0}
            \begin{align*}
              \c^2 \, {(\mdelta{t})}^2 - {(\mdelta{x})}^2
                &= \c^2 \, {(t_2 - t_1)}^2 - {(x_2 - x_1)}^2\\
                &= {(t_2 - t_1)}^2 - {(x_2 - x_1)}^2 \qquad \c = \sol{}\\
                &= {\left(\tb - \ta\right)}^2 - {\left(\xb - \xa\right)}^2\\
                &= \result \rightarrow \text{lightlike interval} \cmark
            \end{align*}
          \end{samepage}
      \end{enumerate}

    \clearpage
    \section{Problem 3: Cause and effect, or vice-versa?}
      The good guys travel away from the bad guys’ planet at \SI{0.8}{\c},
      and it takes the bad guys 5~years to invent and launch a faster-than-light
      ship that could travel at \SI{4}{\c} to catch and attack the good guys.
      \setsol{1}
      \FPeval{goodspeed}{0.8}
      \lorentz{\gammav}{\goodspeed}
      \FPround{\gammav}{\gammav}{2}
      \begin{enumerate}
        \item
          In terms of the frame of reference of the bad guys’ planet, when would
          the bad guys catch up to the good guys (assuming that $t_{[B]} = 0$ is
          when the good guys left the planet)?  Tip: Note that the expression for
          the length covered by the bad guys is $(t_{[B]} - 5)(4\,\c)$, where
          $t_{[B]}$ is greater than 5 (i.e., after the launch of the bad guys’
          super ship at $t_{[B]} = 5$ years). Set this expression equal to
          the expression for the length covered by the good guys in a given time
          $t_{[B]}$, and then solve for $t_{[B]}$.
          \FPeval{badspeed}{4}
          \FPeval{x}{badspeed - goodspeed}
          \FPround{\x}{\x}{1}
          \FPeval{badwait}{5}
          \FPeval{y}{badwait * badspeed}
          \FPround{\y}{\y}{0}
          \FPeval{resulta}{y/x}
          \FPround{\resulta}{\resulta}{2}
          \begin{align*}
            (t_B - \badwait) \times \SI{\badspeed}{\c} &= \SI{\goodspeed}{\c} \times t_B\\
            \badspeed \, t_B - \y &= \goodspeed \, t_B \qquad \c = 1\\
            \x \, t_B &= \y{}\\
            t_B &= \resulta{} \cmark\\
          \end{align*}

        \item
          \begin{samepage}
            According to the good guys’ clocks (in their frame of reference),
            when do the bad guys catch up to them?
            \FPeval{badtravel}{resulta - badwait}
            \FPround{\badtravel}{\badtravel}{2}
            \FPeval{xB}{badspeed * badtravel}
            \FPround{\xB}{\xB}{0}
            \FPeval{resultb}{(resulta - goodspeed * xB)/(gammav * (1 + goodspeed^2))}
            \FPround{\resultb}{\resultb}{3}
            \begin{align*}
              x_B &= \gamma \, \left(x_G + v \, t_G\right)\\
              x_G &= \frac{x_B}{\gamma} - v \, t_G\\
              t_B &= \gamma \, \left(t_G + \frac{v}{\c^2} \, x_G\right)\\
              t_G
                &= \frac{t_B}{\gamma} - \frac{v}{\c^2} \, x_G\\
                &= \frac{t_B}{\gamma} - \frac{v}{\c^2} \, \left(\frac{x_B}{\gamma} - v \, t_G\right)\\
              t_G &= \frac{t_B}{\gamma} - \frac{v \, x_B}{\gamma \, \c^2} - \frac{v^2}{\c^2} \, t_G\\
              \left(1 + \frac{v^2}{\c^2}\right) \, t_G &= \frac{1}{\gamma} \left(t_B - \frac{v}{\c^2} \, x_B\right)\\
              t_G &= \frac{t_B - \frac{v}{\c^2} \, x_B}{\gamma \left(1 + \frac{v^2}{\c^2}\right)}\\
              t_G(t_B = \resulta, \, x_B = \SI{\badspeed}{\c} \times \badtravel, \, v = \SI{\goodspeed}{\c}, \, \c = 1)
                &= \frac{\resulta - \goodspeed \times \xB}{\gammav \times \left(1 + {(\goodspeed)}^2\right)}
                = \resultb{} \qquad \text{\strong{wrong}}\\
            \end{align*}
            \emph{Reason: Wrong sign on multiplication (see Problem 1a)}

            \emph{Correct approach (see lecture notes):}

            \begin{enumerate}[1.]
              \item Obtain ${(t_{attack})}_B$.
              \item Obtain ${(x_{attack})}_B$ from ${(t_{attack})}_B$ and $v$.
              \item From direct Lorentz transformation (note: inverted direction),
                obtain ${(t_{attack})}_G$.
            \end{enumerate}

            \emph{Second attempt (using lecture notes only):}

            \FPeval{xB}{resulta * goodspeed}
            \FPround{\xB}{\xB}{0}
            \FPeval{resultb}{gammav * (resulta - goodspeed * \xB)}
            \FPround{\resultb}{\resultb}{4}
            \begin{align*}
              {(t_{attack})}_B &= 6.25 \qquad \text{from (a)}\\
              {(x_{attack})}_B &= {(t_{attack})}_B \times v = \resulta \times \goodspeed = \xB{}\\
              {(t_{attack})}_G
                &= \gamma \, \left({(t_{attack})}_B - \frac{v}{\c^2} \, {(x_{attack})}_B\right)\\
                &= \gammav \times \left(\resulta - \goodspeed \times \xB\right)\\
                &= \resultb \cmark
            \end{align*}
          \end{samepage}
          \emph{Could also have been solved using time dilation or
            length contraction directly (problem set solutions).}

        \item
          According to the good guys’ clocks (in their frame of reference),
          when do the bad guys launch their super ship and set off after them?
          \FPeval{xB}{0}
          \FPeval{resultc}{gammav * (5 - goodspeed * \xB)}
          \FPround{\resultc}{\resultc}{2}
          \begin{align*}
            {(t_{launch})}_B &= 5 \qquad \text{from (a)}\\
            {(x_{launch})}_B &= {(t_{launch})}_B \times v = \resulta \times \goodspeed = \xB{}\\
            {(t_{launch})}_G
              &= \gamma \, \left({(t_{launch})}_B - \frac{v}{\c^2} \, {(x_{launch})}_B\right)\\
              &= \gammav \times \left(5 - \goodspeed \times \xB\right)\\
              &= \resultc \cmark
          \end{align*}
      \end{enumerate}

    \section{Challenge Problem: Faster than light?}


    \section{Newsgroup question: Do approaching clocks go faster?}
      \begin{spacetimediagram}[Test]
        \plotlight{}
      \end{spacetimediagram}

  \chapter{Week 7: Paradoxes to Ponder}
    \section{Problem 1: distance contraction}
      Star Fleet regulations: Maintain distance of at least $[D_H =]$ \SI{20}{\kilo\meter} at all times
      as measured by an observer at Star Fleet Headquarters.
      \begin{enumerate}[(a)]
        \setsol{1}
        \FPeval{distancespaceship}{25}
        \FPeval{distancehq}{20}
        \FPeval{gammav}{distancespaceship/distancehq}
        \lorentztospeedr[1]{\resulta}{gammav}

        \item Maximum allowed speed for distance $D_S = \SI{25}{\kilo\meter}$
          in spaceships' rest frame:
          \FPevalr[1]{\gammareci}{1/gammav}
          \FPevalr[2]{\gammarecisquared}{gammareci^2}
          \FPevalr[2]{\radicand}{1 - gammarecisquared}
          \begin{align*}
            v &= \c \, \sqrt{1 - \frac{1}{\gamma^2}} \qquad \text{(see~\ref{sec:lorentztospeed})}\\
              &= \c \, \sqrt{1 - \frac{1}{{\left(\frac{D_S}{D_H}\right)}^2}}\\
              &= \c \, \sqrt{1 - \frac{1}{\frac{{\left(D_S\right)}^2}{{\left(D_H\right)}^2}}}\\
              &= \c \, \sqrt{1 - \frac{{\left(D_H\right)}^2}{{\left(D_S\right)}^2}}\\
              &= \c \, \sqrt{1 - {\left(\frac{D_H}{D_S}\right)}^2}\\
              &= \c \, \sqrt{1 - {\left(\frac{\SI{\distancehq}{\kilo\meter}}{\SI{\distancespaceship}{\kilo\meter}}\right)}^2}\\
              &= \c \, \sqrt{1 - {\left(\gammareci\right)}^2}\\
              &= \c \, \sqrt{1 - \gammarecisquared}\\
              &= \c \, \sqrt{\radicand}\\
              &= \SI{\resulta}{\c}
          \end{align*}

        \item Time difference between the clocks at the ends of the distance test:
          \FPevalr[2]{\speed}{1.05 * resulta}
          \lorentzr{\gammav}{\speed}
          \setsol{solSI / 1000}
          \FPround{\sol}{\sol}{3}
          \FPevalr[3]{\resultb}{(distancespaceship * speed)/(gammav * sol) * 10^6}
          \begin{align*}
            {(\mdelta{t})}_{S} &= \frac{D_H \, v}{\c^2}\\
            v &= 1.05 \times \resulta = \SI{\speed}{\c}\\
            D_S &= \SI{\distancespaceship}{\kilo\meter}\\
            D_H &= \frac{D_S}{\gamma(v)}\\
            {(\mdelta{t})}_{S}
              &= \frac{D_S \, v}{\gamma(v) \, \c^2}\\
              &= \frac{\SI{\distancespaceship}{\kilo\meter} \times \SI{\speed}{\c}}{\gamma(v) \, \c^2}\\
              &\approx \frac{\SI{\distancespaceship}{\kilo\meter} \times \speed}{\SI{\gammav}{\c}}\\
              &= \frac{\SI{\distancespaceship}{\kilo\meter} \times \speed}{\gammav \times \SI{\sol}{\kilo\meter\per\second}}\\
              &\approx \SI{\resultb}{\micro\second}
          \end{align*}

        \item Distance of the rear spaceship from the near (left) end of the
          distance test when the observer takes the photograph of the lead spaceship,
          according to the pilots' frame of reference:
          \begin{align*}
            {(\mdelta{x})}_{H} &= v \, {(\mdelta{t})}_S\\
            {(\mdelta{x})}_{S} &= \frac{{(\mdelta{x})}_H}{\gamma(v)}\\
              &= \frac{v \, {(\mdelta{t})}_S}{\gamma(v)}\\
              &= \frac{\SI{0.63}{\c} \times \SI{\resultb}{\micro\second}}{\gammav}\\
          \end{align*}
      \end{enumerate}



  % % References
  % \begingroup
  % %     \raggedright
  %     %\sloppy
  %   \printbibliography[heading=lit]
  % \endgroup
\end{document}
