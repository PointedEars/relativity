% vim:set fileencoding=utf-8 tabstop=2 shiftwidth=2 softtabstop=2 expandtab:
%% Page layout
% Scientific report, European style (A4) (from KOMA-Script)
% BCOR: binding correction
% DIV=calc: Calculate page spread now (recalculated below)
% pagesize: Add page info to (PDF/PS) output
% parskip=half: Do not indent paragraphs, add one half line spacing instead
%   (FFHS requirement)
% Default font size: 10pt (FFHS requirement)
% BCOR=1cm too much?
\documentclass[pagesize,headsepline,10pt,parskip=half]{scrreprt}

% Line spacing should be 1.5 times the line height (FFHS requirement)
\usepackage{setspace}
\onehalfspacing{}

% float compatibility for KOMA-Script
\usepackage{scrhack}

% Use English orthography and hyphenation by default
\usepackage[latin,italian,english]{babel}

% Allow using non-ASCII characters verbatim; disable for LuaLaTeX
% \usepackage[utf8]{inputenc}

% Special characters
\usepackage{textcomp}

% Use fonts having non-ASCII characters; disable fontenc for LuaLaTeX
% \usepackage[T1]{fontenc}
\usepackage{fontspec}
\newfontfamily\DejaSans{DejaVu Sans}

% User-defined English hyphenation
\hyphenation{InfoWorld}

% Use language-specific quotes
\usepackage[autostyle,english=american]{csquotes}
\DeclareQuoteAlias{italian}{latin}

% Advanced Computer Modern fonts
\usepackage{lmodern}

% Allow strike-through with \sout, keep italic for \emph
\usepackage[normalem]{ulem}

% Use medieval numbers except in math mode
% \usepackage{hfoldsty}

% Use sans-serif font ('Arial') by default for headings and normal text;
% disable helvet for LuaLaTeX
% (FFHS requirement)
\renewcommand{\familydefault}{\sfdefault}
\usepackage{mathptmx}
%\usepackage[scaled=.90]{helvet}
\usepackage{courier}

% Improved typography, like hyphenation in words with non-ASCII characters,
% see also http://homepage.ruhr-uni-bochum.de/georg.verweyen/pakete.html
\usepackage[babel]{microtype}

% Number also \subsubsection, but not \paragraph and below
\setcounter{secnumdepth}{3}

% Page heading and footer
\usepackage{scrpage2}
\pagestyle{scrheadings}
\automark[section]{chapter}
% heading on the top inner margin only
\ohead[]{\headmark}
\chead[]{}
% page number on the bottom outer margin only
\ofoot[\pagemark]{\pagemark}
\cfoot[]{}

% Support for list of acronyms
\usepackage[footnote,nohyperlinks,withpage]{acronym}

% References: can use section names
\usepackage{nameref}

% References: generate hyperlinks
\usepackage[plainpages=false]{hyperref}

% References style (default: 'numerical')
\usepackage[backend=biber,
style=authoryear-ibid,
maxcitenames=1,
maxbibnames=3]{biblatex}
\defbibheading{lit}{\chapter*{References}\markboth{References}{References}}
% References: bibliography database
\addbibresource{main.bib}

% prints author names as small caps
\renewcommand{\mkbibnamefirst}[1]{\textsc{#1}}
\renewcommand{\mkbibnamelast}[1]{\textsc{#1}}
\renewcommand{\mkbibnameprefix}[1]{\textsc{#1}}
\renewcommand{\mkbibnameaffix}[1]{\textsc{#1}}

% References: use English ordinal numbers
\usepackage[super]{nth}

% Automatically use teletype for \url argument, use content verbatim
\usepackage{url}
\urlstyle{tt}

% Less vertical spacing between list items (in `compactitem' environment)
\usepackage{paralist}

% Multi-row table cells
\usepackage{multirow}

% Support for horizontal rules in tables (professional style)
\usepackage{booktabs}

% Footnotes in tables
\usepackage{threeparttable}

% Tables across pages (for results)
\usepackage{longtable}

% Word wrap in table columns (calculate p width)
\usepackage{calc}

% Automatic column stretching
% \usepackage{tabularx}

% Stretched tables across pages (for results); requires longtable
% tabularx
\usepackage{ltxtable}

% Improved table formatting
\usepackage{array}

% Support for including PDFs
\usepackage{pdfpages}

% Support for figures
\usepackage{graphicx}

\usepackage{amsthm}
\newtheorem{mydef}{Definition}

\usepackage[fleqn]{amsmath}
\newlength{\normalparindent}
\AtBeginDocument{\setlength{\normalparindent}{\parindent}}
\newcommand{\longintertext}[1]{%
  \intertext{%
    \parbox{\linewidth}{%
      \setlength{\parindent}{\normalparindent}
      \noindent#1%
    }%
  }%
}

\usepackage{esint}
\usepackage{amssymb}
% \usepackage{commath}
\allowdisplaybreaks{}

% degree symbol etc.
\usepackage{gensymb}

% dingbats
\usepackage{pifont}
\newcommand{\cmark}{\ding{51}}

% SI units
\usepackage{siunitx}
\sisetup{per-mode = fraction, math-micro = \text{µ}, text-micro = µ}
% Margin notes
\usepackage{marginnote}

% FSM graphs
\usepackage{tikz}
\usetikzlibrary{arrows,automata}

% Captions
\usepackage{caption}

% Recalculate page spread based on the definitions above
\recalctypearea{}

%% User commands
% Formatting and languages
\newcommand{\strong}[1]{\textbf{#1}}
\newcommand{\code}[1]{\texttt{#1}}
\newcommand{\var}[1]{\textit{#1}}
\newcommand{\en}[1]{\foreignlanguage{english}{#1}}

% Abbreviations
\usepackage{xspace}
\newcommand{\ao}{\mbox{u.\,a.}\xspace}
\newcommand{\cf}{\mbox{vgl.}\xspace}
\newcommand{\ie}{\mbox{d.\,h.}\xspace}
\newcommand{\eg}{\mbox{e.g.}\xspace}
\newcommand{\ital}{\mbox{ital.}\xspace}

% Common terms
\newcommand{\sectionname}{Section}
\newcommand{\eq}{equation\xspace}

% Commands for common math expressions
\newcommand{\abs}[1]{\lvert#1\rvert}
\renewcommand\d[1]{\:\textrm{d}#1}
\newcommand*\diff{\mathop{}\!\mathrm{d}}
\newcommand*\mdelta[1]{\ensuremath{\mathrm{\Delta\,}#1}}
\renewcommand{\qedsymbol}{\ensuremath{\blacksquare}}

% Chemistry
\newcommand*\chem[1]{\ensuremath{\mathrm{#1}}}

% alignment in \cases
\makeatletter
\renewcommand{\env@cases}[1][@{}l@{\quad}l@{}]{%
 \let\@ifnextchar\new@ifnextchar{}
 \left\lbrace{}
 \def\arraystretch{1.2}%
 \array{#1}%
}
\makeatother

% asides
\usepackage{mdframed}
\newenvironment{aside}
{\begin{mdframed}[style=0,%
  leftline=false,rightline=false,leftmargin=2em,rightmargin=2em,%
  innerleftmargin=0pt,innerrightmargin=0pt,linewidth=0.75pt,%
  skipabove=7pt,skipbelow=7pt]\small}
{\end{mdframed}}

%% Shortcuts
%%% arrays (vectors, matrixes, tensors)
\newcommand{\vecb}[1]{\mathbf{#1}}
\newcommand{\parray}[2]{\left(\begin{array}{#1}#2\end{array}\right)}
%% Vector norm
\newcommand{\norm}[1]{\left\|{#1}\right\|}
%% Constants
\newcommand{\const}[1]{\mathrm{#1}}
% Speed of light
\renewcommand{\c}{\const{c}}

% Calculate numbers
\usepackage{fp}

% Use programming in commands
% \usepackage{etoolbox}

%% Relativistic calculations
% Define command #1 as Lorentz factor for a speed #2 in multiples of speed of light
\newcommand{\lorentz}[2]{\FPeval{#1}{1/root(2, 1 - #2^2)}}

% Speed of light
\FPeval{\sol}{299792458}

% Define command #1 as relative speed for a Lorentz factor #2
\newcommand{\lorentztospeed}[2]{\FPeval{#1}{sol * root(2, 1 - 1/(#2^2))}}

% Define command #1 as length contraction of length #2
% for a speed #3 in multiples of speed of light
\newcommand{\lencon}[3]{\FPeval{#1}{#2 * root(2, 1 - #3^2)}}

% Define command #1 as time dilation of time #2
% for a speed #3 in multiples of speed of light
\newcommand{\timedil}[3]{\FPeval{#1}{1/root(2, 1 - #3^2) * #2}}

\begin{document}
  % User-defined language-specific hyphenation
  %  \hyphenation{bezüg-lich einer Da-ten-bank-ope-ra-ti-onen
  %  effizienz-stei-gernd ECMA ECMAScript Firefox Google JavaScript JavaScript-Core
  %  Kom-pa-ti-bi-li-täts-matrix lauf-fähig lenny Linux MySQL proto-typ-ba-sier-ter
  %  robusteren SquirrelFish Wine}

  \begin{titlepage}
    \title{\href{https://www.coursera.org/learn/einstein-relativity/}{Understanding Einstein:\\The Special Theory of Relativity}}
    \subtitle{Own solutions for the problem sets from the Stanford University Online Course}
    \author{Larry Randles Lagerstrom, Instructor\\\href{http://PointedEars.de/}{Thomas Lahn}, Student}
    \maketitle
  \end{titlepage}

  \clearpage
  \pagenumbering{Roman}
  \begin{spacing}{1}
    % Print TOC
    \tableofcontents
    \thispagestyle{empty}
  \end{spacing}

  %   \clearpage
  %   \begin{spacing}{1}
  %     \chapter*{List of acronyms} \label{chapter:acronyms}
  %     \begin{acronym}[]
  %       \setlength{\itemsep}{-\parsep}
  %        \acro{RFC}{Request for Comments (Internet-Standard)}
  %     \end{acronym}
  %   \end{spacing}

  \clearpage
  \pagenumbering{arabic}
  \chapter{Week 4}
    \section{Problem 1: Velocity → Lorentz factor}
      Calculate the value of the Lorentz factor for the following velocities.
      \newcommand{\lorentzr}[2]{
        \lorentz{#1}{#2}
        \FPround{#1}{#1}{5}
      }
      \FPeval{velocitya}{0.01}
      \lorentzr{\resulta}{\velocitya}
      \FPeval{velocityb}{0.1}
      \lorentzr{\resultb}{\velocityb}
      \FPeval{velocityc}{0.25}
      \lorentzr{\resultc}{\velocityc}
      \FPeval{velocityd}{0.5}
      \lorentzr{\resultd}{\velocityd}
      \FPeval{velocitye}{0.75}
      \lorentzr{\resulte}{\velocitye}
      \FPeval{velocityf}{0.9}
      \lorentzr{\resultf}{\velocityf}
      \FPeval{velocityg}{0.99}
      \lorentzr{\resultg}{\velocityg}
      \FPeval{velocityh}{0.999}
      \lorentzr{\resulth}{\velocityh}
      \begin{align*}
        \gamma &= \frac{1}{\sqrt{1 - {\left(\frac{v}{c}\right)}^2}}\\
        \text{(a) } \gamma(v = \SI{\velocitya}{\c}) &= \frac{1}{\sqrt{1 - {\velocitya}^2}}\\
        &= \frac{1}{\sqrt{1 - 0.0001}}\\
        &= \frac{1}{\sqrt{0.9999}}\\
        &= \frac{1}{\sqrt{\frac{9999}{10000}}}\\
        &= \frac{1}{\frac{99.995}{100}}\\
        &= \frac{100}{99.995}\\
        &= \resulta\\
        \text{(b) } \gamma(v = \SI{\velocityb}{\c}) &\approx \resultb\\
        \text{(c) } \gamma(v = \SI{\velocityc}{\c}) &\approx \resultc\\
        \text{(d) } \gamma(v = \SI{\velocityd}{\c}) &\approx \resultd\\
        \text{(e) } \gamma(v = \SI{\velocitye}{\c}) &\approx \resulte\\
        \text{(f) } \gamma(v = \SI{\velocityf}{\c}) &\approx \resultf\\
        \text{(g) } \gamma(v = \SI{\velocityg}{\c}) &\approx \resultg\\
        \text{(h) } \gamma(v = \SI{\velocityh}{\c}) &\approx \resulth
      \end{align*}

    \section{Problem 2: Lorentz factor →~velocity}
      Calculate the value of the relative velocity in km/second between two frames of reference
      for the following values of the Lorentz factor. (Use 300,000 km/second for the speed of light.)
      \newcommand{\lorentztospeedr}[2]{
        \lorentztospeed{#1}{#2}
        \FPround{#1}{#1}{3}
      }
      \FPeval{\sol}{300000}
      \FPeval{gammaa}{1.1}
      \lorentztospeedr{\resulta}{\gammaa}
      \FPeval{gammab}{1.25}
      \lorentztospeedr{\resultb}{\gammab}
      \FPeval{gammac}{1.5}
      \lorentztospeedr{\resultc}{\gammac}
      \FPeval{gammad}{2}
      \lorentztospeedr{\resultd}{\gammad}
      \FPeval{gammae}{5}
      \lorentztospeedr{\resulte}{\gammae}
      \FPeval{gammaf}{10}
      \lorentztospeedr{\resultf}{\gammaf}
      \begin{align*}
        \gamma &= \frac{1}{\sqrt{1 - \frac{v^2}{c^2}}}\\
         \sqrt{1 - \frac{v^2}{c^2}} &= \frac{1}{\gamma}\\
         1 - \frac{v^2}{c^2} &= \frac{1}{\gamma^2}\\
         - \frac{v^2}{c^2} &= \frac{1}{\gamma^2} - 1\\
         \frac{v^2}{c^2} &= 1 - \frac{1}{\gamma^2}\\
         v^2 &= c^2 \, \left(1 - \frac{1}{\gamma^2}\right)\\
         v &= \sqrt{c^2 \, \left(1 - \frac{1}{\gamma^2}\right)}\\
         v &= c \, \sqrt{1 - \frac{1}{\gamma^2}}\\
         \text{(a) } v(\gamma = \gammaa) &\approx \SI{\resulta}{\kilo\meter\per\second}\\
         \text{(b) } v(\gamma = \gammab) &\approx \SI{\resultb}{\kilo\meter\per\second}\\
         \text{(c) } v(\gamma = \gammac) &\approx \SI{\resultc}{\kilo\meter\per\second}\\
         \text{(d) } v(\gamma = \gammad) &\approx \SI{\resultd}{\kilo\meter\per\second}\\
         \text{(e) } v(\gamma = \gammae) &\approx \SI{\resulte}{\kilo\meter\per\second}\\
         \text{(f) } v(\gamma = \gammaf) &\approx \SI{\resultf}{\kilo\meter\per\second}
      \end{align*}

    \section{Problem 3: Length contraction}
      Alice and Bob have identical spaceships, each with a proper length of [$L_0 = $] 50 meters. Bob docks
      his ship at the spaceport while Alice flies by him to the right at a speed of [$v =$]~\SI{0.6}{\c}.
      \begin{enumerate}[(a)]
        \item As Alice flies by, Bob measures the length of Alice’s ship. What value does he get?
          \FPeval{shiplen}{50}
          \FPeval{valuea}{0.6}
          \lencon{\resulta}{\shiplen}{\valuea}
          \FPround{\resulta}{\resulta}{0}
          \begin{align*}
            {L_{Bob}}_{Alice} &= \frac{L_0}{\gamma} = L_0 \, \sqrt{1 - \frac{v^2}{c^2}}\\
            {L_{Bob}}_{Alice}\left(v = \SI{0.6}{\c}\right) &= \SI{\resulta}{\meter}.
          \end{align*}
        \item At the same time, Alice measures the length of Bob’s ship. What value does she get?
          \begin{align*}
            {L_{Alice}}_{Bob}\left(v = \SI{0.6}{\c}\right) &= {L_{Bob}}_{Alice}\left(v = \SI{0.6}{\c}\right) = \SI{\resulta}{\meter}.
          \end{align*}
        \item Alice flies by a second time and this time Bob measures the length of Alice’s ship
          and gets a value of [${L_{Bob}}_{Alice} = $] 20 meters. How fast is Alice going with respect to Bob?
          \FPeval{valuec}{20}
          \FPeval{gammac}{round(shiplen/valuec :1)}
          \lorentztospeedr{\resultc}{\gammac}
          \FPeval{resultcinc}{round(resultc / sol :3)}
          \begin{align*}
            v &= c \, \sqrt{1 - \frac{1}{\gamma^2}}\\
            {L_{Bob}}_{Alice} &= \frac{L_0}{\gamma} \rightarrow~\gamma = \frac{L_0}{{L_{Bob}}_{Alice}} = \frac{\SI{\shiplen}{\meter}}{\SI{\valuec}{\meter}} = \gammac\\
            v(\gamma = \gammac) &= \SI{\resultc}{\kilo\meter\per\second} \approx \SI{\resultcinc}{\c}.
          \end{align*}
      \end{enumerate}

    \section{Problem 4: Time dilation}
      Alice and Bob have identical spaceships, each equipped with identical
      light clocks. Bob docks his ship at the spaceport while Alice flies by
      him to the right at a speed of [$v = $]~\SI{0.6}{c}.
      \FPeval{velocitya}{0.6}
      \begin{enumerate}[(a)]
        \item
          With his light clock next to him, Bob measures the length of one tick
          of the clock to be [$\mdelta{t_{Bob}} =$] \SI{0.001}{seconds}.
          (This value isn’t necessarily very realistic, i.e., it would be
          much shorter, on the order of a nanosecond.  But use the value given
          as it makes the numbers a little easier to work with.)

          As Alice flies by, he compares one tick of his clock to one tick
          of her clock. According to his observation, what is the length of
          her clock tick?
          \FPeval{timea}{0.001}
          \timedil{\resulta}{\timea}{\velocitya}
          \FPround{\resulta}{\resulta}{5}
          \begin{align*}
            \mdelta{{t_{Bob}}_{Alice}} &= \gamma(v) \, \mdelta{t_{Bob}} = \SI{\resulta}{\second}.
          \end{align*}
        \item
          Bob observes that, according to his clock (or really, his lattice of
          synchronized clocks), Alice’s ship takes [$\mdelta{{T_{Bob}}Alice} =$]
          60 minutes to travel from his location to a distant space station.

          When Alice passes that space station, Bob records the time on
          his clock (located at the space station, i.e., it’s part of his lattice
          of synchronized clocks) and the time on Alice’s clock by taking a photo
          of them.

          According to Bob’s photo, how much time [$\mdelta{T_{Alice}}$]
          has elapsed on Alice’s clock between the time she passed him and
          the time she passes the space station?
          \FPeval{timeb}{60}
          \FPeval{resultb}{\timeb * root(2, 1 - \velocitya^2)}
          \FPround{\resultb}{\resultb}{0}
          \begin{align*}
            \mdelta{T_{Alice}} &= \frac{\mdelta{{T_{Bob}}_{Alice}}}{\gamma(v)} = \SI{\resultb}{\minute}.
          \end{align*}
        \item
          \begin{samepage}
            How does Alice, in her frame of reference, see and explain
            the distances and times involved?

            \begin{itemize}
              \item For Alice, the \strong{distance} to the space station is
                length-contracted, so she needs less time to travel to it
                than Bob measures.
              \item \strong{Time} is running normally for Alice in her frame of reference, so
                she has to conclude that Bob's clocks are not synchronized:
                his leading clock (the one at the space station) lags.
            \end{itemize}
          \end{samepage}
      \end{enumerate}

  \section{Problem 5: Relativistic particles}
    Imagine that you are a physicist studying particles created by cosmic rays
    colliding with atoms in the upper atmosphere at a height of [$h = $]~10 kilometers
    above the surface of the Earth. Using special equipment attached to a large
    weather balloon, you discover that a certain kind of particle is detected
    as low as [$ h_{min} =$]~1500 meters above the surface, but no lower.
    In other words, it is created at a height of 10 kilometers and travels down
    to a height of 1.5 kilometers, where it decays into other particles. You
    also measure the speed of the original particle and get a value of
    [$v =$]~\SI{0.9}{\c}.

    Given this information, what is the lifetime [$\tau_0$] of the particle
    (in seconds) in its own frame of reference (i.e., when it is at rest, not
    in motion)?  (The lifetime is the amount of time it exists between being
    created and decaying into other particles. Also note: The numbers given here
    do not correspond to any actual particle, but are just for the exercise.)
    \FPeval{height}{10}
    \FPeval{minheight}{1.5}
    \FPeval{speed}{0.9}
    \lorentz{\gammas}{\speed}
    \FPeval{taume}{(height - minheight)/(speed * sol)}
    \FPeval{result}{(height - minheight)/(speed * sol * gammas)}
    \FPround{\gammas}{\gammas}{5}
    \FPround{\taume}{\taume}{7}
    \FPround{\result}{\result}{7}
    \begin{align*}
      h &= \SI{10}{\kilo\meter}\
        \qquad h_{min} = \SI{1500}{\meter}\
        \qquad v = \SI{0.9}{\c}\\
      \mdelta{h} &= h - h_{min} = \SI{8.5}{\kilo\meter}\\
      \tau_{me} &= \frac{\mdelta{h}}{v} \approx \SI{\taume}{\second}\\
      \tau_0 &= \frac{\tau_{me}}{\gamma(v)} = \frac{\mdelta{h}}{v \, \gamma(v)}\\
        &\approx \frac{\SI{\height}{\kilo\meter} - \SI{\minheight}{\kilo\meter}}{\speed \times \SI{\sol}{\kilo\meter\per\second} \times \gammas}\
        \approx \SI{\result}{\second}.
    \end{align*}

  % % References
  % \begingroup
  % %     \raggedright
  %     %\sloppy
  %   \printbibliography[heading=lit]
  % \endgroup
\end{document}
