% vim:set fileencoding=utf-8 tabstop=2 shiftwidth=2 softtabstop=2 expandtab:
%% Page layout
% Scientific report, European style (A4) (from KOMA-Script)
% BCOR: binding correction
% DIV=calc: Calculate page spread now (recalculated below)
% pagesize: Add page info to (PDF/PS) output
% parskip=half: Do not indent paragraphs, add one half line spacing instead
%   (FFHS requirement)
% Default font size: 10pt (FFHS requirement)
% BCOR=1cm too much?
\documentclass[pagesize,headsepline,10pt,parskip=half]{scrreprt}

% Line spacing should be 1.5 times the line height (FFHS requirement)
\usepackage{setspace}
\onehalfspacing{}

% float compatibility for KOMA-Script
\usepackage{scrhack}

% Use new German orthography and hyphenation (the latter is FFHS requirement)
\usepackage[ngerman,english]{babel}
% \usepackage[ngerman=ngerman-x-latest]{hyphsubst}

% Allow using non-ASCII characters verbatim
\usepackage[utf8]{inputenc}

% Special characters
\usepackage{textcomp}

% Use fonts having non-ASCII characters
\usepackage[T1]{fontenc}
\usepackage{fontspec}
\newfontfamily\DejaSans{DejaVu Sans}

% User-defined English hyphenation
\hyphenation{InfoWorld}

% Use language-specific quotes
\usepackage[autostyle,german=swiss,english=american]{csquotes}

% Advanced Computer Modern fonts
\usepackage{lmodern}

% Allow strike-through with \sout, keep italic for \emph
\usepackage[normalem]{ulem}

% Use medieval numbers except in math mode
% \usepackage{hfoldsty}

% Use sans-serif font ('Arial') by default for headings and normal text
% (FFHS requirement)
\renewcommand{\familydefault}{\sfdefault}
\usepackage{mathptmx}
\usepackage[scaled=.90]{helvet}
\usepackage{courier}

% Improved typography, like hyphenation in words with non-ASCII characters,
% see also http://homepage.ruhr-uni-bochum.de/georg.verweyen/pakete.html
\usepackage[babel]{microtype}

% Number also \subsubsection, but not \paragraph and below
\setcounter{secnumdepth}{3}

% Page heading and footer
\usepackage{scrpage2}
\pagestyle{scrheadings}
\automark[section]{chapter}
% heading on the top inner margin only
\ohead[]{\headmark}
\chead[]{}
% page number on the bottom outer margin only
\ofoot[\pagemark]{\pagemark}
\cfoot[]{}

% Support for list of acronyms
\usepackage[footnote,nohyperlinks,withpage]{acronym}

% References: can use section names
\usepackage{nameref}

% References: generate hyperlinks
\usepackage[plainpages=false]{hyperref}

% References style (default: 'numerical')
\usepackage[style=authoryear-ibid,
maxcitenames=1,
maxbibnames=3]{biblatex}
\defbibheading{lit}{\chapter*{Literaturverzeichnis}\markboth{Literaturverzeichnis}{Literaturverzeichnis}}
% References: bibliography database
% \addbibresource{main.bib}

% prints author names as small caps
\renewcommand{\mkbibnamefirst}[1]{\textsc{#1}}
\renewcommand{\mkbibnamelast}[1]{\textsc{#1}}
\renewcommand{\mkbibnameprefix}[1]{\textsc{#1}}
\renewcommand{\mkbibnameaffix}[1]{\textsc{#1}}

% References: use English ordinal numbers
\usepackage[super]{nth}

% Automatically use teletype for \url argument, use content verbatim
\usepackage{url}
\urlstyle{tt}

% Less vertical spacing between list items (in `compactitem' environment)
\usepackage{paralist}

% Multi-row table cells
\usepackage{multirow}

% Support for horizontal rules in tables (professional style)
\usepackage{booktabs}

% Footnotes in tables
\usepackage{threeparttable}

% Tables across pages (for results)
\usepackage{longtable}

% Word wrap in table columns (calculate p width)
\usepackage{calc}

% Automatic column stretching
% \usepackage{tabularx}

% Stretched tables across pages (for results); requires longtable
% tabularx
\usepackage{ltxtable}

% Improved table formatting
\usepackage{array}

% Support for including PDFs
\usepackage{pdfpages}

% Support for figures
\usepackage{graphicx}

\usepackage{amsthm}
\newtheorem{mydef}{Definition}

\usepackage[fleqn]{amsmath}
\usepackage{esint}
% \usepackage{amssymb}
% \usepackage{commath}
\allowdisplaybreaks{}

% degree symbol etc.
\usepackage{gensymb}

% Margin notes
\usepackage{marginnote}

% FSM graphs
\usepackage{tikz}
\usetikzlibrary{arrows,automata}

% Captions
\usepackage{caption}

% Recalculate page spread based on the definitions above
\recalctypearea{}

%% User commands
% Formatting and languages
\newcommand{\code}[1]{\texttt{#1}}
\newcommand{\var}[1]{\textit{#1}}
\newcommand{\en}[1]{\foreignlanguage{english}{#1}}

% Abbreviations
\usepackage{xspace}
\newcommand{\ao}{\mbox{u.\,a.}\xspace}
\newcommand{\cf}{\mbox{vgl.}\xspace}
\newcommand{\ie}{\mbox{d.\,h.}\xspace}
\newcommand{\eg}{\mbox{e.g.}\xspace}

% Common terms
\newcommand{\sectionname}{Section}
\newcommand{\eq}{equation\xspace}

% Commands for common math expressions
\newcommand{\abs}[1]{\lvert#1\rvert}
\renewcommand\d[1]{\:\textrm{d}#1}
\newcommand*\diff{\mathop{}\!\mathrm{d}}

% alignment in \cases
\makeatletter
\renewcommand{\env@cases}[1][@{}l@{\quad}l@{}]{%
 \let\@ifnextchar\new@ifnextchar{}
 \left\lbrace{}
 \def\arraystretch{1.2}%
 \array{#1}%
}
\makeatother

% asides
\usepackage{mdframed}
\newenvironment{aside}
{\begin{mdframed}[style=0,%
  leftline=false,rightline=false,leftmargin=2em,rightmargin=2em,%
  innerleftmargin=0pt,innerrightmargin=0pt,linewidth=0.75pt,%
  skipabove=7pt,skipbelow=7pt]\small}
{\end{mdframed}}

%% Shortcuts
%%% arrays (vectors, matrixes, tensors)
\renewcommand{\vec}[1]{\mathbf{#1}}
\newcommand{\parray}[2]{\left(\begin{array}{#1}#2\end{array}\right)}
%% Vector norm
\newcommand{\norm}[1]{\left\|{#1}\right\|}
%% Constants
\newcommand{\const}[1]{\mathrm{#1}}
% Speed of light
\renewcommand{\c}{\mathrm{c}}

\begin{document}
  % User-defined language-specific hyphenation
  %  \hyphenation{bezüg-lich einer Da-ten-bank-ope-ra-ti-onen
  %  effizienz-stei-gernd ECMA ECMAScript Firefox Google JavaScript JavaScript-Core
  %  Kom-pa-ti-bi-li-täts-matrix lauf-fähig lenny Linux MySQL proto-typ-ba-sier-ter
  %  robusteren SquirrelFish Wine}

  \begin{titlepage}
    \title{Calculating the Time Dilation for Atmospheric Muons}
    \subtitle{based on special relativity}
    \author{Thomas Lahn}
    \maketitle
  \end{titlepage}

  \clearpage
  \pagenumbering{Roman}
  \begin{spacing}{1}
    % Print TOC
    \tableofcontents
    \thispagestyle{empty}
  \end{spacing}

  %   \clearpage
  %   \begin{spacing}{1}
  %     \chapter*{List of acronyms} \label{chapter:acronyms}
  %     \begin{acronym}[]
  %       \setlength{\itemsep}{-\parsep}
  %        \acro{RFC}{Request for Comments (Internet-Standard)}
  %     \end{acronym}
  %   \end{spacing}

  \clearpage
  \pagenumbering{arabic}
  \chapter{Four-vectors (spacetime vectors)}
    \section{Four-vector algebra}
      \begin{align*}
        \vec{E}_0 &= \parray{r}{1 \\ 0 \\ 0 \\ 0} \quad \vec{E}_1 = \parray{r}{0 \\ 1 \\ 0 \\ 0} \quad \vec{E}_2 = \parray{r}{0 \\ 0 \\ 1 \\ 0} \quad \vec{E}_3 = \parray{r}{0 \\ 0 \\ 0 \\ 1} \qquad \text{Bases of 4-space}\\
        \vec{A} & = \left(A^0, \, A^1, \, A^2, \, A^3\right) = \parray{c}{A^0 \\ A^1 \\ A^2 \\ A^3} \qquad A^i\text{ -- }\mathbf{Contravariant}\text{ components}\\
        & = A^0\vec{E}_0 + A^1 \vec{E}_1 + A^2 \vec{E}_2 + A^3  \vec{E}_3 \\
        & = A^0\vec{E}_0 + A^i \vec{E}_i \qquad \text{Einstein summation notation} \\
        & = A^\alpha\vec{E}_\alpha\\
        & = A^\mu \\
        \eta &= \parray{rrrr}{-1 & 0 & 0 & 0 \\ 0 & 1 & 0 & 0 \\ 0 & 0 & 1 & 0 \\ 0 & 0 & 0 & 1} \qquad \text{Minkowski metric tensor} \\
        A_\mu &= \eta_{\mu\nu} A^\mu \qquad \mathbf{Covariant}\text{ coordinates}
      \end{align*}

    \section{Four-position}
      \begin{align*}
        \vec{r} &= \left(x, \, y, \, z\right)  \qquad \text{3-position} \\
        \vec{R} &= \left(\c t, \, \vec{r}\right) = \left(\c t, \, x, \, y, \, z\right) \qquad \text{4-position (event)} \\
        \mathrm{\Delta}{\vec{R}} &= \left(\c \mathrm{\Delta} t, \, \mathrm{\Delta} \vec{r}\right) = \left(\c \mathrm{\Delta} t, \, \mathrm{\Delta} x, \, \mathrm{\Delta} y, \, \mathrm{\Delta} z\right) \qquad \text{4-displacement (spacetime distance)} \\
        \d{\vec{R}} &= \left(\c \d{t}, \, \d{\vec{r}}\right) = \left(\c \d{t}, \, \d{x}, \, \d{y}, \, \d{z}\right) \qquad \text{differential 4-displacement} \\
        {\norm{\d{\vec{R}}}}^2 &= \vec{d{R}} \cdot \vec{dR} \\
        &= \d{R}^\mu \d{R}_\mu = \d{R}^\mu \eta_{\mu\mu} \d{R}^\mu = \parray{cccc}{\c \d{t} & \d{x} &\d{y} & \d{z}} \parray{rrrr}{-1 & 0 & 0 & 0 \\ 0 & 1 & 0 & 0 \\ 0 & 0 & 1 & 0 \\ 0 & 0 & 0 & 1} \parray{c}{\c \d{t} \\ \d{x} \\ \d{y} \\ \d{z}} \\
        &= \c^2 {\d{\tau}}^2 = {\d{s}}^2 \qquad \d{s}\text{ -- differential line element; }\d{\tau}\text{ -- differential proper time}
      \end{align*}

    \section{Four-velocity}
      \begin{align*}
        \vec{u} &= \frac{\d{\vec{r}}}{\d{t}} \qquad \text{3-velocity} \\
        \gamma\left(\vec{u}\right) &= \frac{1}{\sqrt{1 {-} \frac{\vec{u} \cdot \vec{u}}{\c^2}}} \qquad \text{Lorentz factor} \\
        \vec{U} &= \frac{\d{\vec{R}}}{\d{\tau}} = \frac{\d{\vec{R}}}{\d{t}} \frac{\d{t}}{\d{\tau}} = \vec{\gamma}\left(\vec{u}\right) \left(\c, \, \vec{u}\right) \qquad \text{4-velocity}
      \end{align*}

    \section{Four-momentum}
      \begin{align*}
        \vec{p} & = \left(p_x, \, p_y, \, p_z\right) = \gamma\left(\vec{u}\right) m \vec{u} \qquad \text{3-momentum} \\
        E &= \gamma\left(\vec{u}\right) m \c^2 \qquad \text{total energy} \\
        \vec{P} & = m \vec{U} = m \gamma\left(\vec{u}\right)\left(\c, \, \vec{u}\right) = \parray{c}{m \gamma\left(\vec{u}\right) \c \\ m \gamma\left(\vec{u}\right) \vec{u}} = \parray{c}{\gamma\left(\vec{u}\right) m \c \\ \gamma\left(\vec{u}\right) m \vec{u}} = \parray{c}{\frac{E}{\c} \\ \vec{p}} \qquad \text{4-momentum}
      \end{align*}

  \chapter{Energy{--}momentum relation}
    \begin{align*}
      &\text{Squared Minkowski norm of the four-momentum:} \\
      \langle\vec P, \, \vec P \rangle &= {\norm{\vec{P}}}^2 \\
      &=  \\
      &= -{\left(m\c\right)}^2 & \text{(1)} \\
      &\text{and} \\
      \left\langle\vec{P}, \, \vec{P}\right\rangle &= \vec P^\alpha \vec \eta_{\alpha\beta} \vec P^\beta
      = \parray{cccc}{\frac{E}{\c} & p_x & p_y & p_z} \parray{rrrr}{-1 & 0 & 0 & 0 \\ 0 & 1 & 0 & 0 \\ 0 & 0 & 1 & 0 \\ 0 & 0 & 0 & 1} \parray{c}{\frac{E}{\c} \\ p_x \\ p_y \\ p_z} \\
      &= -\left(\frac{E}{\c}\right)^2 + {p_x}^2 + {p_y}^2 + {p_z}^2 \\
      &= -\left(\frac{E}{\c}\right)^2 + \norm{\vec{p}}^2 \\
      \left\langle\vec{P}, \, \vec{P} \right\rangle{} &= -\left(\frac{E}{\c}\right)^2 + p^2 & \text{(2)} \\
      &\text{in general:} \\
      p_i &= \frac{\partial{S}}{\partial{q_i}}, \; E = -\frac{\partial{S}}{\partial{t}} \\
      x^0 &= ct, \; x^1 = x, \; x^2 = y, \; x^3 = z \\
      x_0 &= -x_0, \; x_1 = x_1, \; x_2 = x_2, \; x_3 = x_3 \\
      p_\mu &= -\frac{\partial{S}}{\partial{x^\mu}} = \left(\frac{E}{c}, -\vec{p}\right) \\
      &\text{so [(1) = (2)]:} \\
      -{\left(m\c\right)}^2 &= -\left(\frac{E}{\c}\right)^2 + p^2 \\
      \left(\frac{E}{\c}\right)^2 &= {\left(m\c\right)}^2 + p^2 \\
      \frac{E^2}{\c^2} &= m^2 \c^2 + p^2 \\
      E^2 &= m^2 \c^4 + p^2 \c^2 \qquad \text{Energy{--}momentum relation}
    \end{align*}

    \begin{align*}
      \gamma & = \frac{1}{\sqrt{1 - \frac{v^2}{\c^2}}} = {\left(1 - \frac{v^2}{\c^2}\right)}^{-\frac{1}{2}} \qquad \text{Lorentz factor}\\
      p &= \gamma{}mv \qquad \text{Relativistic momentum} \\
      E^2 &= m^2 \c^4 + {\left(\gamma{}mv\right)}^2 \c^2 \\
          &= m^2 \c^4 + {\left(\frac{mv}{\sqrt{1 {-} \frac{v^2}{\c^2}}}\right)}^2 \c^2 \\
          &= m^2 \c^4 + \frac{m^2 v^2}{1 {-} \frac{v^2}{\c^2}} \c^2 \\
          &= m^2 \c^4 + \frac{m^2 v^2}{\frac{\c^2 {-} v^2}{\c^2}} \c^2 \\
          &= m^2 \c^4 + \frac{m^2 v^2 \c^2}{\c^2 {-} v^2} \c^2 \\
          &= m^2 \c^4 + \frac{m^2 v^2 \c^4}{\c^2 {-} v^2} \\
          &= \frac{m^2 \c^6 {-} m^2 v^2 \c^4 + m^2 v^2 \c^4}{\c^2 {-} v^2} \\
      E^2 &= \frac{m^2 \c^6}{\c^2 {-} v^2} \\
      \c^2 {-} v^2 &= \frac{m^2 \c^6}{E^2} \\
      v^2 &= \c^2 {-} \frac{m^2 \c^6}{E^2} \\
      \gamma    & = {\left(1 - \frac{\c^2 - \frac{m^2 \c^6}{E^2}}{\c^2}\right)}^{-\frac{1}{2}} \\
             & = {\left(1 - \left(\frac{\c^2}{\c^2} - \frac{\frac{m^2 \c^6}{E^2}}{\c^2}\right)\right)}^{-\frac{1}{2}} \\
             & = {\left(1 - \left(1 - \frac{m^2 \c^4}{E^2}\right)\right)}^{-\frac{1}{2}} \\
             & = {\left(\frac{m^2 \c^4}{E^2}\right)}^{-\frac{1}{2}} \\
             & = {\left(\frac{m \c^2}{E}\right)}^{-1} \\
      \gamma    & = \frac{E}{m \c^2} \quad (!) \\
             & \text{and that is equivalent to} \\
      E_{total} & = \gamma m \c^2 \qquad \text{Total (relativistic) energy {\DejaSans ☺}}
  \end{align*}
\end{document}
