% vim:set fileencoding=utf-8 tabstop=2 shiftwidth=2 softtabstop=2 expandtab:
%% Page layout
% Scientific report, European style (A4) (from KOMA-Script)
% BCOR: binding correction
% DIV=calc: Calculate page spread now (recalculated below)
% pagesize: Add page info to (PDF/PS) output
% parskip=half: Do not indent paragraphs, add one half line spacing instead
%   (FFHS requirement)
% Default font size: 10pt (FFHS requirement)
% BCOR=1cm too much?
\documentclass[pagesize,headsepline,10pt,parskip=half]{scrreprt}

% Line spacing should be 1.5 times the line height (FFHS requirement)
\usepackage{setspace}
\onehalfspacing

% float compatibility for KOMA-Script
\usepackage{scrhack}

% Use new German orthography and hyphenation (the latter is FFHS requirement)
\usepackage[ngerman,english]{babel}
% \usepackage[ngerman=ngerman-x-latest]{hyphsubst}

% Allow using non-ASCII characters verbatim
\usepackage[utf8]{inputenc}

% Special characters
\usepackage{textcomp}

% Use fonts having non-ASCII characters
\usepackage[T1]{fontenc}

% User-defined English hyphenation
\hyphenation{InfoWorld}

% Use language-specific quotes
\usepackage[autostyle,german=swiss,english=american]{csquotes}

% Advanced Computer Modern fonts
\usepackage{lmodern}

% Allow strike-through with \sout, keep italic for \emph
\usepackage[normalem]{ulem}

% Use medieval numbers except in math mode
% \usepackage{hfoldsty}

% Use sans-serif font ('Arial') by default for headings and normal text
% (FFHS requirement)
\renewcommand{\familydefault}{\sfdefault}
\usepackage{mathptmx}
\usepackage[scaled=.90]{helvet}
\usepackage{courier}

% Improved typography, like hyphenation in words with non-ASCII characters,
% see also http://homepage.ruhr-uni-bochum.de/georg.verweyen/pakete.html
\usepackage[babel]{microtype}

% Number also \subsubsection, but not \paragraph and below
\setcounter{secnumdepth}{3}

% Page heading and footer
\usepackage{scrpage2}
\pagestyle{scrheadings}
\automark[section]{chapter}
% heading on the top inner margin only
\ohead[]{\headmark}
\chead[]{}
% page number on the bottom outer margin only
\ofoot[\pagemark]{\pagemark}
\cfoot[]{}

% Support for list of acronyms
\usepackage[footnote,nohyperlinks,withpage]{acronym}

% References: can use section names
\usepackage{nameref}

% References: generate hyperlinks
\usepackage[plainpages=false]{hyperref}

% References style (default: 'numerical')
\usepackage[style=authoryear-ibid,
              maxcitenames=1,
              maxbibnames=3]{biblatex}
\defbibheading{lit}{\chapter*{Literaturverzeichnis}\markboth{Literaturverzeichnis}{Literaturverzeichnis}}
% References: bibliography database
%\addbibresource{main.bib}

% prints author names as small caps
\renewcommand{\mkbibnamefirst}[1]{\textsc{#1}}
\renewcommand{\mkbibnamelast}[1]{\textsc{#1}}
\renewcommand{\mkbibnameprefix}[1]{\textsc{#1}}
\renewcommand{\mkbibnameaffix}[1]{\textsc{#1}}

% References: use English ordinal numbers
\usepackage[super]{nth}

% Automatically use teletype for \url argument, use content verbatim
\usepackage{url}
\urlstyle{tt}

% Less vertical spacing between list items (in `compactitem' environment)
\usepackage{paralist}

% Multi-row table cells
\usepackage{multirow}

% Support for horizontal rules in tables (professional style)
\usepackage{booktabs}

% Footnotes in tables
\usepackage{threeparttable}

% Tables across pages (for results)
\usepackage{longtable}

% Word wrap in table columns (calculate p width)
\usepackage{calc}

% Automatic column stretching
% \usepackage{tabularx}

% Stretched tables across pages (for results); requires longtable
% tabularx
\usepackage{ltxtable}

% Improved table formatting
\usepackage{array}

% Support for including PDFs
\usepackage{pdfpages}

% Support for figures
\usepackage{graphicx}

\usepackage{amsthm}
\newtheorem{mydef}{Definition}

\usepackage{amsmath,esint}
% \usepackage{amssymb}
% \usepackage{commath}

% degree symbol etc.
\usepackage{gensymb}

% Margin notes
\usepackage{marginnote}

% FSM graphs
\usepackage{tikz}
\usetikzlibrary{arrows,automata}

% Captions
\usepackage{caption}

% Recalculate page spread based on the definitions above
\recalctypearea

%% User commands
% Formatting and languages
\newcommand{\code}[1]{\texttt{#1}}
\newcommand{\var}[1]{\textit{#1}}
\newcommand{\en}[1]{\foreignlanguage{english}{#1}}

% Abbreviations
\usepackage{xspace}
\newcommand{\ao}{\mbox{u.\,a.}\xspace}
\newcommand{\cf}{\mbox{vgl.}\xspace}
\newcommand{\ie}{\mbox{d.\,h.}\xspace}
\newcommand{\eg}{\mbox{e.g.}\xspace}

% Common terms
\newcommand{\sectionname}{Section}
\newcommand{\eq}{equation\xspace}

% Commands for common math expressions
\newcommand{\abs}[1]{\lvert#1\rvert}
\renewcommand\d[1]{\:\textrm{d}#1}
\newcommand*\diff{\mathop{}\!\mathrm{d}}

% alignment in \cases
\makeatletter
\renewcommand{\env@cases}[1][@{}l@{\quad}l@{}]{%
  \let\@ifnextchar\new@ifnextchar
  \left\lbrace
  \def\arraystretch{1.2}%
  \array{#1}%
}
\makeatother

% asides
\usepackage{mdframed}
\newenvironment{aside}
  {\begin{mdframed}[style=0,%
      leftline=false,rightline=false,leftmargin=2em,rightmargin=2em,%
          innerleftmargin=0pt,innerrightmargin=0pt,linewidth=0.75pt,%
      skipabove=7pt,skipbelow=7pt]\small}
  {\end{mdframed}}

\begin{document}
  % User-defined language-specific hyphenation
%  \hyphenation{bezüg-lich einer Da-ten-bank-ope-ra-ti-onen
%  effizienz-stei-gernd ECMA ECMAScript Firefox Google JavaScript JavaScript-Core
%  Kom-pa-ti-bi-li-täts-matrix lauf-fähig lenny Linux MySQL proto-typ-ba-sier-ter
%  robusteren SquirrelFish Wine}

  \begin{titlepage}
    \title{Classical Mechanics}
    \subtitle{as taught by Leonard\xspace Susskind,
    Ph.D.\xspace \\
      at Stanford University, 2011}
    \author{Thomas Lahn}
    \maketitle
  \end{titlepage}

  \clearpage
  \pagenumbering{Roman}
  \begin{spacing}{1}
    % Print TOC
    \tableofcontents
    \thispagestyle{empty}
  \end{spacing}

%   \clearpage
%   \begin{spacing}{1}
%     \chapter*{List of acronyms} \label{chapter:acronyms}
%     \begin{acronym}[]
%       \setlength{\itemsep}{-\parsep}
%        \acro{RFC}{Request for Comments (Internet-Standard)}
%     \end{acronym}
%   \end{spacing}

  \clearpage
  \pagenumbering{arabic}
  \chapter{Lecture 1 (2011-09-26)}
    \url{http://www.youtube.com/watch?v=ApUFtLCrU90}

    Classical mechanics is a set of rules how the laws of motion
    look like.

    Questions:
    \begin{compactitem}
      \item What are the specific laws for particular kinds of
      systems? Those will provide us with illustrations.
      \item What are the rules for the allowable laws?  What is the
      grand framework in which all the various specific laws are
      framed in?
    \end{compactitem}

    \section{Illustrations}
      \subsection{A coin}
        A system with two states, heads ($H$) and tails ($T$).

        Initial condition: Either $H$ or $T$

        First law of motion: nothing happens.
        \begin{align*}
          H &\rightarrow H \\
          T &\rightarrow T
        \end{align*}
        \begin{center}
          \begin{tikzpicture}[>=stealth',shorten >=1pt,auto,node distance=2cm]
            \node[state] (H)      {$H$};
            \node[state] (T) [right of=H]  {$T$};

            \path[->] (H) edge [loop above] node {} (H)
                       (T) edge [loop above] node {} (T);
          \end{tikzpicture}
        \end{center}
        This law may be boring but it is very powerful; in every
        instant of time you can say what the previous and next
        state of the system will be:
        \begin{align*}
          H &\rightarrow H \rightarrow H \rightarrow \cdots \\
          T &\rightarrow T \rightarrow T \rightarrow \cdots
        \end{align*}
        This is an example of a dynamic system with a law of motion
        -- in this case an updating.

        There is only one other possible law for this system -- a
        change of state:
        \begin{align*}
          H &\rightarrow T \\
          T &\rightarrow H
        \end{align*}
        \begin{center}
          \begin{tikzpicture}[>=stealth',shorten >=1pt,auto,node distance=2cm]
            \node[state] (H)      {$H$};
            \node[state] (T) [right of=H]  {$T$};

            \path[->] (H) edge [bend left=60] node {} (T)
                       (T) edge [bend left=60] node {} (H);
          \end{tikzpicture}
        \end{center}

        And the history of the world would be:
        \begin{align*}
          H &\rightarrow T \rightarrow H \rightarrow \cdots \\
          T &\rightarrow H \rightarrow T \rightarrow \cdots
        \end{align*}

        These laws can be expressed mathematically.  We can define $\sigma$ as a
        variable that can have only two values\footnote{$\sigma$ is a
        traditional variable in physics for things that have to do with two
        values}:
        \[ \sigma =  \begin{cases}[@{}r@{}l]
             1 & \quad \textrm{if the system is in state } H \\
            -1 & \quad \textrm{if the system is in state } T
          \end{cases} \]
        \begin{center}
          \begin{tikzpicture}[>=stealth',shorten >=1pt,auto,node distance=2cm]
            \node[state] (H) [label=120:{$\sigma=1$}] {$H$};
            \node[state] (T) [right of=H,label=60:{$\sigma=-1$}] {$T$};

            \path[->] (H) edge [bend left=60] node {} (T)
                       (T) edge [bend left=60] node {} (H);
          \end{tikzpicture}
        \end{center}

        We have now the idea of a \emph{configuration space} which
        is labeled by the two possible values of a certain
        variable.  The value of the variable tells in which
        configuration the system is.

        Let time be $t$ and assume that $t$ is an integer (stroboscopic
        world) . Then,
        \[ \sigma(t + 1) = \begin{cases}[@{}r@{}l]
            \sigma(t)  & \quad \textrm{if the coin stays the same} \\
            -\sigma(t) & \quad \textrm{if the coin flips between time indexes. }
          \end{cases} \]

        Note that this system is completely predictive; it is
        \emph{deterministic} as long as it is \emph{closed}
        (no interference from outside it).

      \subsection{A die}
        A die\footnote{singular of \emph{dice}} has six possible configurations
        (of lying it down on a table):
        \begin{center}
          \begin{tikzpicture}[>=stealth',shorten >=1pt,auto,node distance=2cm]
            \node[state] (1) at ( 180:1) {$1$};
            \node[state] (2) at ( 120:1) {$2$};
            \node[state] (3) at (  60:1) {$3$};
            \node[state] (4) at (   0:1) {$4$};
            \node[state] (5) at ( -60:1) {$5$};
            \node[state] (6) at (-120:1) {$6$};
          \end{tikzpicture}
        \end{center}
        An initial condition is a choice of one of these configurations.

        What are the possible laws of motion for this system?  The simplest law
        would be that nothing changes:
        \begin{center}
          \begin{tikzpicture}[>=stealth',shorten >=1pt,auto,node distance=2cm]
            \node[state] (1) at ( 180:1) {$1$};
            \node[state] (2) at ( 120:1) {$2$};
            \node[state] (3) at (  60:1) {$3$};
            \node[state] (4) at (   0:1) {$4$};
            \node[state] (5) at ( -60:1) {$5$};
            \node[state] (6) at (-120:1) {$6$};

            \path[->] (1) edge [loop left] node {} (1)
                       (2) edge [out=150,in=120,loop] node {} (2)
                       (3) edge [out=60,in=30,loop] node {} (3)
                       (4) edge [loop right] node {} (4)
                       (5) edge [out=-30,in=-60,loop] node {} (5)
                       (6) edge [out=-120,in=-150,loop] node {} (6);
          \end{tikzpicture}
        \end{center}
        A more interesting law would be to cycle around:
        \begin{center}
          \begin{tikzpicture}[>=stealth',shorten >=1pt,auto,node distance=2cm]
            \node[state] (1) at ( 180:1.5) {$1$};
            \node[state] (2) at ( 120:1.5) {$2$};
            \node[state] (3) at (  60:1.5) {$3$};
            \node[state] (4) at (   0:1.5) {$4$};
            \node[state] (5) at ( -60:1.5) {$5$};
            \node[state] (6) at (-120:1.5) {$6$};

            \path[->] (1) edge node {} (2)
                       (2) edge node {} (3)
                       (3) edge node {} (4)
                       (4) edge node {} (5)
                       (5) edge node {} (6)
                       (6) edge node {} (1);
          \end{tikzpicture}
        \end{center}
        \begin{aside}
          Homework: Write an equation for this law of motion.  \\
          Solution:
          \begin{center}
            \begin{tikzpicture}[>=stealth',shorten >=1pt,auto,node distance=2cm]
              \node[state] (1) [label=180:{$\sigma = 0$}] at ( 180:1.5) {$1$};
              \node[state] (2) [label=180:{$\sigma = 1$}] at ( 120:1.5) {$2$};
              \node[state] (3) [label=  0:{$\sigma = 2$}] at (  60:1.5) {$3$};
              \node[state] (4) [label=  0:{$\sigma = 3$}] at (   0:1.5) {$4$};
              \node[state] (5) [label=  0:{$\sigma = 4$}] at ( -60:1.5) {$5$};
              \node[state] (6) [label=180:{$\sigma = 5$}] at (-120:1.5) {$6$};

              \path[->] (1) edge node {} (2)
                         (2) edge node {} (3)
                         (3) edge node {} (4)
                         (4) edge node {} (5)
                         (5) edge node {} (6)
                         (6) edge node {} (1);
            \end{tikzpicture}
          \end{center}
          \[\sigma(t + 1) \equiv \sigma(t) + 1 \pmod 6\]
        \end{aside}

        We can write down other laws, for example:
        \begin{center}
          \begin{tikzpicture}[>=stealth',shorten >=1pt,auto,node distance=2cm]
            \node[state] (1) at ( 180:1.5) {$1$};
            \node[state] (2) at ( 120:1.5) {$2$};
            \node[state] (3) at (  60:1.5) {$3$};
            \node[state] (4) at (   0:1.5) {$4$};
            \node[state] (5) at ( -60:1.5) {$5$};
            \node[state] (6) at (-120:1.5) {$6$};

            \path[->] (1) edge node {} (2)
                       (2) edge node {} (5)
                       (3) edge node {} (4)
                       (4) edge node {} (6)
                       (5) edge node {} (3)
                       (6) edge node {} (1);
          \end{tikzpicture}
        \end{center}
        This law is logically equivalent to the previous one; we could just
        relabel the states, and rearrange them, and get the same one cycle:
        \begin{center}
          \begin{tikzpicture}[>=stealth',shorten >=1pt,auto,node distance=2cm]
            \node[state] (1) at ( 180:1.5) {$1$};
            \node[state] (2) at ( 120:1.5) {$2$};
            \node[state] (3) at (  60:1.5) {$4$};
            \node[state] (4) at (   0:1.5) {$5$};
            \node[state] (5) at ( -60:1.5) {$3$};
            \node[state] (6) at (-120:1.5) {$6$};

            \path[->] (1) edge node {} (2)
                       (2) edge node {} (5)
                       (3) edge node {} (4)
                       (4) edge node {} (6)
                       (5) edge node {} (3)
                       (6) edge node {} (1);
          \end{tikzpicture}
        \end{center}

        But we can think of laws which are not logically equivalent to that,
        for example
        \begin{center}
          \begin{tikzpicture}[>=stealth',shorten >=1pt,auto,node distance=2cm]
            \node[state] (1) at ( 180:1.5) {$1$};
            \node[state] (2) at ( 120:1.5) {$2$};
            \node[state] (3) at (  60:1.5) {$3$};
            \node[state] (4) at (   0:1.5) {$4$};
            \node[state] (5) at ( -60:1.5) {$5$};
            \node[state] (6) at (-120:1.5) {$6$};

            \path[->] (1) edge node {} (2)
                       (2) edge node {} (3)
                       (3) edge node {} (1)
                       (4) edge node {} (6)
                       (5) edge node {} (4)
                       (6) edge node {} (5);
          \end{tikzpicture}
        \end{center}
        because there are \emph{two} cycles, and if you are in one of the
        cycles, you will never get to the other one.

        This behavior of a law is called ``having a conserved quantity''.
        A \emph{conserved quantity} is something you can label a system with,
        that does not change with time.  Such a law is called a
        \emph{conservation law}.

        Another example:
        \begin{center}
          \begin{tikzpicture}[>=stealth',shorten >=1pt,auto,node distance=2cm]
            \node[state] (1) at ( 180:1.5) {$1$};
            \node[state] (2) at ( 120:1.5) {$2$};
            \node[state] (3) at (  60:1.5) {$3$};
            \node[state] (4) at (   0:1.5) {$4$};
            \node[state] (5) at ( -60:1.5) {$5$};
            \node[state] (6) at (-120:1.5) {$6$};

            \path[->] (1) edge [out=150,in=120,loop] node {} (1)
                       (2) edge [bend left=60] node {} (3)
                       (3) edge [bend left=60] node {} (2)
                       (4) edge node {} (5)
                       (5) edge node {} (6)
                       (6) edge node {} (4);
          \end{tikzpicture}
        \end{center}
        Again this is not logically equivalent to the above because now there
        are three cycles.  However, in each cycle a quantity is conserved.

        A configuration of a system consists of all you need to know to predict
        the future of the system.

        Note that nothing prevents us from having an infinite number of
        configurations.  (We do not need an infinite number of particles for
        that.) For example, if we have an infinite line on which we mark off
        integers, and a particle that can be in any of the marked positions:
        \begin{center}
          \begin{tikzpicture}[>=stealth',shorten >=1pt,auto,node distance=2cm]
            \node[state] (1) [label=below:$n-2$] {};
            \node[state] (2) [right of=1,label=below:$n-1$] {};
            \node[state] (3) [right of=2,label=below:$n$] {};
            \node[state] (4) [right of=3,label=below:$n+1$] {};
            \node[state] (5) [right of=4,label=below:$n+2$] {};
            \node[state] (6) [right of=5,label=below:$n+3$] {};

            \path[->] (1) edge [bend left=60] node {} (2)
                       (2) edge [bend left=60] node {} (3)
                       (3) edge [bend left=60] node {} (4)
                       (4) edge [bend left=60] node {} (5)
                       (5) edge [bend left=60] node {} (6);
          \end{tikzpicture}
        \end{center}

        This law does not have a conserved quantity.  That is different with the
        following:
        \begin{center}
          \begin{tikzpicture}[>=stealth',shorten >=1pt,auto,node distance=2cm]
            \node[state] (1) [label=below:$n-2$] {};
            \node[state] (2) [right of=1,label=above:$n-1$] {};
            \node[state] (3) [right of=2,label=below:$n$] {};
            \node[state] (4) [right of=3,label=above:$n+1$] {};
            \node[state] (5) [right of=4,label=below:$n+2$] {};
            \node[state] (6) [right of=5,label=above:$n+3$] {};

            \path[->] (1) edge [bend left=60] node {} (3)
                       (2) edge [bend left=-60] node {} (4)
                       (3) edge [bend left=60] node {} (5)
                       (4) edge [bend left=-60] node {} (6);
          \end{tikzpicture}
        \end{center}
        Here the ``oddness'' or ``evenness'' of the position of the particle is
        conserved.  We can define a quantity
        \[\sigma(t) = \begin{cases}
          0 & \textrm{if } n \equiv 0 \pmod 2 \textsf{ (the integer is even)} \\
          1 & \textrm{if } n \equiv 1 \pmod 2 \textsf{ (the integer is odd)}
        \end{cases}\]
        so that $\sigma$ does not change with time $t$.

        Until now we have discussed \emph{allowable laws of physics}, i.e.
        allowable in the sense of classical mechanics.

        Figure \ref{fig:not-allowable} is an example of a not-allowable law for
        a three-sided coin (heads $H$, tails $T$, and edges $E$):
        \begin{center}
          \begin{tikzpicture}[>=stealth',shorten >=1pt,auto,node distance=2cm]
            \node[state] (H) at (180:1) {$H$};
            \node[state] (T) at ( 60:1) {$T$};
            \node[state] (E) at (-60:1) {$E$};

            \path[->] (H) edge [bend left=60] node {} (T)
                       (T) edge [bend left=60] node {} (E) (E) edge [bend
                       left=60] node {} (T);
          \end{tikzpicture}
          \captionof{figure}{Not-allowable law of motion}
          \label{fig:not-allowable}
        \end{center}
        The history of the world would look as follows:
        \begin{align*}
          H &\rightarrow T \rightarrow E \rightarrow T \rightarrow \cdots \\
          T &\rightarrow E \rightarrow T \rightarrow E \rightarrow \cdots \\
          E &\rightarrow T \rightarrow E \rightarrow T \rightarrow \cdots
        \end{align*}
        This law is completely predictive into the future but it is not
        ``predictive into the past''.  If you are at $T$, you could have come
        from either $E$ or $H$.

        You cannot retrodict the past from this law of motion; IOW this law is
        \emph{not reversible}.  To see this, reverse every arrow in the
        configuration graph, thereby reversing time:
         \begin{center}
          \begin{tikzpicture}[>=stealth',shorten >=1pt,auto,node distance=2cm]
            \node[state] (H) at (180:1) {$H$};
            \node[state] (T) at ( 60:1) {$T$};
            \node[state] (E) at (-60:1) {$E$};

            \path[->] (T) edge [bend left=-60] node {} (H)
                       (E) edge [bend left=-60] node {} (T)
                       (T) edge [bend left=-60] node {} (E);
          \end{tikzpicture}
        \end{center}
        You have an unpredictive situation: from $T$ you can go either
        to $H$ or to $E$. This is a law that is not allowed in classical
        mechanics.

        \begin{aside}
          Limits on predictability

          If we have a perfectly predictive system of equations, it will not
          allow us to be completely predictive if we do not know the initial
          conditions exactly.  We need two things to predict the future: One,
          what the laws are and, two, what the initial conditions are.  This is not
          true in the real world where we are faced with degrees of freedom
          which are continuous: small changes in the initial conditions may or
          may not give rise to large changes in the future.

          If you do not know the initial conditions perfectly, you may want to
          quantify that: how far into the future can you predict things
          (example: weather forecast)?
        \end{aside}

        A law is reversible if in its graph each state has one, and only one,
        incoming and one, and only one, outgoing arrow.

    \section{Point particles moving in space}
      \subsection{Math preliminary: Coordinate systems, points, vectors}
        [\textellipsis]

        \subsubsection{Dot product}
          Let $A_B$ be the component of $\vec A$ along the axis $\vec B$, and
          $B_B$ the component of $\vec B$ along $\vec B$, then
          \begin{align*}
            \vec A \cdot \vec B :&= A_B B_B \\
            &= A_B \abs{\vec B} \\
            \theta :&= \angle (\vec A, \vec B) \\
            A_B &= \abs{\vec A} \cos{\theta} \\
            \vec A \cdot \vec B = \vec B \cdot \vec A &= \abs{\vec A}
            \abs{\vec B} \cos{\theta} \\
          \end{align*}

          It follows that
          \begin{align*}
            \vec A \cdot \vec B &< 0 \textsf{\,if\,} \theta > 90\degree \\
            \vec A \cdot \vec B &= 0 \textsf{\,if\,} \theta = 90\degree \\
            \vec A \cdot \vec B &= A_x B_x + A_y B_y + A_z B_z \\
            \vec A \cdot \vec A &= {\abs{\vec A}}^2 \\
            \vec B \cdot \vec B &= {\abs{\vec B}}^2 \\
            \cos{\theta} &= \frac{\vec A \cdot \vec B}{\abs{\vec A}
            \abs{\vec B}} = \frac{\vec A \cdot \vec B}{\sqrt{\vec A \cdot
              \vec A} \sqrt{\vec B \cdot \vec B}}
          \end{align*}
          showing that the angle between two vectors can be fully determined by
          having a dot product.

          \paragraph{Proof of the Law of Cosines}
            \begin{align*}
              \vec C &= \vec A - \vec B \\
              \vec C \cdot \vec C &= (\vec A - \vec B) \cdot (\vec A - \vec B)
              \\
              &= \vec A \cdot \vec A + \vec B \cdot \vec B - 2 (\vec A \cdot
              \vec B) \\
              {\abs{\vec C}}^2 &= {\abs{\vec A}}^2 + {\abs{\vec
              B}}^2 - 2 \abs{\vec A} \abs{\vec B}
              \cos{\theta}
            \end{align*}

      \subsection{Particles}
        There is a vector $\vec r = (r_x, r_y, r_z)$ starting in the origin that
        determines the position of a particle in space:
        \begin{align*}
          r_x = x \\
          \vdots
        \end{align*}

        Classical mechanics is concerned with the motion of particles, so that
        vector can be thought of as a function of time: $\vec r(t)$.  This also
        applies to its components:
        \begin{align*}
          r_x(t) = x(t) \\
          r_y(t) = y(t) \\
          r_y(t) = z(t)
        \end{align*}
        So the position of a particle (in three-dimensional space) is
        defined by three functions of time.

        The velocity of a particle also is a vector.  (It is not necessarily in
        the same direction as the position vector of the particle.)  The
        velocity is the time derivative of the position:
        \begin{align*}
          v_x &= \frac{\d }{\d t} r_x =
          \frac{\d x}{\d t} \\
          \vdots \\
          \vec v &= \left(\frac{\d x}{\d t},
          \frac{\d y}{\d t},
          \frac{\d z}{\d t}\right)
        \end{align*}
        Notation:
        \begin{align*}
          \frac{\d f(t)}{\d t} &= \dot f(t) \\
          v_i = \frac{\d x_i}{\d t} &= \dot x_i
        \end{align*}

        Acceleration is the rate of change with time, or the time derivative, of
        velocity:
        \begin{align*}
          a_i &= \dot v_i = \ddot x_i
        \end{align*}
        This can also be written in vector form:
        \begin{align*}
          \vec v &= \frac{\d \vec r}{\d t} = \dot{\vec r} \\
          \vec a &= \frac{\d ^2\vec r}{\d t^2} = \ddot{\vec r}
        \end{align*}
        \subsubsection{Examples}
          \paragraph{Motion along a line}
            \begin{align*}
              x(t) &= a + bt + ct^2 \\
              v = \dot x &= b + 2ct \\
              v(0) &= b \\
              a = \ddot x &= 2c
            \end{align*}
            This is a uniformly accelerated particle that has acceleration $2c$.
          \paragraph{Motion in a circle}
            Let $\theta$ be the angle between the radius vector of the particle
            and the positive part of the $x$ axis, and let it be described by
            \begin{align*}
              \theta :&= \omega t
            \end{align*}
            How long does it take for the particle to describe a full circle
            ($\theta$ going from $0$ to $2 \pi$)?
            \begin{align*}
              2 \pi &= \omega t \\
              t &= \frac{2 \pi}{\omega}
            \end{align*}
            $t$ is then called the \emph{period}, $\omega$ is the \emph{angular
            frequency}.

            What are the $x$ and $y$ components of the position of the particle,
            assuming motion on a unit circle ($\abs{\vec r} = 1$)?
            \begin{align*}
              r_x = x(t) &= \cos{\omega t} \\
              r_y = y(t) &= \sin{\omega t}
            \end{align*}

            If follows that
            \begin{align*}
              v_x = \dot x(t) &= -\omega \sin{\omega t} \\
              v_y = \dot y(t) &= \omega \cos{\omega t}.
            \end{align*}

            What is the angle between the position $\vec r$ and the velocity
            $\vec v$?
            \begin{align*}
              \vec r \cdot \vec v &= - \omega \cos{\omega t} \sin{\omega t} +
              \omega \sin{\omega t} \cos{\omega t} = 0
            \end{align*}
            shows that $\vec r \perp \vec v$: $\angle(\vec r, \vec v) =
            90\degree$.

            The acceleration is
            \begin{align*}
              a_x = \dot v_x(t) &= - \omega^2 \cos{\omega t} = - \omega^2 r_x \\
              a_y = \dot v_y(t) &= - \omega^2 \sin{\omega t} = - \omega^2 r_y
            \end{align*}
            so it is directed opposite to the position vector (which points
            outward), pointing inward towards the origin.

            The magnitude of velocity is
            \begin{align*}
              \abs{\vec v} &= \sqrt{{v_x}^2 + {v_y}^2} \\
              &= \sqrt{\left(-\omega
              \sin{\omega t}\right)^2 + \left(\omega \cos{\omega t}\right)^2} \\
              &= \sqrt{\omega^2 \underbrace{\left(\sin^2{\omega t} +
              \cos^2{\omega t}\right)}_{\sin^2{x} + \cos^2{x} = 1}} \\
              &= \omega.
            \end{align*}

            The magnitude of acceleration is
            \begin{align*}
              \abs{\vec a} &= \sqrt{{a_x}^2 + {a_y}^2} \\
              &= \sqrt{\left(- \omega^2 \cos{\omega t}\right)^2 + \left(-
              \omega^2 \sin{\omega t}\right)^2} \\
              &= \sqrt{\omega^4 \left(\cos^2{\omega t} +
              \sin^2{\omega t}\right)} \\
              &= \omega^2.
            \end{align*}

  \chapter{Lecture 2 (2011-10-03)}
    \url{https://www.youtube.com/watch?v=mYDrufxpW9E}

    \section{Aristotle's Law of Motion}
      Aristotle (384–322 BCE; ca. 1800 years before Newton) got mechanics
      wrong, but nevertheless we will spend a little time talking about his law
      of motion, and how it differs from Newton's law.

      Aristotle did not know how to write equations, but he used words
      to describe motion.  He got mechanics wrong because he lived in a
      world dominated by friction: In such a world, if you apply forces to
      objects, the object only moves as long as a force is applied to it;
      the heavier the object is, the harder you have to push it to keep it at a
      certain velocity; and the harder you push it, the faster it goes.

      So Aristotle believed that the connection between forces and motion were
      that the velocity of an object were proportional to the force that you
      put on it; not the acceleration. (It is quite clear that Aristotle had
      never gone ice-skating; otherwise he would have realized that it takes
      force to change motion, to start \emph{and} stop an object.)

      Let us see if we can find anything wrong with Aristotle's law of motion
      from the point of view of things we have already discussed.  One of the
      things we have discussed is the \emph{reversibility} of the laws of
      physics. Reversibility meaning that you do not lose information; that you can
      recoup the past as well as the future in the laws of physics.  We will see
      what Aristotle's laws have to say about that.
      
      Aristotle's hypothesis was that something called \emph{force} ($F$) was
      equal to mass ($m$) times velocity ($v$). We can write this as a vector
      equation:
      \begin{align*}
        \vec F &= m \vec v
      \end{align*}
      (This is of course \emph{wrong}; do not memorize it. The point of the
      following is merely to explain in what sense this is a law of motion, and
      in what sense it fails one of the tests that we already posed for the laws
      of motion.)
      
      To illustrate, let us go to a stroboscopic world where we break up
      time into very small intervals, each of size $\Delta$.  Then
      \begin{align*}
        \vec v = \dot x = \frac{x(t + \Delta) - x(t)}{\Delta}
      \end{align*}
      and Aristotle's force law is
      \begin{align*}
        \vec F(t) = m \frac{x(t + \Delta) - x(t)}{\Delta}.
      \end{align*}
      where we exert a known force that is a function of time.  We can solve
      this for $x(t + \Delta)$:
      \begin{align}\label{eq:aristotle}
        x(t + \Delta) = \frac{\Delta}{m} \vec F(t) + x(t)
      \end{align}
      This is the sort of law of physics which is telling you where the particle
      is at time $t + \Delta$ if you know where it is at time $t$.  If there is
      no force ($F(t) = 0$), then there is no velocity ($v = 0$) and the object
      stands still; you can predict its future and past.
      
      Let us now apply this law to a harmonic oscillator, to a spring, where we
      pretend that we do not know that the force on a particle moving on the
      $x$-axis is function of time, but assume that it is a function of
      position:
      the force on the particle depends on where it is.
      
      There is an equilibrium position, the \emph{origin},
      where $x = 0$.
      If the particle is away from that position, we assume that there is a force proportional
      to the distance that we displace the particle that pulls it back to
      equilibrium ($k$ then is called the \emph{spring constant}):
      \begin{align}\label{eq:spring}
        F(x) = -k\,x
      \end{align}
      
      What would that do in Aristotle's equations?  We replace
      $F(x)$ from \eq \ref{eq:spring} in \eq \ref{eq:aristotle} and get
      \begin{align*}
        x(t + \Delta) &= -k \frac{\Delta}{m} x(t) + x(t) \\
        &= x(t) \left[1 - \frac{k\Delta}{m} \right]
      \end{align*}
      This equation also allows you to predict the future. If you know a
      position at any time, it tells you to multiply the $x$ coordinate by a
      number slightly less than 1 (think of $\Delta$ as small), so that in each
      interval of time the position is diminished by the same common factor\ --
      the particle is just exponentially moving towards the origin.
      
      Instead of in the discrete form, you can also work this out in the
      continous form of Aristotle's law of motion:
      \begin{align*}
        m v = m \frac{\d x}{\d t} &= -k x = F(x) \\
        \frac{\d x}{\d t} &= -\frac{k}{m} x
      \end{align*}
      What is the solution of the equation – which function’s derivatives are
      proportional to itself? The exponential function:
      \begin{align*}
        x(t) &= A e^{-\frac{k}{m} t} \quad(A = const.) \\
        \dot x(t) &= -A \frac{k}{m} e^{-\frac{k}{m} t}
      \end{align*}
      $x(0) = A$, so
      \begin{align*}
        x(t) &= x(0)\,e^{-\frac{k}{m} t}.
      \end{align*}
      But this means that \emph{using this law you cannot predict the past}.  As
      all trajectories end in the origin, if the particle is displaced just a tiny amount from
      the origin and practically sits in the origin, with a \emph{finite}
      accuracy you cannot tell where it came from (in which direction the
      displacement was).  Therefore, Aristotle's law of motion is \emph{not
      reversible}.
      
      This does not happen with Newton's law of motion.
      
    \section{Newton's Law of Motion}
      With Newton's equations, simple systems like this do \emph{not}
      run all the to same point. If you start with measurably different
      configuration, you end up with measurably different configurations.
    
      Newton said that force is responsible for the \emph{change} of motion.
      Because friction is itself a force. In a world of friction, there are at
      least two balanced forces applied to an object put into motion: one the
      force that causes the motion, and the force of friction.  Not force is
      necessary to make velocity, but one force is necessary to overcome the
      other force.
      
      \begin{align*}
        F &= m\,a \\
          &= m\,\ddot{x}
      \end{align*}
      
      [How to test the proportionality in Newton's law using several
      masses and springs]
      
      Is this law predictive into the future?  Can you tell if you
      know where the object is, where and when it will be next?

      \begin{align*}
        F &= m\,\ddot{x} \\
          &= m\,\dot{v} \\
          &= m\,\left[\frac{x(t + \Delta) - x(t)}{\Delta^2} - \frac{x(t) - x(t
          - \Delta)}{\Delta^2} \right] \\
        \Delta^2\,\frac{F}{m} &= x(t + \Delta) + x(t - \Delta) - 2x(t) \\
        x(t + \Delta) &= \Delta^2\,\frac{F}{m} + 2x(t) - x(t - \Delta)
      \end{align*}
      
      So in order to tell where the object will be next, you need to know where
      it is now and where it was before.  IOW, you need to know its position and
      its velocity: the position tells you where it is now, the velocity where
      it was a moment ago,
\end{document}
