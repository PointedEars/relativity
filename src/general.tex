% vim:set fileencoding=utf-8 tabstop=2 shiftwidth=2 softtabstop=2 expandtab:
%% Page layout
% Scientific report, European style (A4) (from KOMA-Script)
% BCOR: binding correction
% DIV=calc: Calculate page spread now (recalculated below)
% pagesize: Add page info to (PDF/PS) output
% parskip=half: Do not indent paragraphs, add one half line spacing instead
%   (FFHS requirement)
% Default font size: 10pt (FFHS requirement)
% BCOR=1cm too much?
\documentclass[pagesize,headsepline,10pt,parskip=half,BCOR=12mm]{scrreprt}

% Line spacing should be 1.5 times the line height (FFHS requirement)
\usepackage{setspace}
\onehalfspacing

% float compatibility for KOMA-Script
\usepackage{scrhack}

% Use new German orthography and hyphenation (the latter is FFHS requirement)
\usepackage[ngerman,english]{babel}
% \usepackage[ngerman=ngerman-x-latest]{hyphsubst}

% Allow using non-ASCII characters verbatim
\usepackage[utf8]{inputenc}

% Special characters
\usepackage{textcomp}

% Use fonts having non-ASCII characters
\usepackage[T1]{fontenc}

% User-defined English hyphenation
\hyphenation{InfoWorld}

% Use language-specific quotes
\usepackage[autostyle,german=swiss,english=american]{csquotes}

% Advanced Computer Modern fonts
\usepackage{lmodern}

% Allow strike-through with \sout, keep italic for \emph
\usepackage[normalem]{ulem}

% Use medieval numbers except in math mode
% \usepackage{hfoldsty}

% Use sans-serif font ('Arial') by default for headings and normal text
% (FFHS requirement)
\renewcommand{\familydefault}{\sfdefault}
\usepackage{helvet}

% Improved typography, like hyphenation in words with non-ASCII characters,
% see also http://homepage.ruhr-uni-bochum.de/georg.verweyen/pakete.html
\usepackage[babel]{microtype}

% Number also \subsubsection, but not \paragraph and below
\setcounter{secnumdepth}{3}

% Page heading and footer
\usepackage{scrpage2}
\pagestyle{scrheadings}
\automark[section]{chapter}
% heading on the top inner margin only
\ohead[]{\headmark}
\chead[]{}
% page number on the bottom outer margin only
\ofoot[\pagemark]{\pagemark}
\cfoot[]{}

% Support for list of acronyms
\usepackage[footnote,nohyperlinks,withpage]{acronym}

% References: can use section names
\usepackage{nameref}

% References: generate hyperlinks
\usepackage[plainpages=false]{hyperref}

% References style (default: 'numerical')
\usepackage[style=authoryear-ibid,
              maxcitenames=1,
              maxbibnames=3]{biblatex}
\defbibheading{lit}{\chapter*{Literaturverzeichnis}\markboth{Literaturverzeichnis}{Literaturverzeichnis}}
% References: bibliography database
%\addbibresource{main.bib}

% prints author names as small caps
\renewcommand{\mkbibnamefirst}[1]{\textsc{#1}}
\renewcommand{\mkbibnamelast}[1]{\textsc{#1}}
\renewcommand{\mkbibnameprefix}[1]{\textsc{#1}}
\renewcommand{\mkbibnameaffix}[1]{\textsc{#1}}

% References: use English ordinal numbers
\usepackage[super]{nth}

% Automatically use teletype for \url argument, use content verbatim
\usepackage{url}
\urlstyle{tt}

% Less vertical spacing between list items (in `compactitem' environment)
\usepackage{paralist}

% Multi-row table cells
\usepackage{multirow}

% Support for horizontal rules in tables (professional style)
\usepackage{booktabs}

% Footnotes in tables
\usepackage{threeparttable}

% Tables across pages (for results)
\usepackage{longtable}

% Word wrap in table columns (calculate p width)
\usepackage{calc}

% Automatic column stretching
% \usepackage{tabularx}

% Stretched tables across pages (for results); requires longtable
% tabularx
\usepackage{ltxtable}

% Improved table formatting
\usepackage{array}

% Support for including PDFs
\usepackage{pdfpages}

% Support for figures
\usepackage{graphicx}

\usepackage{amsthm}
\newtheorem{mydef}{Definition}

\usepackage{amsmath,esint}
% \usepackage{amssymb}

% Margin notes
\usepackage{marginnote}

% Recalculate page spread based on the definitions above
\recalctypearea

%% User commands
% Formatting and languages
\newcommand{\code}[1]{\texttt{#1}}
\newcommand{\var}[1]{\textit{#1}}
\newcommand{\en}[1]{\foreignlanguage{english}{#1}}

% Abbreviations
\usepackage{xspace}
\newcommand{\ao}{\mbox{u.\,a.}\xspace}
\newcommand{\cf}{\mbox{vgl.}\xspace}
\newcommand{\ie}{\mbox{d.\,h.}\xspace}
\newcommand{\eg}{\mbox{e.g.}\xspace}

% Common terms
\newcommand{\sectionname}{Section}

\begin{document}
  % User-defined language-specific hyphenation
%  \hyphenation{bezüg-lich einer Da-ten-bank-ope-ra-ti-onen
%  effizienz-stei-gernd ECMA ECMAScript Firefox Google JavaScript JavaScript-Core
%  Kom-pa-ti-bi-li-täts-matrix lauf-fähig lenny Linux MySQL proto-typ-ba-sier-ter
%  robusteren SquirrelFish Wine}

  \begin{titlepage}
    \title{Einstein's Theory of Relativity}
    \subtitle{as taught by Leonard\xspace Susskind,
    Ph.D.\xspace \\
      at Stanford University, 2008}
    \author{Thomas Lahn}
    \maketitle
  \end{titlepage}

  \clearpage
  \pagenumbering{Roman}
  \begin{spacing}{1}
    % Print TOC
    \tableofcontents
    \thispagestyle{empty}
  \end{spacing}

%   \clearpage
%   \begin{spacing}{1}
%     \chapter*{List of acronyms} \label{chapter:acronyms}
%     \begin{acronym}[]
%       \setlength{\itemsep}{-\parsep}
%        \acro{RFC}{Request for Comments (Internet-Standard)}
%     \end{acronym}
%   \end{spacing}

  \clearpage
  \pagenumbering{arabic}
  \chapter{Lecture 1}\label{chapter:introduction}
    \url{http://www.youtube.com/watch?v=hbmf0bB38h0}
    \section{Newtonian Gravity}
      \begin{mydef}
      In an \textbf{inertial frame of reference}, if there are no
      objects around to exert forces on a test object, that object
      will move with uniform motion, with no acceleration.
      \end{mydef}

      Newton's First/Second Law:
      \begin{align}\label{eq:newton}
        \vec F = m\vec a
      \end{align}
      (``$Force = mass \times acceleration$.''
      This is a \emph{vector equation}.  $\vec F$ is a
      \emph{vector}. A vector has \emph{components}, a
      \emph{direction} and a \emph{magnitude}.)

      In Newtonian physics, the mass is \emph{conserved} (it does
      not change).

      Assuming a coordinate system with axes $x_1$, $x_2$, and
      $x_3$ (or $x$, $y$, and $z$), then the coordinates of a
      particle are components of the \emph{position vector} or
      \emph{radial vector} of that particle, $\vec R = (x_1, x_2,
      x_3)$ (``R'' for radius, meaning \eg the distance between the
      particle and the origin).

      The acceleration is a vector whose components are the
      second time derivatives of $x$, $y$, and $z$ (or $x_1$,
      $x_2$, and $x_3$), $\vec a = \frac{d^2\vec x}{dt^2}$, where
      $\vec x = (x_1, x_2, x_3)$.

      (The first derivative of position is velocity: $\vec v =
      \frac{d\vec x}{dt}$ [Acceleration is the change in velocity
      over time.])

      So Newton's First/Second Law can be written as
      \begin{align}\label{eq:newton-diff}
        \vec F = m\frac{d^2\vec x}{dt^2}.
      \end{align}

      \subsection{Galilean Gravity}

        Galileo studied the motion of objects in the gravitational
        field of the Earth \emph{in the approximation that the Earth
        is infinite and flat}, so that

        \begin{compactitem}
          \item the direction of gravitational forces is the same
          everywhere (downwards; as opposed to the real
          gravitational field of the Earth where they all point to
          the center) and
          \item the gravitational force does not depend on how high
          you are, meaning the acceleration of gravity does not
          depend on position.
        \end{compactitem}

        He found:
        \begin{align}\label{eq:galileo}
          F_2 &= -mg
        \end{align}
        [$F_2$ for the force component in the $x_2$ (vertical)
        direction; the minus sign says that the force acts downward.]

        That the strength of the gravitional force on an object is
        proportional to its mass is a special property of gravity.
        It is not true for, \eg the electromagnetic force, which is
        proportional to the electric charge.

        Galileo's (equation~\ref{eq:galileo}) combined with the
        Newton's force law (equation~\ref{eq:newton-diff}):
        \begin{align}
          m\frac{d^2\vec x}{dt^2} &= -mg \\
          \xout{m}\frac{d^2\vec x}{dt^2} &= -\xout{m}g
        \end{align}

        As the mass cancels out from both sides, the motion of an
        object, in particular its acceleration due to gravity, does
        not depend on the mass of the object or anything else. (All
        objects fall the same way, not considering air resistance.)

        $g = 9.81 \frac{m}{s^2}$ on the Earth (``10 meters per second
        per second''). It is smaller on the Moon, larger on Jupiter.

        So the gravitational acceleration does not depend on the mass
        of the accelerated object, only on the mass of the object it
        is accelerated by (``that you are dropping it onto''). This
        is the simplest form of the \emph{equivalence principle}.

        As a result, the shape of a cloud of particles, large and
        small (heavy and light), does not change when it is falling.
        \emph{Falling in a gravitational field is undetectable} by
        looking from one particle to another.  You cannot tell the
        difference between free fall due to gravity, and absence of
        gravity in free space. (This only holds in the flat Earth
        approximation.)

      \subsection{Newton's Law of Gravity}
        \begin{align}\label{eq:newton-gravity}
          F = \frac{mMG}{R^2}
        \end{align}
        where $m$ is a (small) mass, $M$ is a large/other mass, and
        $R$ is the distance between them.  $F$ is the magnitude of
        each of the gravitational forces, equal and opposite to
        each other, exerted by the masses on each other.

        $G \approx 6.7 \times 10^{-11} m^3 kg^{-1} s^{-2}$ is the
        \emph{gravitational constant}, a numerical constant so that
        the equation is dimensionally consistent ($1 N = 1 kg
        \times m \times s^{-2}$).
        That it is so small shows that gravity is comparably a very
        weak force.
        [Experiment: An object hanging by a string. Charge it
        electrically and put another charged object near it, they
        will visibly repel (electrostatic). Use a ball of iron and
        put a magnet next to it, it will be visibly attracted
        (magnetic). Put a 10,000 lbs weight next to it, and no
        change will be visible (gravitational).]

        However, we feel the gravitational force on the Earth
        strongly because the Earth is so heavy (has a large mass
        $M$), which makes up for the small $G$.

        Why is it that when I am falling towards the Earth, I
        accelerate so much and the Earth so little even though the
        gravitational forces are equal? Because (per equation~\ref{eq:newton}) acceleration involves the force \emph{and}
        the mass.  The bigger the mass, the less the acceleration.

        [Q: How did Newton derive his law? A: Probably from
        Kepler's Laws of planetary motion. Audience: Edmund Halley
        asked him (Newton) ``What kind of force law do you need for
        elliptical orbits?''. Audience: I think he asked the
        question for inverse square laws, and that Newton already
        knew it was a ellipse.
        A: An ellipse was not necessary, the orbits might have been
        circular.  It was the fact that the period varies as the
        three-halves power of the radius. Circular motion has an
        acceleration towards the center. If you know the period and
        the radius, then you know the acceleration toward the
        center:

        $\omega^2 R$ is the acceleration, where $\omega$ is the
        angular period of going round in an orbit, roughly the
        inverse period (number of cycles per second).  Suppose
        Newton set that equal to some unknown force law
        \begin{align}
          \omega^2 R = F(R)
        \end{align}
        and divided by $R$:
        \begin{align}
          \omega^2 = \frac{F(R)}{R}
        \end{align}
        For the real case, inverse square law:
        \begin{align}
          \omega^2 = \frac{F(R)}{R} = \frac{\frac{X}{R^2}}{R} =
          \frac{X}{R^3}
        \end{align}
        and in that form it is Kepler's Third Law: ``The square of
        the orbital period of a planet is directly proportional to
        the cube of the semi-major axis of its orbit.''

        Then he realized that if you did not have a perfectly
        circular orbit, the inverse square law was the unique law
        which would give elliptical orbits.]

        [Q: What happens if the objects are touching? A: Then it
        (the law) breaks down. Other forces come into play, for
        example electrostatic forces. The force then is not
        proportional to mass, but \eg to electric charge: $F
        \propto C$.]

        Equation~\ref{eq:newton-gravity} written as a vector
        equation, where $M$ is placed in the origin:
        \begin{align}\label{eq:newton-gravity-vector}
          \vec F = -\frac{mMG}{R^2} \frac{\vec R}{R}
        \end{align}
        [$\vec R$ indicates the direction of the force, the minus
        sign indicates that the force exerted on $m$ by $M$ is
        opposite to the radial vector from $M$ to $m$, and we must
        divide by $R$ to keep the magnitude ($\frac{\vec R}{R}$ is
        a \emph{unit vector}.)]

        If you have $i$ particles, the force exerted by the other
        ($j$) particles on the $i$-th particle is
        \begin{align}
          \vec F_i &= -\sum_{j \neq i}{\frac{m_i
          M_jG\vec R_{ji}}{{R_{ji}}^3}}\\
          &= \sum_{j \neq i}{\frac{m_i
          M_jG\vec R_{ij}}{{R_{ij}}^3}}
        \end{align}
        [The force the $M_j$'s exert on $m_i$ along $\vec R_{ji}$
        is attractive, therefore negative by definition. If we
        reverse the direction of the vector, $\vec R_{ij} = -\vec R_{ji}$, we can and must remove the minus sign in front of
        the sum.]

        Combined with equation~\ref{eq:newton}:
        \begin{align}\label{eq:newton-combined}
          m_i \vec a &= \sum_{j \neq i}{\frac{m_i
          M_jG\vec R_{ij}}{{R_{ij}}^3}} \\
          \xout{m_i} \vec a &= \sum_{j \neq i}{\frac{\xout{m_i}
          M_jG\vec R_{ij}}{{R_{ij}}^3}}
        \end{align}
        [As the $m_i$ cancel out, again the acceleration of
        particle $i$, its motion, does not depend on its mass, but
        that of all the other particles (equivalence principle).]

        \subsubsection{Tidal forces}
          Note that different to Galileo's flat Earth approximation,
          in Newton's Law of Gravity the gravitational acceleration
          depends on position, on the distance between the particles.

          So you can tell to some extent the difference between
          being in free fall in gravitational field and in free
          space:
          Because the acceleration on the lower part of your body
          is larger than on the upper part of your body, you are
          stretched vertically (towards the center of gravity); and
          as the forces are not parallel to each other, but point
          to the center of gravity, you are squished
          (compressed) horizontally.
          But if the object is small enough, then the gradient of
          the gravitational field across the size of the object
          will be negligible, and all parts of it will experience
          the same gravitational acceleration.

          See also the tides created by the Moon's gravitational
          pull on the water shell of the Earth's oceans; hence
          \emph{tidal} forces.

      \subsection{The gravitational field}
        Imagine next to a set of particles one more particle, a
        test particle, and ask what the force on it is. You can use
        it to map out the force/acceleration on it by observing how
        it accelerates in each point.

        Acceleration is a vector, so mapping out the acceleration
        in every point of space gives us a \emph{vector field}, the
        \emph{gravitational field}, where acceleration is a
        function of position (since we already know the force does
        not depend on the mass of the test particle, we can
        concentrate on the acceleration even though it is a
        \emph{force field}):
        \begin{align*}
          \vec A(x)
        \end{align*}
        If the particles are moving, that field also depends on
        time:
        \begin{align*}
          \vec A(x, t)
        \end{align*}
        To work out how it is defined, we use it in equation
        \ref{eq:newton-combined}:
        \begin{align}
          m_i \vec A &= \sum_{j \neq i}{\frac{m_i
          M_jG\vec R_{ij}}{{R_{ij}}^3}} \\
          \xout{m_i} \vec A &= \sum_{j \neq i}{\frac{\xout{m_i}
          M_jG\vec R_{ij}}{{R_{ij}}^3}}
        \end{align}

      \subsection{Gauss's Theorem}
        Imagine a vector field $\vec A = \vec A(\vec x)$ (``field''
        means it depends on position, $\vec x$).

        The \emph{divergence} $ \vec \nabla \cdot \vec A$ of a vector
        field $\vec A$ is the degree to which the field is spread
        out.
        If all vectors point outwards, it has a positive divergence. If they all point
        inwards, they converge -- it has a \emph{negative
        divergence}.

        Suppose the field has components $A_x$, $A_y$, and $A_z$.
        Divergence has to do with how the field varies in space. If
        the field is the same everywhere (same direction and
        magnitude), it has no divergence, no tendency to spread out.
        But if the magnitude or direction varies, it does. So
        divergence has to do with derivatives of the components of
        the field:
        \begin{align}
          \vec \nabla \cdot \vec A &= \partial_x A_x + \partial_y
          A_y + \partial_z A_z \\
          &= \frac{\partial A_x}{\partial x} + \frac{\partial
          A_y}{\partial y} + \frac{\partial A_z}{\partial z}
        \end{align}
        [$\vec \nabla \cdot \vec A$ is a scalar quantity. $\nabla$
        is spoken ``del'' as it looks like an upside-down delta.]

        [Visualization: Water pumped into the center of a flat
        lake. Also, if more water was flowing out of the lake than
        flowing in, which creates a divergence, this must mean that
        water was being pumped in.]

        Gauss's Theorem: Take any surface (or any curve, in two
        dimensions) and suppose there is a vector field. Think of
        it as the flow of water.  Now let's take the total amount
        of water that's flowing out of the surface. There is some
        water flowing out, and we want to subtract the water that
        is flowing in.

        The total amount of water flowing out of the surface is an
        integral over the surface: We have to add up the flow of
        water outward by breaking up the surface into little pieces
        and asking how much flow is coming out from each little
        piece. If the water is incompressible, and the depth of the
        water is fixed, there has to be a divergence in the
        interior.  So there is a connection between the water
        flowing out and the divergence in the interior, described
        by \emph{Gauss's Theorem}:
        \begin{align}\label{eq:gauss}
          \int (\vec \nabla \cdot \vec A)\, \mathrm{d}x\,
          \mathrm{d}y\, \mathrm{d}z &= \int A_\perp\,
          \mathrm{d}\sigma
          \\
          \iiint\limits_V(\vec \nabla \cdot \vec A)\, \mathrm{d}V
          &= \oiint \limits_{\sigma} A_\perp\, \mathrm{d}\sigma
        \end{align}
        [$d\sigma$ is an infinitesimal area of the surface,
        $A_\perp$ is the component of $\vec A$ perpendicular to the
        surface.]

        Application: Let there be a divergence of a vector
        field, $\vec \nabla \cdot \vec A$, that is concentrated in
        a region of space that has spherical symmetry.  What is the
        field $\vec A$ elsewhere?

        Draw a surface around.  As long as the divergence is
        concentrated inside the spherical region, the left-hand
        side of the equation does not depend on the size of the
        surface.  The left hand side is a number, $Q$.
        \begin{align}\label{eq:gauss}
          Q &= \int A_\perp\, \mathrm{d}\sigma
        \end{align}
        As for the right-hand side, if the flow is radially going
        outward, it is perpendicular to the surface of $\vec A$
        and the same everywhere: it is just the magnitude of
        $\vec A$, $A$.  And the integral over the surface areas of
        the sphere $d \sigma$ is just the surface of the sphere,
        $4 \pi R^2$:
        \begin{align}
          Q &= \int A_\perp\, \mathrm{d}\sigma \\
            &= 4 \pi A R^2
        \end{align}
        So the magnitude of the field is equal to the total
        integrated divergence divided by $4\pi R^2$.
        \begin{align}
          A &= \frac{Q}{4 \pi R^2} \\
          \vec A &= \frac{Q}{4 \pi R^2} \frac{\vec R}{R}
        \end{align}
        This is a vector field, it is pointed radially outward (if
        the divergence is positive), and its magnitude is
        $\frac{1}{R^2}$.  It is exactly the gravitational field of
        a point particle at the center of the divergence.

        So a point mass can be thought of as a concentrated
        divergence of a gravitational field.  Only that the real
        gravitational acceleration points inward, which is an
        indication that the divergence is negative.  The Newtonian
        gravitational field is isomorphic to a flow field where it
        is all sucked out from a single point.

        It also points out Newton's theorem: The size of the
        object, as long as it is spherical symmetrical, does not
        matter for the shape of the gravitational field; it can be
        thought of as a \emph{point mass}.

        Examples:
        \begin{compactitem}
          \item Calculate the gravitational force (or acceleration)
            of a particle outside the Earth, but not so near that
            you can make the flat Earth approximation.  You do not
            have to consider every point of the Earth pulling at
            the particle; just assume the mass of the Earth is
            concentrated in its center.
          \item If you have a spherical shell of material from
            which water was flowing outward, from the outside it
            looked as the flow was concentrated at its center; but
            inside there would be no flow. Likewise with gravity:
            from the outside it looked as if the mass was
            concentrated in the center, on the inside there
            would be no gravity.  (Analog in general relativity:
            The gravitational field of everything, as long as it is
            spherical symmetrical, looks exactly like the
            gravitational field of a black hole.)
        \end{compactitem}
  \chapter{Lecture 2}
    \url{http://www.youtube.com/watch?v=s8UrYIZhm60&list=PL6C8BDEEBA6BDC78D}
    \section{Aside: On Dark Energy and the Big Rip Theory}
    \section{Some mathematics}
      $\vec \nabla = \left(\frac{\partial}{\partial x},
      \frac{\partial}{\partial y}, \frac{\partial}{\partial
      z}\right)$ is a \emph{differential operator}.

      Examples:
      \begin{align*}
        \vec \nabla \phi(x, y, z) &= \left(\frac{\partial
        \phi}{\partial x}, \frac{\partial \phi}{\partial y}, \frac{\partial
        \phi}{\partial z}\right)\\
        \vec W \cdot \vec V &= W_x V_x + W_y V_y + W_z V_z\\
        \vec \nabla \cdot \vec V &= \frac{\partial V_x}{\partial x}
        + \frac{\partial V_y}{\partial y} + \frac{\partial
        V_z}{\partial z}
      \end{align*}

      $\phi(x, y, z)$ is a scalar that depends on position. It is a
      \emph{scalar field}.
    \section{Gravitational field}
      \begin{align*}
        \vec A(x) = \sum_i \frac{Gm_i}{{R_i}^2} \hat{R_i}
      \end{align*}

      where the force direction is from a test particle to particle
      $i$ along $\vec R_i$. [$\hat{R_i} = \frac{\vec R_i}{R_i}$ is
      a unit vector.]

      $\vec \nabla \cdot \vec A =\, ?$

      \emph{Mass density}: The amount of mass concentrated per unit
      volume.
      \begin{align}
        \rho = \frac{\Delta m}{\Delta V}
      \end{align}
      The mass density varies with position, so it is itself
      a scalar field.

      Gauss's Law:
      \begin{align}
        \vec \nabla \cdot \vec A = -4 \pi \rho G
      \end{align}
      This can be thought of as a \emph{field equation} as it
      relates the field of gravitational acceleration $\vec A$ to
      the field of mass density $\rho$. [The minus sign points out
      that the gravitational field is a convergence, or a negative
      divergence.]

      From Gauss's Theorem follows:
      \begin{align}
        \int (\vec \nabla \cdot \vec A)\,\mathrm{d}x\,\mathrm{d}y\,\mathrm{d}z &= \int
        \vec A_\perp\,\mathrm{d}\sigma\\
        -4 \pi G \int \rho\,\mathrm{d}^3\vec x &= \int \vec
        A_\perp\,\mathrm{d}\sigma\\
        -4 \pi G M &= A_\perp 4 \pi R^2\\
        -\xout{4 \pi} G M &= A_\perp \xout{4 \pi} R^2\\
        A_\perp &= \frac{-G M}{R^2}
      \end{align}
      where $\vec R$ goes from the center of mass to the
      borders of the surrounding sphere.

      \subsection{The gravitational field inside a rigid body}
        \begin{align}
          A_\perp 4 \pi R^2 &= -4 \pi G M
        \end{align}
        where $\vec R$ goes from the center of mass to the
        borders of the relevant mass $M$ (\emph{not the entire mass
        of the rigid body}).

        We assume the mass density is uniform:
        \begin{align}
          A_\perp 4 \pi R^2 &= -\frac{4\pi}{3} R^3 \rho G \\
          A_\perp R^2 &= -\frac{1}{3} R^3 \rho G \\
          A_\perp &= -\frac{\rho G}{3} R
        \end{align}
        The acceleration on a test mass within a rigid body is
        towards the center and proportional to the distance
        from the center. A system where the force/acceleration
        grows linearily with distance is a \emph{harmonic
        oscillator}, so such a test object behaves like on a
        spring; the spring constant here is $\frac{\rho G}{3}$.

    \section{Equivalence principle}
      Elevator analogy: A person in an elevator accelerated in one
      direction (an accelerated frame of reference) cannot tell the
      difference between that motion and being in uniform motion in
      a gravitational field which points in the opposite direction.

    \section{Coordinate transformations}
      Let an frame of reference move along the $x$ axis of space
      with speed $v$ relative to an inertial frame of reference,
      and have two coordinates in the latter, one for space, $x$,
      one for time, $t$. An observer within that moving frame of
      reference has its own coordinates, $x'$ and $t'$,
      respectively. (We use the elevator analogy where $x=0$ and
      $x'=0$ are at the bottom of the elevator rotated clockwise
      and moving from left to right.)

      Because the $t'$ axis is tilted towards the right as the
      frame of reference is moving with time (see space-time
      diagram), if the frame of reference it is moving with
      uniform relative velocity $v$, the distance between $(x, t)$
      and $(x', t')$ is the distance $s$ that the elevator has
      moved in our frame of reference in time $t$: $s = vt$.
      Also, in Newtonian physics all observers have the same time:
      \begin{align}
        x' &= \left[x - s =\right] x - vt\\
        t' &= t
      \end{align}

      Example: If a ball in the elevator was standing still in my
      frame of reference ($v = 0$), what would be the velocity $v'$
      in the elevator observer's frame of reference?
      \begin{align}
        x &= const.\\
        v' = \dot{x'} :&= \frac{dx'}{dt} = -v \\
        a' = \ddot{x'} &= 0
      \end{align}
      IOW, the ball would be moving to the bottom of the elevator
      with speed $v$ for the observer in the elevator.  And if it
      is not accelerated for me, it also is not accelerated for the
      observer in the elevator.

      If the frame of reference is moving with uniform
      acceleration $a=g$, then the distance $s$ that the elevator
      moves in our frame of reference is \emph{not} $s = vt$,
      because

      \begin{align}
        \dot{s} &= v \\
        \dot{v} &= g \\
        \dot{s} &= v = \left[\int\, g\,\mathrm{d}t =\right]
        gt\,[+ \mathrm{C}]\footnotemark
        \\
        s &= \left[\int\, (gt + \mathrm{C})\,\mathrm{d}t
        =\right] \frac{1}{2} gt^2\,[+ \mathrm{C}t] \\
        x' &= x - \frac{1}{2} gt^2\,[+ \mathrm{C}t] \\
        t' &= t \\
        \dot{x'} &= -gt\,[+ \mathrm{C}] \\
        \ddot{x'} &= -g
      \end{align}
      \footnotetext{$\mathrm{C}$ is an arbitrary integration
      constant that becomes irrelevant with regard to
      acceleration. Prof.\,Susskind did not use the exact
      integrals.}

      This is a mathematical statement saying
      that acceleration works the same way as gravity.

      Note that the $x'$ coordinates of the accelerated frame of
      reference are described by \emph{parabolas}: $x' = x -
      \frac{1}{2} gt^2\,[+ \mathrm{C}t]$. IOW: \emph{An accelerated
      frame of reference is described by a coordinate transformation to a
      curvilinear coordinate system.}

    \section{The bending of light}
      In an elevator standing still relative to an inertial frame
      of reference, shine a light ray parallel to the bottom of
      the elevator.  It will travel in a straight line.

      However, if the elevator is accelerating upward with $g$, and
      the light ray is emitted at the instant that the
      acceleration starts, the light ray will travel on a parabolic
      trajectory with respect to the moving frame of reference,
      because it hits the opposite wall when the elevator has moved
      upward.  You will find that the light ray has a downward
      component of acceleration which is equal to $g$.

      From this, Einstein deduced that light falls in a
      gravitational field.

      [Q: If a light ray passes the Sun, by what angle is the light
      ray deflected?
      A: Let the light ray skim the Sun (so that the effect is
      maximal).  When the ray passes past the Sun (within twice the
      radius of the Sun, $2R$), it has an acceleration towards the
      Sun which is the acceleration field of the Sun:
      \begin{align*}
        A_y &= \frac{MG}{R^2} \\
      \end{align*}
      The downward component of velocity is:
      \begin{align*}
        v_y = A_y \Delta t &= \frac{MG}{R^2} \Delta t\\
        \Delta t &= \frac{2R}{v} = \frac{2R}{c} \\
        v_y &= \frac{MG}{R^2} \frac{2R}{c}\\
            &= \frac{2MG}{Rc}
      \end{align*}
      If the deflection angle $\theta$ is small, it is
      approximately the ratio between the downward component and
      the horizontal component of velocity:
      \begin{align*}
        \theta &\approx \frac{v_y}{v_x} \\
        &\approx \frac{\frac{2MG}{Rc}}{\frac{1}{c}} \\
        &\approx \frac{2MG}{Rc^2}
      \end{align*}]

    \section{TODO}
      Because the gravitational field of the Earth (or any rigid
      body) is not uniform -- the acceleration has different
      directions to begin with --, it cannot be replaced with a
      single accelerated frame of reference.  But in an
      infinitesimally small frame of reference (elevator) for an
      infinitesimal amount of time (small enough and short enough
      to prevent detection of the different direction of gravity)
      we can assume equivalence.

      However, a person outside the Earth can tell the difference
      between being in the Earth's gravitational field and free
      space because the gravitational field of the Earth is not
      uniform: it has a different direction everywhere, and in
      varies with distance.  There are tidal forces that are an
      \emph{obstruction} to eliminating the gravitational field by
      replacing it by an accelerated frame of reference.

    \section{Geometry}
      A space can be fully described by the specifying the distance
      between neighboring points.

      Assume two points $\vec x_i = (x, y)$ and $\vec y_i = (x +
      \mathrm{d}x, y + \mathrm{d}y)$.  The square of the distance $\mathrm{d}s$
      between them is
      \begin{align}\label{eq:flat}
        \mathrm{d}s^2 = \mathrm{d}x^2 + \mathrm{d}y^2
      \end{align}
      but only if the coordinates are rectangular.

      If the coordinates are curved, that is, the lines of equal
      coordinate value are curves, the following \emph{quadratic
      form} holds:
      \begin{align*}
        \mathrm{d}s^2 = g_{11}\,\mathrm{d}x^2 + 2\,g_{12}\,
        \mathrm{d}x\,\mathrm{d}y + g_{22}\,\mathrm{d}y^2
      \end{align*}
      where the $g_{ij}$ depend on position; they are scalar
      fields:
      \begin{align*}
        \vec x &= (x, y) = (x_1, x_2)
      \end{align*}
      \begin{align}
        \mathrm{d}s^2 &= g_{11}(\vec x)\,{\mathrm{d}x_1}^2 + 2\,
        g_{12}(\vec x)\,\mathrm{d}x_1\,\mathrm{d}x_2 + g_{22}(\vec
        x)\,{\mathrm{d}x_2}^2
      \end{align}

      Interpretation:

      If $2\,g_{12}(\vec x)\,\mathrm{d}x_1\,
      \mathrm{d}x_2 \neq 0$, the coordinate axes
      $\vec x_1$ and $\vec x_2$ are not perpendicular.

      The $g_{11}$ and $g_{22}$ tell about the relative spread of
      coordinates: if the $x_1$ set of coordinates was denser
      than the $x_2$ set, both values would differ.

      The functions of position $g_{11}(\vec x)$, $g_{12}(\vec x)$,
      and $g_{22}(\vec x)$ are called the \emph{metric} of the
      space.

      A geometry/space that can be described by a simple form
      like equation \ref{eq:flat} is called \emph{flat}.
      All other geometries/spaces are called \emph{curved}.

      \emph{Curvature} tells whether a geometry/space is flat.  It
      is the obstruction to flattening out the coordinates.  Tidal forces
      are the obstruction to getting rid of the gravitational field
      by changing coordinates to accelerated frames of reference.
      That connection is not accidental.

  \chapter{Lecture 3}
    \section{Light moving in the direction of acceleration}
      Assume a light beam shot downward an upwardly accelerated
      moving elevator, and the elevator at rest at first.

      \begin{align*}
ds^2 &= g_{\mu\nu} dx^\mu dx^\nu \\
g_{\mu\nu} &= \begin{bmatrix}
g_{00} & g_{01} & g_{02} & g_{03}\\
g_{10} & g_{11} & g_{12} & g_{13}\\
g_{20} & g_{21} & g_{22} & g_{23}\\
g_{30} & g_{31} & g_{32} & g_{33}
\end{bmatrix} = \begin{bmatrix}
c^2 & 0 & 0 & 0\\
0 & -1 & 0 & 0\\
0 & 0 & -1 & 0\\
0 & 0 & 0 & -1
\end{bmatrix} &
dx^{\mu} &= \begin{pmatrix}
dx^0\\
dx^1\\
dx^2\\
dx^3
\end{pmatrix}\\
ds^2 &= g_{00} dx^0 dx^0 + g_{11} dx^1 dx^1 + g_{22} dx^2 dx^2
+ g_{33} dx^3 dx^3\\
&= \begin{pmatrix}
dx^0 dx^0\\
dx^1 dx^1\\
dx^2 dx^2\\
dx^3 dx^3
\end{pmatrix} \begin{pmatrix}
c^2\\
0\\
0\\
0
\end{pmatrix} + \begin{pmatrix}
dx^0 dx^0\\
dx^1 dx^1\\
dx^2 dx^2\\
dx^3 dx^3
\end{pmatrix} \begin{pmatrix}
0\\
-1\\
0\\
0
\end{pmatrix} + \begin{pmatrix}
dx^0 dx^0\\
dx^1 dx^1\\
dx^2 dx^2\\
dx^3 dx^3
\end{pmatrix} \begin{pmatrix}
0\\
0\\
-1\\
0
\end{pmatrix} + \begin{pmatrix}
dx^0 dx^0\\
dx^1 dx^1\\
dx^2 dx^2\\
dx^3 dx^3
\end{pmatrix} \begin{pmatrix}
0\\
0\\
0\\
-1
\end{pmatrix}\\
\left\|\begin{pmatrix}
c^2\\
0\\
0\\
0
\end{pmatrix}\right\| &= \sqrt{c^2 c^2} = c^2
\end{align*}

\end{document}
