\text{"Wir definieren die Geschwindigkeit } v^\mu = \gamma(c,\vec v) \text{ mit offb. } v^\mu_\mu=c^2.\\ \text{Zeigen Sie mit }
\begin{align*}
  L&=-mc\sqrt{v^\mu_\mu}-\frac{e}{c}v^\mu A_\mu
  & \tau:&=\frac{t}{\gamma}
\text{ und der Euler-Lagrange-Gleichung }
  \frac{\mathrm d}{\mathrm d\tau}\frac{\partial L}{\partial v^\mu}
  -\frac{\partial L}{\partial x^\mu}&=0
\end{align*}\\
\text{die Bewegungsgleichung [...] folgt."}\\

\text{Wie die Bewegungsgleichungen genau aussehen ist gerade nicht sehr wichtig. In der Musterl\"osung kommt der Schritt}\\
\begin{align*}
  \frac{\partial}{\partial v^\mu} m c \sqrt{v^\nu v_\nu} = m v_\mu
\end{align*}\\
\text{vor. Genau um den geht es. Den m\"ochte ich a) gerne nachvollziehen, und
b) verstehen, was mit dem Konstrukt }
\begin{align*}
  \frac{\partial}{\partial v^\mu}
\end{align*}
\text{ gemeint sein soll/sein kann/ist.}
